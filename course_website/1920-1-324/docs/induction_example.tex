\documentclass[12pt]{article}
\usepackage{amsmath,amssymb,amsthm,graphicx,multicol,enumitem}
\usepackage[margin=1in]{geometry}

\newcommand{\wtsbox}[1]{\begin{center}
		\fbox{\begin{minipage}{.9\textwidth} #1\end{minipage}}
	\end{center}\vskip .25in}
\begin{document}
	\begin{center}
		\large
		Math 324/524: Linear Algebra and Matrix Theory
		
		Induction Example
	\end{center}

	
	
	\noindent\textbf{Proposition}
		For all positive integers $n$,
		$$1+2+\cdots+n=\frac{n(n+1)}{2}.$$
	\begin{proof}\ \\
		\textbf{Base Case:} If $n=1$, then the LHS is $1+2+\cdots+n=1$ and the RHS is 
			$$\frac{n(n+1)}{2}=\frac{1(1+1)}{2}=1.$$ 
		Since the LHS and RHS are equal, the statement is true for $n=1$.
		
		\vskip .25in
		\noindent\textbf{Inductive Step:} Assume the statement is true for $n=k$.  That is, assume 
			$$1+2+\cdots+k=\frac{k(k+1)}{2}.$$
		
		\wtsbox{We want to show that the statement is true for $n=k+1$.  In particular, we want to show
				\begin{equation}1+2+\cdots+k+(k+1)=\frac{(k+1)(k+2)}{2}.\end{equation}}

		\noindent Consider the LHS of (1):
			$$\begin{array}{rcll}
			1+2+\cdots+k+(k+1)&=&\displaystyle\frac{k(k+1)}{2}+(k+1),&\textit{\small inductive hypothesis}\\\\
			&=&\displaystyle\frac{k(k+1)+2(k+1)}{2},&\textit{\small common denominator}\\\\
			&=&\displaystyle\frac{(k+1)(k+2)}{2},&\textit{\small common factor}\\\\
			&=&\text{RHS of (1)}.
			\end{array}$$
		Therefore the equality holds when $n=k+1$ and the claim is true by induction.
	\end{proof}
\end{document}