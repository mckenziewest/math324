\documentclass{beamer}
%\usepackage[margin=1in]{geometry}
\usepackage{amsthm,amsmath,amsfonts,hyperref,graphicx,color,multicol}
\usepackage{enumitem,tikz}

%%%%%%%%%%
%Beamer Template Customization
%%%%%%%%%%
\setbeamertemplate{navigation symbols}{}
\setbeamertemplate{theorems}[ams style]
\setbeamertemplate{blocks}[rounded]

\definecolor{Blu}{RGB}{43,62,133} % UWEC Blue
\setbeamercolor{structure}{fg=Blu} % Titles

%Unnumbered footnotes:
\newcommand{\blfootnote}[1]{%
	\begingroup
	\renewcommand\thefootnote{}\footnote{#1}%
	\addtocounter{footnote}{-1}%
	\endgroup
}


%%%%%%%%%%
%Custom Commands
%%%%%%%%%%
\newcommand{\R}{\mathbb{R}}
\newcommand{\veca}{\vec{a}}
\newcommand{\vecb}{\vec{b}}
\newcommand{\vece}{\vec{e}}
\newcommand{\vecu}{\vec{u}}
\newcommand{\vecv}{\vec{v}}
\newcommand{\vecw}{\vec{w}}
\newcommand{\vecx}{\vec{x}}
\newcommand{\zerovector}{\vec{0}}

\newcommand{\ds}{\displaystyle}

\newcommand{\fn}{\insertframenumber}

\newcommand{\rank}{\operatorname{rank}}
\newcommand{\adj}{\operatorname{adj}}

\newcommand{\blank}[1]{\underline{\hspace*{#1}}}


%%%%%%%%%%
%Custom Theorem Environments
%%%%%%%%%%
\theoremstyle{definition}
\newtheorem{exercise}{Exercise}
\newtheorem{question}[exercise]{Question}
\newtheorem*{defn}{Definition}
\newtheorem*{exa}{Example}
\newtheorem*{disc}{Group Discussion}
\newtheorem*{nb}{Note}
\newtheorem*{recall}{Recall}
\renewcommand{\emph}[1]{{\color{blue}\texttt{#1}}}

\definecolor{Gold}{RGB}{237, 172, 26}
%Statement block
\newenvironment{statementblock}[1]{%
	\setbeamercolor{block body}{bg=Gold!20}
	\setbeamercolor{block title}{bg=Gold}
	\begin{block}{\textbf{#1.}}}{\end{block}}





\begin{document}
	\title{Math 324: Linear Algebra}
	\subtitle{Section 4.5: Dimension}
	\author{Mckenzie West}
	\date{Last Updated: \today}
\begin{frame}
\maketitle
\end{frame}

\begin{frame}{\insertframenumber}
	\begin{block}{\textbf{Last Time.}}
	\begin{itemize}[label=--]
		\item Basis
	\end{itemize}
	\end{block}
	\begin{block}{\textbf{Today.}}
		\begin{itemize}[label=--]
			\item Dimension
		\end{itemize}
	\end{block}
\end{frame}

\begin{frame}{\fn}
	\begin{statementblock}{Theorem 4.11}
		If a vector space $V$ has a basis with $n$ vectors then every basis for $V$ has $n$ vectors.
	\end{statementblock}
	\begin{exercise}
		Fill details into this terse proof of Theorem 4.11.
		\begin{proof}
			Let $S=\{\vec v_1,\vec v_2,\dots, \vec v_n\}$ be a basis for the vector space $V$.  Let $T=\{\vec u_1,\vec u_2,\dots,\vec u_m\}$ be another basis for $V$.
			\begin{enumerate}[label=(\alph*)]
				\item Since $S$ is a basis and $T$ is linearly independent, by Theorem 4.10 we must have $m\leq n$.
				\item Similarly $n\leq m$.
			\end{enumerate}
			Therefore, $m=n$, as desired.
		\end{proof}
	\end{exercise}
\end{frame}
\begin{frame}{\fn}
	\begin{exercise}
		How many vectors must a basis for each of the following have?
		\begin{enumerate}[label=(\alph*)]
			\item $\R^2$
			\item $\R^3$
			\item $\R^n$
			\item $P_2$
			\item $P_n$
			\item $M_{2,2}$
			\item $M_{m,n}$
		\end{enumerate}
	\end{exercise}
\end{frame}
\begin{frame}{\fn}
	\begin{defn}
		If the vector space $V$ has a basis with $n$ elements, then we say that the \emph{dimension of V} is $n$, denoted $\dim(V)=n$.
		
		The vector space consisting of only the zero vector is said to have dimension 0.
	\end{defn}
\end{frame}
\begin{frame}{\fn}
	\begin{exercise}
		What is the dimension of each of the following vector spaces?
		\begin{enumerate}[label=(\alph*)]
			\item $\R^2$
			\item $\R^3$
			\item $\R^n$
			\item $P_2$
			\item $P_n$
			\item $M_{2,2}$
			\item $M_{m,n}$
		\end{enumerate}
	\end{exercise}
\end{frame}
\begin{frame}{\fn}
	\begin{exa}
		Consider the subspace $W=\{(a+b,a-b,3*a):a,b\in\R\}$ of $\R^3$.  A basis for this space is $S=\{(1,1,3),(1,-1,0)\}$ because every vector in $W$ can be written as a combination of these two 
			\[(a+b,a-b,3a)=a(1,1,3)+b(1,-1,0),\]
		and the set $S$ is linearly independent. (How would you check independence?)
		
		Thus $W$ is 2-dimensional.
	\end{exa}
	\begin{exercise}
		\begin{enumerate}[label=(\alph*)]
			\item Find a basis for the subspace $W=\{(t,2t):t\in\R\}$ of $\R^2$. What is $\dim(W)$?
			\item Find a basis for the subspace $W=\{(a,b,b,b):a,b\in\R\}$ of $\R^4$. What is $\dim(W)$?
		\end{enumerate}
	\end{exercise}
\end{frame}
\begin{frame}{\fn}
	\begin{block}{\textbf{Brain Break.}}
		If you have a tattoo, what does it mean to you? If you don’t, what tattoo would you get?
		\begin{center}
			\includegraphics[width=1in]{images/ole}
			
			\footnotesize
			This lion is on my foot.  It is the mascot for St.~Olaf College, where I got my undergraduate degree.
		\end{center}
	\end{block}
\end{frame}
\begin{frame}{\fn}
\begin{statementblock}{Theorem 4.12}
	Let $V$ be a vector space of dimension $n$.
	\begin{enumerate}[label=\textbf{\arabic*.}]
		\item If $S=\{\vec v_1,\vec v_2,\dots,\vec v_n\}$ is a linearly independent set of $n$ vectors in $V$, then $S$ is a basis for $V$. 
		\item  If $S=\{\vec v_1,\vec v_2,\dots,\vec v_n\}$ has $n$ vectors and spans $V$, then $S$ is a basis for $V$.
	\end{enumerate}
\end{statementblock}
\begin{exercise}
	Discuss this Theorem in the context of $V=\R^3$.  
	
	If $S=\{\vec v_1,\vec v_2,\vec v_3\}$ is a linearly independent set, why does $S$ also span $\R^3$?
	
	If $S=\{\vec v_1,\vec v_2,\vec v_3\}$ spans $\R^3$, why is it also linearly independent?
\end{exercise}
\end{frame}
\begin{frame}{\fn}
\begin{statementblock}{Theorem: Pruning}
	Let $S=\{\vec v_1,\vec v_2,\dots,\vec v_k\}$ be a spanning set of a vector space $V$.  
	Then there is a subset $T$ of $S$ that is a basis for $V$.
\end{statementblock}
\begin{exercise}
	Consider the set $S=\{(1,2,4),(1,3,5),(2,5,9),(0,1,3),(0,2,6)\}$.  Find a subset of $S$ that is a basis for $\R^3$.
\end{exercise}
\end{frame}
\end{document}

