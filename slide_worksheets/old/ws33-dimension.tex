\documentclass{beamer}
%\usepackage[margin=1in]{geometry}
\usepackage{amsthm,amsmath,amsfonts,hyperref,graphicx,color,multicol}
\usepackage{enumitem,tikz}

%%%%%%%%%%
%Beamer Template Customization
%%%%%%%%%%
\setbeamertemplate{navigation symbols}{}
\setbeamertemplate{theorems}[ams style]
\setbeamertemplate{blocks}[rounded]

\definecolor{Blu}{RGB}{43,62,133} % UWEC Blue
\setbeamercolor{structure}{fg=Blu} % Titles

%Unnumbered footnotes:
\newcommand{\blfootnote}[1]{%
	\begingroup
	\renewcommand\thefootnote{}\footnote{#1}%
	\addtocounter{footnote}{-1}%
	\endgroup
}


%%%%%%%%%%
%Custom Commands
%%%%%%%%%%
\newcommand{\R}{\mathbb{R}}
\newcommand{\veca}{\vec{a}}
\newcommand{\vecb}{\vec{b}}
\newcommand{\vece}{\vec{e}}
\newcommand{\vecu}{\vec{u}}
\newcommand{\vecv}{\vec{v}}
\newcommand{\vecw}{\vec{w}}
\newcommand{\vecx}{\vec{x}}
\newcommand{\zerovector}{\vec{0}}

\newcommand{\ds}{\displaystyle}

\newcommand{\fn}{\insertframenumber}

\newcommand{\col}{\operatorname{col}}
\newcommand{\row}{\operatorname{row}}
\newcommand{\rank}{\operatorname{rank}}
\newcommand{\adj}{\operatorname{adj}}

\newcommand{\blank}[1]{\underline{\hspace*{#1}}}


%%%%%%%%%%
%Custom Theorem Environments
%%%%%%%%%%
\theoremstyle{definition}
\newtheorem{exercise}{Exercise}
\newtheorem{question}[exercise]{Question}
\newtheorem*{defn}{Definition}
\newtheorem*{exa}{Example}
\newtheorem*{disc}{Group Discussion}
\newtheorem*{nb}{Note}
\newtheorem*{recall}{Recall}
\renewcommand{\emph}[1]{{\color{blue}\texttt{#1}}}

\definecolor{Gold}{RGB}{237, 172, 26}
%Statement block
\newenvironment{statementblock}[1]{%
	\setbeamercolor{block body}{bg=Gold!20}
	\setbeamercolor{block title}{bg=Gold}
	\begin{block}{\textbf{#1.}}}{\end{block}}





\begin{document}
	\title{Math 324: Linear Algebra}
	\subtitle{Section 4.5: Dimension}
	\author{Mckenzie West}
	\date{Last Updated: \today}
\begin{frame}
\maketitle
\end{frame}

\begin{frame}{\insertframenumber}
	\begin{block}{\textbf{Last Time.}}
	\begin{itemize}[label=--]
		\item Definition of a Basis
		\item Every basis for a vector space has the same number of vectors
		\item Dimension of a Vector Space
	\end{itemize}
	\end{block}
	\begin{block}{\textbf{Today.}}
		\begin{itemize}[label=--]
			\item Pruning or extending for a basis
			\item Basis for and Dimension of a Subspace
		\end{itemize}
	\end{block}
\end{frame}
\begin{frame}{\fn}
	\begin{statementblock}{Theorem 4.12}
		Let $V$ be a vector space of dimension $n$.
			\begin{enumerate}[label=\textbf{\arabic*.}]
				\item If $S=\{\vec v_1,\vec v_2,\dots,\vec v_n\}$ is a linearly independent set of $n$ vectors in $V$, then $S$ is a basis for $V$. 
				\item  If $S=\{\vec v_1,\vec v_2,\dots,\vec v_n\}$ has $n$ vectors and spans $V$, then $S$ is a basis for $V$.
			\end{enumerate}
	\end{statementblock}
	\begin{exercise}
		Discuss this Theorem in the context of $V=\R^3$.  
		
		If $S=\{\vec v_1,\vec v_2,\vec v_3\}$ is a linearly independent set, why does $S$ also span $\R^3$?
		
		If $S=\{\vec v_1,\vec v_2,\vec v_3\}$ spans $\R^3$, why is it also linearly independent?
	\end{exercise}
\end{frame}
\begin{frame}{\fn}
	\begin{statementblock}{Theorem: Pruning}
		Let $S=\{\vec v_1,\vec v_2,\dots,\vec v_k\}$ be a spanning set of a vector space $V$.  
		Then there is a subset $T$ of $S$ that is a basis for $V$.
	\end{statementblock}
	\begin{exercise}
		Consider the set $S=\{(1,2,4),(1,3,5),(2,5,9),(0,1,3),(0,2,6)\}$.  Find a subset of $S$ that is a basis for $\R^3$.
	\end{exercise}
	\begin{center}
		\includegraphics[width=1in]{images/stop}
	\end{center}
\end{frame}
\begin{frame}{\fn}
	\begin{exercise}
		Determine all subsets of $\{(1,0),(0,1),(1,1)\}$ that form a basis for $\R^2$.
	\end{exercise}
\end{frame}
\begin{frame}{\fn}
	\begin{statementblock}{Theorem: Extending}
		Let $S=\{\vec v_1,\vec v_2,\dots,\vec v_k\}$ be a linearly independent subset of a vector space $V$.  
		Then there is a set $T$ containing $S$ that is a basis for $V$.
	\end{statementblock}
	\begin{exercise}
		Consider the set $S=\{(3,0,1),(1,0,2)\}$.  
		\begin{enumerate}[label=(\alph*)]
			\item Verify that $S$ is linearly independent.
			\item Find a vector $\vec u\in\R^3$ that is not in $\mathrm{span}(S)$.
			\item Show that $T=\{(3,0,1),(1,0,2),\vec u\}$ is a basis for $\R^3$.
		\end{enumerate}
	\end{exercise}
\end{frame}


\begin{frame}{\fn}
	\begin{nb}
		If $W$ is a subspace of a vector space, a similar process can be performed.
	\end{nb}
	\begin{exercise}
		Let $W=\mathrm{span}(S)$ for $S=\{1+x+x^2,1-x+3x^2,3+x+5x^2\}$, a subspace of $P_2$.  Find a basis for $W$.  Then determine the dimension of $W$.
	\end{exercise}
	\begin{center}
		\includegraphics[width=1in]{images/stop}
	\end{center}
\end{frame}

\begin{frame}{\fn}
	\begin{exercise}
		Let $W=\mathrm{span}(S)$ for $S=\left\{\sin^2 x,\cos^2 x,\cos 2x,\sin 2x\right\}$, a subspace of $C(-\infty,\infty)$.  
		\begin{enumerate}[label=(\alph*)]
			\item Find a basis for $W$. (Hint: Look up the double angle formulas.)
			\item  What is the dimension of $W$?
			\pause
			\item Show that $\mathcal{B}=\{1,\sin^2 x,\sin 2x\}$ is also a basis for $W$.
		\end{enumerate}
	\end{exercise}
\end{frame}

\begin{frame}{\fn}
	\begin{exercise}
		Consider the subspace of $M_{2,3}$ from Quiz 4,
			\[W=\left\{\begin{bmatrix} a&b&0\\0&a+b&a-b\end{bmatrix}\ :\ a,b\in\R\right\}.\]
		\begin{enumerate}[label=(\alph*)]
			\item Show that every ``vector'' $A=\begin{bmatrix} a&b&0\\0&a+b&a-b\end{bmatrix}$ can be written as a linear combination of
				\[U=\begin{bmatrix} 1&0&0\\0&1&1\end{bmatrix}\textup{ and }
				V=\begin{bmatrix} 0&1&0\\0&1&-1\end{bmatrix}.\]
			\item Show that $S=\{U,V\}$ is linearly independent.
			\item Deduce that $S$ is a basis for $W$ and that $\dim(W)=2$.
		\end{enumerate}
	\end{exercise}
\end{frame}
\begin{frame}{\fn}
	\begin{exercise}
		Let $W=\{(3t,2t)\ :\ t\in\R\}$.
		\begin{enumerate}[label=(\alph*)]
			\item Show that $W$ is a subspace of $\R^2$.
			\item Give a geometric description of $W$.
			\item Find a basis for $W$.
			\item Determine the dimension of $W$.
		\end{enumerate}
	\end{exercise}
\end{frame}
\begin{frame}{\fn}
\begin{exercise}
	Let $W=\{(s,t,2s+3t)\ :\ s,t\in\R\}$.
	\begin{enumerate}[label=(\alph*)]
		\item Show that $W$ is a subspace of $\R^3$.
		\item Give a geometric description of $W$.
		\item Find a basis for $W$.
		\item Determine the dimension of $W$.
	\end{enumerate}
\end{exercise}
\end{frame}
%\begin{frame}{\fn}
%	\begin{defn}
%		Let $A$ be an $m\times n$ matrix.
%		\begin{enumerate}[label=\textbf{\arabic*.}]
%			\item The \emph{column space} of $A$, denoted $\col(A)$, is the subspace of $\R^m$ that is spanned by the columns of $A$.
%			\item the \emph{row space} of $A$, denoted $\row(A)$, is the subspace of $\R^n$ that is spanned by the rows of $A$.
%		\end{enumerate}
%	\end{defn}
%	\begin{exa}
%		For $A=\begin{bmatrix}
%			8 & -1 & 7 & 0 \\
%			0 & -1 & -1 & 3 \\
%			0 & -3 & -3 & 9
%			\end{bmatrix}$,
%		\begin{itemize}[label=--]
%			\item $\col(A)$ is a subspace of $\R^3$ because each column has 3 entries
%			\item  $\row(A)$ is a subspace of $\R^4$ because each row has 4 entries
%		\end{itemize}
%	\end{exa}
%\end{frame}
%\begin{frame}{\fn}
%	\begin{nb}
%		When we're looking at row and column space of a matrix, some of the work is already done for us because the vectors are already in a matrix.
%		
%		Relate the problem of finding $\row(A)$ and $\col(A)$ to exercises 2 and 5 as well as the linear independence and span questions we studied last week.
%	\end{nb}
%	\begin{exercise}
%		Let $A=\begin{bmatrix}
%		8 & -1 & 7 & 0 \\
%		0 & -1 & -1 & 3 \\
%		0 & -3 & -3 & 9
%		\end{bmatrix}$.
%		\begin{enumerate}[label=(\alph*)]
%			\item Find a basis for $\row(A)$.  What is the dimension of $\row(A)$?
%			\item Find a basis for $\col(A)$.  What is the dimension of $\col(A)$?
%		\end{enumerate}
%	\end{exercise}
%\end{frame}
%\begin{frame}{\fn}
%	\begin{exercise}
%		Let $A=\begin{bmatrix}
%		3 & 4 \\
%		3 & -3 \\
%		8 & 2 \\
%		6 & 10
%		\end{bmatrix}$.
%		\begin{enumerate}[label=(\alph*)]
%			\item Find a basis for $\row(A)$.  What is the dimension of $\row(A)$?
%			\item Find a basis for $\col(A)$.  What is the dimension of $\col(A)$?
%		\end{enumerate}
%	\end{exercise}
%\end{frame}
%\begin{frame}{\fn}
%	\begin{exercise}
%		What have you noticed about the relationship between $\dim(\row(A))$ and $\dim(\col(A))$? 
%	\end{exercise}
%\end{frame}
\end{document}

