\documentclass{beamer}
%\usepackage[margin=1in]{geometry}
\usepackage{amsthm,amsmath,amsfonts,hyperref,graphicx,color,multicol}
\usepackage{enumitem,tikz}

%%%%%%%%%%
%Beamer Template Customization
%%%%%%%%%%
\setbeamertemplate{navigation symbols}{}
\setbeamertemplate{theorems}[ams style]
\setbeamertemplate{blocks}[rounded]

\definecolor{Blu}{RGB}{43,62,133} % UWEC Blue
\setbeamercolor{structure}{fg=Blu} % Titles

%Unnumbered footnotes:
\newcommand{\blfootnote}[1]{%
	\begingroup
	\renewcommand\thefootnote{}\footnote{#1}%
	\addtocounter{footnote}{-1}%
	\endgroup
}


%%%%%%%%%%
%Custom Commands
%%%%%%%%%%
\newcommand{\R}{\mathbb{R}}
\newcommand{\veca}{\vec{a}}
\newcommand{\vecb}{\vec{b}}
\newcommand{\vece}{\vec{e}}
\newcommand{\vecu}{\vec{u}}
\newcommand{\vecv}{\vec{v}}
\newcommand{\vecw}{\vec{w}}
\newcommand{\vecx}{\vec{x}}
\newcommand{\zerovector}{\vec{0}}

\newcommand{\ds}{\displaystyle}

\newcommand{\fn}{\insertframenumber}

\newcommand{\rank}{\operatorname{rank}}
\newcommand{\adj}{\operatorname{adj}}

\newcommand{\blank}[1]{\underline{\hspace*{#1}}}


%%%%%%%%%%
%Custom Theorem Environments
%%%%%%%%%%
\theoremstyle{definition}
\newtheorem{exercise}{Exercise}
\newtheorem{question}[exercise]{Question}
\newtheorem*{defn}{Definition}
\newtheorem*{exa}{Example}
\newtheorem*{disc}{Group Discussion}
\newtheorem*{nb}{Note}
\newtheorem*{recall}{Recall}
\renewcommand{\emph}[1]{{\color{blue}\texttt{#1}}}

\definecolor{Gold}{RGB}{237, 172, 26}
%Statement block
\newenvironment{statementblock}[1]{%
	\setbeamercolor{block body}{bg=Gold!20}
	\setbeamercolor{block title}{bg=Gold}
	\begin{block}{\textbf{#1.}}}{\end{block}}





\begin{document}
	\title{Math 324: Linear Algebra}
	\subtitle{Section 4.4: Spanning Sets}
	\author{Mckenzie West}
	\date{Last Updated: \today}
\begin{frame}
\maketitle
\end{frame}

\begin{frame}{\insertframenumber}
	\begin{block}{\textbf{Last Time.}}
	\begin{itemize}[label=--]
		\item Subspaces of $\R^2$
		\item Subspaces of $\R^3$
		\item Intersections of Subspaces
	\end{itemize}
	\end{block}
	\begin{block}{\textbf{Today.}}
		\begin{itemize}[label=--]
			\item Linear Combinations
			\item Spanning Sets
			\item Spans
		\end{itemize}
	\end{block}
\end{frame}
\begin{frame}{\fn}
	Just like in the case of $\R^n$ we can consider linear combinations for general vector spaces.
	\begin{defn}
		A vector $\vec v$ in a vector space $V$ is a \emph{linear combination} of $\vec u_1,\vec u_2,\dots,\vec u_n\in V$ if there exist scalars $c_1,c_2,\dots,c_n$ such that
			\[\vec v=c_1\vec u_1+c_2\vec u_2+\cdots+c_n\vec u_n.\]
	\end{defn}
	\begin{exa}
		The polynomial $p(x)=3+x-2x^2$ is a linear combination of $q_1(x)=1+x$, $q_2(x)=1+x^2$, $q_3(x)=1+x+x^2$, and $q_4(x)=x^2$ because \begin{eqnarray*}p(x)&=&3+x-2x^2\\ &=&5(1+x)+2(1+x^2)-4(1+x+x^2)+0x^2\\&=&5q_1(x)+2q_2(x)+(-4)q_3(x)+0q_4(x)\end{eqnarray*}
	\end{exa}
\end{frame}
\begin{frame}{\fn}
	\begin{exercise}
		Can the polynomial $p(x)=2+x+x^2$ be written as a linear combination of $q_1(x)=1+x$ and $q_2(x)=1+x^2$?
	\end{exercise}
	\begin{exercise}
		Can the polynomial $p(x)=1-4x+x^2$ be written as a linear combination of $q_1(x)=1+x$ and $q_2(x)=1+x^2$?
	\end{exercise}
%	\begin{exercise}
%		Describe all of the polynomials can be written as a linear combination of $q_1(x)=1+x$ and $q_2(x)=1+x^2$.
%	\end{exercise}
\end{frame}
\begin{frame}{\fn}
	\begin{exercise}
		Can $A = \begin{bmatrix}1&2\\3&4\end{bmatrix}$ be written as a linear combination of $M_1=\begin{bmatrix}1&1\\0&0\end{bmatrix}$, $M_2=\begin{bmatrix}1&0\\0&1\end{bmatrix}$, and $M_3=\begin{bmatrix}0&0\\1&1\end{bmatrix}$?
	\end{exercise}
	\begin{exercise}
		Can $A = \begin{bmatrix}1&1\\1&1\end{bmatrix}$ be written as a linear combination of $M_1=\begin{bmatrix}1&1\\0&0\end{bmatrix}$, $M_2=\begin{bmatrix}1&0\\0&1\end{bmatrix}$, and $M_3=\begin{bmatrix}0&0\\1&1\end{bmatrix}$?
	\end{exercise}
\end{frame}
\begin{frame}{\fn}
	\begin{defn}
		If every vector a vector space $V$ can be written as a linear combination of the vectors in some set $S=\{\vec v_1,\vec v_2,\dots,\vec v_n\}$ then we call $S$ a \emph{spanning set} of $V$.
		In this case, we say $S$ \emph{spans} $V$.
	\end{defn}
	\begin{exa}
		The set $S=\{\vec e_1,\vec e_2,\vec e_3\}$, where $\vec e_1=(1,0,0)$, $\vec e_2=(0,1,0)$, and $\vec e_3=(0,0,1)$, is a spanning set of $\R^3$.  That is, any vector $\vec v\in \R^3$ has the form $\vec v=(v_1,v_2,v_3)=v_1(1,0,0)+v_2(0,1,0)+v_3(0,0,1)$.
	\end{exa}\begin{exa} 	
		Similarly, $S=\{1,x,x^2\}$ is a spanning set of $P_2$ because any polynomial function $p(x)$ in $P_2$ can be written as $p(x)=a+bx+cx^2 =a(1)+b(x)+c(x^2).$
	\end{exa}
		These are the \emph{standard spanning sets} of $\R^3$ and $P_2$.
\end{frame}
\begin{frame}{\fn}
	\begin{exa}
		Other things can be spanning sets.  For example, I claim that $\{(1,0,1),(1,1,0),(0,1,1)\}$ is a spanning set of $\R^3$.  To verify, consider a generic $(x,y,z)\in\R^3$.  Write the desired equation:
			\[(x,y,z)=a(1,0,1)+b(1,1,0)+c(0,1,1)\]
		Then solve for $a,b,c$:
			$$\begin{array}{rcl}\left.\begin{array}{rcrcrcl}
				1a&+&1b&&&=&x\\
				&&1b&+&1c&=&y\\
				1a&&&+&1c&=&z
			\end{array}\right\}&\rightarrow&
			\begin{bmatrix}1&1&0&x\\0&1&1&y\\1&0&1&z\end{bmatrix}
			\\&\rightarrow&
			\begin{bmatrix}
				2 & 0 & 0 & x - y + z \\
				0 & 2 & 0 & x + y - z \\
				0 & 0 & 2 & -x + y + z
			\end{bmatrix}
			\end{array}$$
		Therefore, $(x,y,z)=\frac{1}{2}\left[(x-y+z)(1,0,1)+(x+y-z)(1,1,0)+(-x+y+z)(0,1,1)\right]$
	\end{exa}
\end{frame}
\begin{frame}{\fn}
	\begin{exercise}
		Using the formula (repeated below) from the previous example, write $(1,2,3)$ as a linear combination of $(1,0,1),(1,1,0)$ and $(0,1,1)$:
		
		 {\footnotesize $ (x,y,z)~=~\frac{1}{2}\left[(x-y+z)(1,0,1)+(x+y-z)(1,1,0)+(-x+y+z)(0,1,1)\right]$}
	\end{exercise}
	\begin{nb}
		If we are only checking whether or not a set spans $S$, we simply have to make sure there is always a solution.  That is, check that the rref of the coefficient matrix has no zero rows.
	\end{nb}
\end{frame}
\begin{frame}{\fn}
	\begin{exercise}
		Does the set $S=\{(2,0),(-1,1),(3,1)\}$ span $\R^2$?  If not, describe all vectors that can be written as a combination of the vectors in $S$.
	\end{exercise}
	\begin{exercise}
		Does the set $S=\{(2,0),(-1,1)\}$ span $\R^2$? If not, describe all vectors that can be written as a combination of the vectors in $S$.
	\end{exercise}
	\begin{exercise}
	Does the set $S=\{(1,-1,0),(-1,1,0),(0,1,1)\}$ span $\R^3$? If not, describe all vectors that can be written as a combination of the vectors in $S$.
	\end{exercise}
\end{frame}
\begin{frame}{\fn}
	\begin{exercise}
		Does the set $S=\{1+x,1+x^2\}$ span $P_2$?  If not, describe all polynomials that can be written as a combination of the vectors in $S$.
	\end{exercise}
	\begin{exercise}
		Does the set $S=\left\{\begin{bmatrix}1&1\\0&0\end{bmatrix},\begin{bmatrix}1&0\\0&1\end{bmatrix},\begin{bmatrix}0&0\\1&1\end{bmatrix},\begin{bmatrix}0&1\\1&0\end{bmatrix}\right\}$ span $M_2$?  If not, describe all matrices that can be written as a combination of the vectors in $S$.
	\end{exercise}
\end{frame}
\begin{frame}{\fn}
	\begin{defn}
		If $S=\{\vec v_1,\vec v_2,\dots,\vecv_k\}$ is a set of vectors in a vector space $V$, then the \emph{span of $S$}, denoted $\operatorname{span}(S)$, is the set of all linear combinations of the vectors in $S$,
			\[\operatorname{span}(S)=\{c_1\vec v_1+c_2\vec v_2+\cdots+c_k\vec v_k\ :\ c_1,c_2,\dots,c_k\in\R\}.\]
	\end{defn}
	\begin{nb}
		In fact, $S$ spans $V$ exactly when $\operatorname{span}(S)=V$.
	\end{nb}
\end{frame}
\begin{frame}{\fn}
	\begin{exercise}
		Compute $\operatorname{span}(S)$ for $S=\{(1,2,3),(4,5,6)\}$ as a subset of $\R^3$. That is, describe all vectors in the span.
	\end{exercise}
	\begin{exercise}
		Compute $\operatorname{span}(S)$ for $S=\{1+x,1-x\}$ as a subset of $P_1$.
	\end{exercise}
\end{frame}
\begin{frame}{\fn}
	\begin{statementblock}{Theorem 4.7}
		If $S=\{\vec v_1,\vec v_2,\dots,\vec v_k\}$ is a set of vectors in a vector space $V$, then $\operatorname{span}(S)$ is a subspace of $V$.
		
		Moreover, $\operatorname{span}(S)$ is the \emph{smallest} subspace of $V$ that contains~$S$.  This means that if $W$ is any other subspace of $V$ that contains $S$, then $W$ also contains $\operatorname{span}(S)$.
	\end{statementblock}
\end{frame}
\begin{frame}{\fn}	
	\begin{exercise}
		Prove that $\operatorname{span}(S)$ is a subspace of $V$. - Write your scratch work on a whiteboard so I can see it as I walk around.
		\begin{proof}
			Let $S=\{\vec v_1,\vec v_2,\dots,\vec v_n\}$ be a subset of a vector space $V$. To show that $\operatorname{span}(S)$ is a subspace of $V$, we must verify the subspace properties:
			\begin{enumerate}[label=(\alph*)]
				\item Write $\vec 0$ as a linear combination of the vectors in $S$. *Be sure to reference any previous Theorems.*
				\item Let $\vec u$ and $\vec w$ be elements of $\operatorname{span}(S)$.  Then we can write $\vec u=c_1\vec v_1+c_2\vec v_2+\cdots+c_k\vec v_k$ and $\vec w=d_1\vec v_1+d_2\vec v_2+\cdots +c_k\vec v_k$.  Show that $\vec u+\vec w$ can be written as a linear combination of the vectors in $S$.
				\item Let $c$ be a scalar.  Show that $c\vec u$ can be written as a linear combination of the scalars in $S$.
			\end{enumerate}
		\end{proof}
	\end{exercise}
\end{frame}
\begin{frame}{\fn}
	\begin{exercise}
		Let $S=\{(1,2,3),(4,5,6)\}$ we now know that $\operatorname{span}(S)$ is a subspace of $\R^3$ but is not equal to all of $\R^3$, what could it be, geometrically?
	\end{exercise}
\end{frame}
\begin{frame}{\fn}
	One thing we want to avoid in a spanning set is redundancies. A very easy redundancy check is for non-trivial solutions to
		\[c_1\vec v_1+c_2\vec v_2+\cdots + c_k\vec v_k=\vec 0.\]
	\begin{defn}
		A subset $S=\{\vec v_1,\vec v_2,\dots,\vec v_k\}$ of a vector space $V$ is called \emph{linearly independent} if the vector equation
			\[c_1\vec v_1+c_2\vec v_2+\cdots+c_k\vec v_k=\vec 0\]
		has only the trivial solution,
			\[c_1=0,c_2=0,\dots,c_k=0.\]
		If there are non-trivial solutions, then we call $S$ \emph{linear dependent}.
	\end{defn}
\end{frame}
\begin{frame}{\fn}
	\begin{exercise}
		Is the set $S=\{(1,2),(3,4)\}$ linear independent?
	\end{exercise}
	\begin{exercise}
		Is the set $S=\{(1,2),(3,4), (5,6)\}$ linearly independent?
	\end{exercise}
	\begin{exercise}
		Is the set $S=\{(1,2,0),(3,0,4), (0,5,6)\}$ linearly independent?
	\end{exercise}
	\begin{exercise}
		Is the set $S=\{1,1+x,1+x+x^2\}$ linearly independent?
	\end{exercise}
	\begin{exercise}
		Is the set $S=\left\{\begin{bmatrix}1&2&0\\0&1&1\end{bmatrix},\begin{bmatrix}0&0&1\\1&1&1\end{bmatrix}\right\}$ linearly independent?
	\end{exercise}
\end{frame}
\end{document}

