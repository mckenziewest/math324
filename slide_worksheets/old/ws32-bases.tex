\documentclass{beamer}
%\usepackage[margin=1in]{geometry}
\usepackage{amsthm,amsmath,amsfonts,hyperref,graphicx,color,multicol}
\usepackage{enumitem,tikz}

%%%%%%%%%%
%Beamer Template Customization
%%%%%%%%%%
\setbeamertemplate{navigation symbols}{}
\setbeamertemplate{theorems}[ams style]
\setbeamertemplate{blocks}[rounded]

\definecolor{Blu}{RGB}{43,62,133} % UWEC Blue
\setbeamercolor{structure}{fg=Blu} % Titles

%Unnumbered footnotes:
\newcommand{\blfootnote}[1]{%
	\begingroup
	\renewcommand\thefootnote{}\footnote{#1}%
	\addtocounter{footnote}{-1}%
	\endgroup
}


%%%%%%%%%%
%Custom Commands
%%%%%%%%%%
\newcommand{\R}{\mathbb{R}}
\newcommand{\veca}{\vec{a}}
\newcommand{\vecb}{\vec{b}}
\newcommand{\vece}{\vec{e}}
\newcommand{\vecu}{\vec{u}}
\newcommand{\vecv}{\vec{v}}
\newcommand{\vecw}{\vec{w}}
\newcommand{\vecx}{\vec{x}}
\newcommand{\zerovector}{\vec{0}}

\newcommand{\ds}{\displaystyle}

\newcommand{\fn}{\insertframenumber}

\newcommand{\rank}{\operatorname{rank}}
\newcommand{\adj}{\operatorname{adj}}

\newcommand{\blank}[1]{\underline{\hspace*{#1}}}


%%%%%%%%%%
%Custom Theorem Environments
%%%%%%%%%%
\theoremstyle{definition}
\newtheorem{exercise}{Exercise}
\newtheorem{question}[exercise]{Question}
\newtheorem*{defn}{Definition}
\newtheorem*{exa}{Example}
\newtheorem*{disc}{Group Discussion}
\newtheorem*{nb}{Note}
\newtheorem*{recall}{Recall}
\renewcommand{\emph}[1]{{\color{blue}\texttt{#1}}}

\definecolor{Gold}{RGB}{237, 172, 26}
%Statement block
\newenvironment{statementblock}[1]{%
	\setbeamercolor{block body}{bg=Gold!20}
	\setbeamercolor{block title}{bg=Gold}
	\begin{block}{\textbf{#1.}}}{\end{block}}





\begin{document}
	\title{Math 324: Linear Algebra}
	\subtitle{Section 4.5: Basis}
	\author{Mckenzie West}
	\date{Last Updated: \today}
\begin{frame}
\maketitle
\end{frame}

\begin{frame}{\insertframenumber}
	\begin{block}{\textbf{Last Time.}}
	\begin{itemize}[label=--]
		\item Linear Independence
	\end{itemize}
	\end{block}
	\begin{block}{\textbf{Today.}}
		\begin{itemize}[label=--]
			\item Definition of a Basis
			\item Finding Bases
		\end{itemize}
	\end{block}
\end{frame}
\begin{frame}{\fn}
	\begin{defn}
		A set of vectors $S=\{\vec v_1,\vec v_2,\dots,\vec v_n\}$ in a vector space $V$ is called a \emph{basis} for $V$ if both (\textbf{1}) $S$ spans $V$ and (\textbf{2}) $S$ is linearly independent.
	\end{defn}
	\begin{nb}
		Bases (base-eaze) are ``Goldilocks sets'' in that they aren't too small (\textbf{1}) and they aren't too big (\textbf{2}).
		
		Recall that for out usual vector spaces,
			\begin{center}
				\begin{tabular}{rcl} 
					spanning &$\longleftrightarrow$& no zero rows in RREF\\
					linearly independent &$\longleftrightarrow$& no free variables
			\end{tabular}
			\end{center}
	\end{nb}
\end{frame}
\begin{frame}{\fn}
	\begin{exercise}
		Show that $S=\{(1,0,0),(0,1,0),(0,0,1)\}$ is a basis for $\R^3$. 
	\end{exercise}\pause
	\begin{defn}
		The basis $S=\{(1,0,0),(0,1,0),(0,0,1)\}$ is called the \emph{standard basis for $\R^3$}.  The \emph{standard basis for $\R^n$} is the set $S=\{\vec e_1,\vec e_2,\dots,\vec e_n\}$ of vectors where,
			$$\begin{array}{rcl}
				\vec e_1&=&(1,0,\dots,0)\\
				\vec e_2 &=&(0,1,\dots,0)\\
				&\vdots\\
				\vec e_n&=&(0,0,\dots,1).
			\end{array}$$
	\end{defn}
\end{frame}
\begin{frame}{\fn}
	\begin{exercise}\label{exercise:basisforR2}
		Show that $S=\{(1,2),(5,-2)\}$ is a basis for $\R^2$.
	\end{exercise}
	\pause
	\begin{exercise}
		Which of the following are bases for $\R^3$? Explain why or why not.
		\begin{enumerate}[label=(\alph*)]
			\item $\{(1,2,3),(4,5,6),(7,8,9)\}$
			\item $\{(1,0,1),(1,1,0),(0,1,1),(1,1,1)\}$
			\item $\{(4,0,2),(1,1,3)\}$
			\item $\{(1,1,1),(1,1,2),(0,1,0)\}$
		\end{enumerate}
	\end{exercise}
\end{frame}
\begin{frame}{\fn}
	\begin{exercise}
		Show that $S=\{1,x,x^2,x^3\}$ is a basis for $P_3$, called the \emph{standard basis for $P_3$}.
	\end{exercise}\pause
	\begin{exa}
		The \emph{standard basis for $M_{2,2}$} is the set 
			\[S=\left\{\begin{bmatrix}1&0\\0&0\end{bmatrix},\begin{bmatrix}0&1\\0&0\end{bmatrix},\begin{bmatrix}0&0\\1&0\end{bmatrix},\begin{bmatrix}0&0\\0&1\end{bmatrix}\right\}.\]
	\end{exa}
	\begin{exercise}
		What do you think the standard basis for $M_{m,n}$ is?
	\end{exercise}
\end{frame}
\begin{frame}{\fn}
	\begin{defn}
		If a basis exists for a vector space, we call it \emph{finite dimensional} otherwise a vector space is said to be \emph{infinite dimensional}.
	\end{defn}
	\begin{exa}
		The vector space $P$ of all polynomials is infinite dimensional, as is $C(-\infty,\infty)$ the collection of all real valued continuous functions.
	\end{exa}
\end{frame}
\begin{frame}{\fn}
	\begin{statementblock}{Theorem 4.9}
		If $S=\{\vec v_1,\vec v_2,\dots, \vec v_n\}$ is a basis for a vector space $V$, then every vector in $V$ can be written in one and only one way as a linear combination of the vectors in $S$.
	\end{statementblock}
	\vskip .25in
	\begin{center}
		\includegraphics[width=1in]{images/stop}
	\end{center}
\end{frame}
\begin{frame}{\fn}
	\begin{exercise}
		Verify that every vector $\vec u=(x,y)$ in $\R^2$ can be written uniquely in terms of the vectors in the basis $S=\{(1,2),(5,-2)\}$ from exercise~\ref{exercise:basisforR2}.  That is, show there is a unique solution to \[c_1(1,2)+c_2(5,-2)=(x,y),\]
		no matter the value of $x$ and $y$.  What is that representation?
	\end{exercise}
\end{frame}
\begin{frame}{\fn}
	\begin{statementblock}{Theorem 4.10}
		If $S=\{\vec v_1,\vec v_2,\dots,\vec v_n\}$ is a basis for a vector space $V$, then every set containing more than $n$ vectors in $V$ is linearly dependent.
	\end{statementblock}
	\begin{nb}
		The proof of Theorem 4.10 is available in the book.  It relies on the fact that you can write the vectors in the set $\{\vec u_1,\vec u_2,\dots,\vec u_m\}$ $(m>n)$ in terms of the vectors in the basis $S$ and insert them into the equation $k_1\vec u_1+k_2\vec u_2+\cdots+k_m\vec u_m=\vec 0$.  Ultimately you will get a homogeneous system with more variable then equations, so there is a nontrivial solution.
	\end{nb}
\end{frame}
\begin{frame}{\fn}
	\begin{exercise}
		Explain by looking, no computations only references to theorems, why the following are linearly dependent:
		\begin{enumerate}[label=(\alph*)]
			\item $S=\{(1,1,0),(1,0,1),(0,1,1),(1,1,1)\}$ in $\R^3$
			\item $S=\{x-1,x+1,x^2-1,x^2+1\}$ in $P_2$
			\item $S=\{(6,2),(18,6)\}$ in $\R^2$
			\item $S=\left\{\begin{bmatrix}1&2\\3&4\end{bmatrix},\begin{bmatrix}2&1\\3&4\end{bmatrix},\begin{bmatrix}1&2\\4&3\end{bmatrix},\begin{bmatrix}2&1\\4&3\end{bmatrix},\begin{bmatrix}3&4\\1&2\end{bmatrix}\right\}$.
		\end{enumerate}
	\end{exercise}
\end{frame}
\begin{frame}{\fn}
	\begin{statementblock}{Theorem 4.11}
		If a vector space $V$ has a basis with $n$ vectors then every basis for $V$ has $n$ vectors.
	\end{statementblock}
\begin{exercise}
Fill in any missing details in the proof of Theorem 4.11.
\begin{proof}
	Let $S=\{\vec v_1,\vec v_2,\dots, \vec v_n\}$ be a basis for the vector space $V$.  Let $T=\{\vec u_1,\vec u_2,\dots,\vec u_m\}$ be another basis for $V$.
	\begin{enumerate}[label=(\alph*)]
		\item Since $S$ is a basis and $T$ is linearly independent, by Theorem 4.10 we must have $m\leq n$.
		\item Similarly $n\leq m$.
	\end{enumerate}
	Therefore, $m=n$, as desired.
\end{proof}
\end{exercise}
\end{frame}
\begin{frame}{\fn}
	\begin{exercise}
		How many vectors must a basis for each of the following have?
		\begin{enumerate}[label=(\alph*)]
			\item $\R^2$
			\item $\R^3$
			\item $\R^n$
			\item $P_2$
			\item $P_n$
			\item $M_{2,2}$
			\item $M_{m,n}$
		\end{enumerate}
	\end{exercise}
\end{frame}
\begin{frame}{\fn}
	\begin{defn}
		If the vector space $V$ has a basis with $n$ elements, then we say that the \emph{dimension of V} is $n$, denoted $\dim(V)=n$.
		
		The vector space consisting of only the zero vector is said to have dimension 0.
	\end{defn}
\end{frame}
\begin{frame}{\fn}
\begin{exercise}
	What is the dimension of each of the following vector spaces?
	\begin{enumerate}[label=(\alph*)]
		\item $\R^2$
		\item $\R^3$
		\item $\R^n$
		\item $P_2$
		\item $P_n$
		\item $M_{2,2}$
		\item $M_{m,n}$
	\end{enumerate}
\end{exercise}
\end{frame}
\begin{frame}{\fn}
	\begin{exercise}
		Determine whether the following statements are True or False.  Explain your answer.
		\begin{enumerate}[label=(\alph*)]
			\item If $\dim(V)=n$, then there is a set of $n-1$ vectors in $V$ that spans $V$.
			\item If $\dim(V)=n$, then there is a set of $n+1$ vectors in $V$ that spans $V$.
			\item If $\dim(V)=n$, then there is a set of $n-1$ vectors in $V$ that is linearly independent.
			\item If $\dim(V)=n$, then there is a set of $n+1$ vectors in $V$ that is linearly dependent.
		\end{enumerate}
	\end{exercise}
\end{frame}
\end{document}

