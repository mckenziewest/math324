\documentclass{beamer}
%\usepackage[margin=1in]{geometry}
\usepackage{amsthm,amsmath,amsfonts,hyperref,graphicx,color,multicol}
\usepackage{enumitem,tikz}

%%%%%%%%%%
%Beamer Template Customization
%%%%%%%%%%
\setbeamertemplate{navigation symbols}{}
\setbeamertemplate{theorems}[ams style]
\setbeamertemplate{blocks}[rounded]

\definecolor{Blu}{RGB}{43,62,133} % UWEC Blue
\setbeamercolor{structure}{fg=Blu} % Titles

%Unnumbered footnotes:
\newcommand{\blfootnote}[1]{%
	\begingroup
	\renewcommand\thefootnote{}\footnote{#1}%
	\addtocounter{footnote}{-1}%
	\endgroup
}


%%%%%%%%%%
%Custom Commands
%%%%%%%%%%
\newcommand{\R}{\mathbb{R}}
\newcommand{\veca}{\vec{a}}
\newcommand{\vecb}{\vec{b}}
\newcommand{\vece}{\vec{e}}
\newcommand{\vecu}{\vec{u}}
\newcommand{\vecv}{\vec{v}}
\newcommand{\vecw}{\vec{w}}
\newcommand{\vecx}{\vec{x}}
\newcommand{\zerovector}{\vec{0}}

\newcommand{\ds}{\displaystyle}

\newcommand{\fn}{\insertframenumber}

\newcommand{\rank}{\operatorname{rank}}
\newcommand{\adj}{\operatorname{adj}}

\newcommand{\blank}[1]{\underline{\hspace*{#1}}}


%%%%%%%%%%
%Custom Theorem Environments
%%%%%%%%%%
\theoremstyle{definition}
\newtheorem{exercise}{Exercise}
\newtheorem{question}[exercise]{Question}
\newtheorem*{defn}{Definition}
\newtheorem*{exa}{Example}
\newtheorem*{disc}{Group Discussion}
\newtheorem*{nb}{Note}
\newtheorem*{recall}{Recall}
\renewcommand{\emph}[1]{{\color{blue}\texttt{#1}}}

\definecolor{Gold}{RGB}{237, 172, 26}
%Statement block
\newenvironment{statementblock}[1]{%
	\setbeamercolor{block body}{bg=Gold!20}
	\setbeamercolor{block title}{bg=Gold}
	\begin{block}{\textbf{#1.}}}{\end{block}}





\begin{document}
	\title{Math 324: Linear Algebra}
	\subtitle{Section 4.4: Spans and Linear Independence}
	\author{Mckenzie West}
	\date{Last Updated: \today}
\begin{frame}
\maketitle
\end{frame}

\begin{frame}{\insertframenumber}
	\begin{block}{\textbf{Last Time.}}
	\begin{itemize}[label=--]
		\item Linear Combinations
		\item Spanning Sets
		\item Spans
	\end{itemize}
	\end{block}
	\begin{block}{\textbf{Today.}}
		\begin{itemize}[label=--]
			\item Spans
			\item Linear Independence
		\end{itemize}
	\end{block}
\end{frame}
\begin{frame}{\fn}
	\vskip -.35in
	\begin{exa}
		Recall from last time that $\{(1,-1,0),(-1,1,0),(0,1,1)\}$ does not span $\R^3$.  To verify, consider a generic $(x,y,z)\in\R^3$.  Write the desired equation:
			\[(x,y,z)=a(1,-1,0)+b(-1,1,0)+c(0,1,1)\]
		Then solve for $a,b,c$:
			$$\begin{array}{rcl}\left.\begin{array}{rcrcrcl}
				1a&-&1b&&&=&x\\
				-1a&+&1b&+&1c&=&y\\
				&&&&1c&=&z
			\end{array}\right\}&\rightarrow&
			\begin{bmatrix}1&-1&0&x\\-1&1&1&y\\0&0&1&z\end{bmatrix}
			\\&\xrightarrow{rref}&
			\begin{bmatrix}
				1 & -1 & 0 & x \\
				0 & 0 & 1 & x + y\\
				0 & 0 & 0 & z-x -y
			\end{bmatrix}
			\end{array}$$
		Therefore, there is only a solution for $a,b,c$ if $z-x-y=0$.  Given there is a restriction, there is no way for this set to span all of $\R^3$.  In particular there is no way to write, say $(0,0,1)$ as a linear combination of the vectors in $S$.
	\end{exa}
\end{frame}
\begin{frame}[fragile]
	\frametitle{\fn}
	\begin{exercise}
		Continuing, open \url{https://sagecell.sagemath.org/} and use the code below to plot the plane $z-x-y=0$ along with the vectors in $S=\{(1,-1,0),(-1,1,0),(0,1,1)\}$.
		
		\begin{verbatim}
			x,y = var("x,y")
			P = plot3d(x+y,(x,-2,2),(y,-2,2),opacity=.5)
			P += arrow((0,0,0),(1,-1,0),color="red",thickness=5)
			P += arrow((0,0,0),(-1,1,0),color="green",thickness=5)
			P += arrow((0,0,0),(0,1,1),color="black",thickness=5)
			P.show()
		\end{verbatim}
	\end{exercise}
\end{frame}
\begin{frame}{\fn}
	\begin{defn}
		If $S=\{\vec v_1,\vec v_2,\dots,\vecv_k\}$ is a set of vectors in a vector space $V$, then the \emph{span of $S$}, denoted $\operatorname{span}(S)$, is the set of all linear combinations of the vectors in $S$,
			\[\operatorname{span}(S)=\{c_1\vec v_1+c_2\vec v_2+\cdots+c_k\vec v_k\ :\ c_1,c_2,\dots,c_k\in\R\}.\]
	\end{defn}
	\begin{nb}
		In fact, $S$ spans $V$ exactly when $\operatorname{span}(S)=V$.
	\end{nb}
\end{frame}
\begin{frame}[fragile]
	\frametitle{\fn}
	\begin{exercise}
		Compute $\operatorname{span}(S)$ for $S=\{(1,2,3),(4,5,6)\}$ as a subset of $\R^3$. That is, describe all vectors in the span.
		
		Use the following code to row reduce a matrix with variables in Sage: (\url{https://sagecell.sagemath.org/})
		\begin{verbatim}
		R.<x,y,z> = QQ[]
		A = matrix([[1,4,x],[2,5,y],[3,6,z]])
		print A.echelon_form()
		\end{verbatim}
	\end{exercise}
	\begin{exercise}
		Compute $\operatorname{span}(S)$ for $S=\{1+x,1-x\}$ as a subset of $P_1$.
	\end{exercise}
\end{frame}
\begin{frame}{\fn}
	\begin{statementblock}{Theorem 4.7}
		If $S=\{\vec v_1,\vec v_2,\dots,\vec v_k\}$ is a set of vectors in a vector space $V$, then $\operatorname{span}(S)$ is a subspace of $V$.
		
		Moreover, $\operatorname{span}(S)$ is the \emph{smallest} subspace of $V$ that contains~$S$.  This means that if $W$ is any other subspace of $V$ that contains $S$, then $W$ also contains $\operatorname{span}(S)$.
	\end{statementblock}
	\begin{center}
		\includegraphics[width=1.5in]{images/stop}
	\end{center}
\end{frame}
%\begin{frame}{\fn}	
%	\begin{exercise}
%		Prove that $\operatorname{span}(S)$ is a subspace of $V$. - Write your scratch work on a whiteboard so I can see it as I walk around.
%		\begin{proof}
%			Let $S=\{\vec v_1,\vec v_2,\dots,\vec v_n\}$ be a subset of a vector space $V$. To show that $\operatorname{span}(S)$ is a subspace of $V$, we must verify the subspace properties:
%			\begin{enumerate}[label=(\alph*)]
%				\item Write $\vec 0$ as a linear combination of the vectors in $S$. *Be sure to reference any previous Theorems.*
%				\item Let $\vec u$ and $\vec w$ be elements of $\operatorname{span}(S)$.  Then we can write $\vec u=c_1\vec v_1+c_2\vec v_2+\cdots+c_k\vec v_k$ and $\vec w=d_1\vec v_1+d_2\vec v_2+\cdots +d_k\vec v_k$.  Show that $\vec u+\vec w$ can be written as a linear combination of the vectors in $S$.
%				\item Let $c$ be a scalar.  Show that $c\vec u$ can be written as a linear combination of the scalars in $S$.
%			\end{enumerate}
%		\end{proof}
%	\end{exercise}
%\end{frame}
\begin{frame}{\fn}
	\begin{exercise}
		Let $S=\{(1,2,3),(4,5,6)\}$ we now know that $\operatorname{span}(S)$ is a subspace of $\R^3$ but is not equal to all of $\R^3$, what could it be, geometrically?
	\end{exercise}
\end{frame}
\begin{frame}{\fn}
	One thing we want to avoid in a spanning set is redundancies - as in the first example today. 
	
	A very easy redundancy check is for non-trivial solutions to:
		\[c_1\vec v_1+c_2\vec v_2+\cdots + c_k\vec v_k=\vec 0.\]
	\begin{defn}
		A subset $S=\{\vec v_1,\vec v_2,\dots,\vec v_k\}$ of a vector space $V$ is called \emph{linearly independent} if the vector equation
			\[c_1\vec v_1+c_2\vec v_2+\cdots+c_k\vec v_k=\vec 0\]
		has only the trivial solution,
			\[c_1=0,\ c_2=0,\ \dots,\ c_k=0.\]
		If there are non-trivial solutions, then we call $S$ \emph{linearly~dependent}.
	\end{defn}
\end{frame}
\begin{frame}{\fn}
	\begin{exa}
		Is the set $S=\{1+x,1-x,x^2,1+2x+3x^2\}$ linearly independent?
		
		Consider the homogeneous system:
			\[a(1+x)+b(1-x)+cx^2+d(1+2x+3x^2)=0.\]
		Then solve for $a,b,c,d$ by considering each power of $x$:
			$$\begin{array}{rcl}\left.\begin{array}{rcrcrcrcl}
			a&+&b&&&+&d&=&0\\
			a&-&b&&&+&2d&=&0\\
			&&&&c&+&3d&=&0
			\end{array}\right\}&\rightarrow&
			\begin{bmatrix}1 & 1 & 0 & 1 & 0 \\
			1 & -1 & 0 & 2 & 0 \\
			0 & 0 & 1 & 3 & 0\end{bmatrix}
			\\&\xrightarrow{rref}&
			\begin{bmatrix}
				1 & 0 & 0 & \frac{3}{2} & 0 \\
				0 & 1 & 0 & -\frac{1}{2} & 0 \\
				0 & 0 & 1 & 3 & 0
			\end{bmatrix}.
			\end{array}$$
		Since there is a free variable in the fourth column, the set is linearly \textbf{dependent}.
	\end{exa}
\end{frame}
\begin{frame}{\fn}
	\begin{nb}
		In each of the examples today, we're essentially putting the elements of $S$ into columns of matrices and row reducing.  Our conclusions are as follows:
			\begin{itemize}[label=--]
				\item Zero Row in Coefficient Matrix $\Rightarrow$ Does NOT Span
				\item Free Variable (column w/o a leading 1) $\Rightarrow$ Linearly Dependent
			\end{itemize}
	\end{nb}
\end{frame}
\begin{frame}{\fn}
	\begin{exercise}
		\begin{enumerate}[label = (\alph*)]\setlength{\itemsep}{.15in}
			\item Is the set $S=\{(1,2),(3,4)\}$ linear independent? Does $S$ span $\R^2$?
			\item Is the set $S=\{(1,2),(3,4), (5,6)\}$ linearly independent? Does $S$ span $\R^2$?
			\item Is the set $S=\{(1,2,0),(3,0,4), (4,2,4)\}$ linearly independent? Does $S$ span $\R^3$?
			\item Is the set $S=\{1,1+x,1+x+x^2\}$ linearly independent? Does $S$ span $P_2$?
			\item Is the set $S=\left\{\begin{bmatrix}1&2&0\\0&1&1\end{bmatrix},\begin{bmatrix}0&0&1\\1&1&1\end{bmatrix}\right\}$ linearly independent? Does $S$ span $M_{2,3}$?
		\end{enumerate}
	\end{exercise}
\end{frame}
\end{document}

