\documentclass[handout]{beamer}
%\usepackage[margin=1in]{geometry}
\usepackage{amsthm,amsmath,amsfonts,hyperref,graphicx,color,multicol}
\usepackage{enumitem,tikz}

%%%%%%%%%%
%Beamer Template Customization
%%%%%%%%%%
\setbeamertemplate{navigation symbols}{}
\setbeamertemplate{theorems}[ams style]
\setbeamertemplate{blocks}[rounded]

\definecolor{Blu}{RGB}{43,62,133} % UWEC Blue
\setbeamercolor{structure}{fg=Blu} % Titles

%Unnumbered footnotes:
\newcommand{\blfootnote}[1]{%
	\begingroup
	\renewcommand\thefootnote{}\footnote{#1}%
	\addtocounter{footnote}{-1}%
	\endgroup
}


%%%%%%%%%%
%Custom Commands
%%%%%%%%%%
\newcommand{\R}{\mathbb{R}}
\newcommand{\veca}{\vec{a}}
\newcommand{\vecb}{\vec{b}}
\newcommand{\vece}{\vec{e}}
\newcommand{\vecu}{\vec{u}}
\newcommand{\vecv}{\vec{v}}
\newcommand{\vecw}{\vec{w}}
\newcommand{\vecx}{\vec{x}}
\newcommand{\zerovector}{\vec{0}}

\newcommand{\ds}{\displaystyle}

\newcommand{\fn}{\insertframenumber}

\newcommand{\rank}{\operatorname{rank}}
\newcommand{\adj}{\operatorname{adj}}

\newcommand{\blank}[1]{\underline{\hspace*{#1}}}


%%%%%%%%%%
%Custom Theorem Environments
%%%%%%%%%%
\theoremstyle{definition}
\newtheorem{exercise}{Exercise}
\newtheorem{question}[exercise]{Question}
\newtheorem*{defn}{Definition}
\newtheorem*{exa}{Example}
\newtheorem*{disc}{Group Discussion}
\newtheorem*{nb}{Note}
\newtheorem*{recall}{Recall}
\renewcommand{\emph}[1]{{\color{blue}\texttt{#1}}}

\definecolor{Gold}{RGB}{237, 172, 26}
%Statement block
\newenvironment{statementblock}[1]{%
	\setbeamercolor{block body}{bg=Gold!20}
	\setbeamercolor{block title}{bg=Gold}
	\begin{block}{\textbf{#1.}}}{\end{block}}



\begin{document}
	\title{Math 324: Linear Algebra}
	\subtitle{1.1: Introduction to Systems of Linear Equations\\1.2: Gaussian Elimination and Gauss-Jordan Elimination}
	\author{Mckenzie West}
	\date{Last Updated: \today}
\begin{frame}
\maketitle
\end{frame}

\begin{frame}{\insertframenumber}
	\begin{block}{\textbf{Last Time.}}
	\begin{itemize}[label=--]
		\item Linear Equations
		\item Solving Linear Equations and Parametrization
		\item Linear Systems of Equations
		\item Solving Linear Systems of Equations
		\item Consistent vs Inconsistent Systems
	\end{itemize}
	\end{block}
	\begin{block}{\textbf{Today.}}
	\begin{itemize}[label=--]
		\item $m\times n$ matrix
		\item coefficient and augmented matrix
		\item elementary row operations
		\item row-echelon form and leading 1's
		\item reduced row-echelon form
	\end{itemize}
	\end{block}
\end{frame}



\begin{frame}{\insertframenumber}
	Today we will use matrices and elementary row operations to simplify the process of solving a system of linear equations.
\end{frame}

\begin{frame}{\insertframenumber}
	\begin{question}
		Last time you found solutions to the system of equations:
		\[\begin{array}{rcrcrcl}
		3x&-&2y&+&3z&=&4\\
		x&&&+&z&=&5\\
		\end{array}\]
		what were some steps you took to do this?
	\end{question}
\end{frame}

\begin{frame}{\insertframenumber}
	\begin{exa}
	I don't know about you, but I think that writing variables is a waste of time, when all we're looking at is canceling coefficients.  Let's save ourselves some time by turning everything into a nicely arranged table:
	\[\begin{array}{rcrcrcl}
	3x&-&2y&+&3z&=&4\\
	x&&&+&z&=&5\\
	\end{array}\dashrightarrow
	\begin{bmatrix}
	3&-2&3&4\\
	1&0&1&5
	\end{bmatrix}\]
	\end{exa}
\begin{center}
\includegraphics[width=1in]{images/stop}
\end{center}
\end{frame}

\begin{frame}{\insertframenumber}
	\begin{defn}
		The tables from the previous page are called \emph{matrices}.
		
		If $m$ and $n$ are positive integers, an \emph{$m\times n$ matrix} is a rectangular array with $m$ rows and $n$ columns:
			\[\begin{bmatrix}
				a_{11}&a_{12}&a_{13}&\dots&a_{1n}\\
				a_{21}&a_{22}&a_{23}&\dots&a_{2n}\\
				a_{31}&a_{32}&a_{33}&\dots&a_{3n}\\
				\vdots&\vdots&\vdots&&\vdots\\
				a_{m1}&a_{m2}&a_{m3}&\dots&a_{mn}\\
			\end{bmatrix}\]
		where the entries $a_{ij}$ are real numbers.  
		
		We say the \emph{size} of the matrix is $m\times n$.
		
		A \emph{square} matrix is one where $m=n$, in this case we call the entries $a_{11},a_{22},\dots,a_{nn}$ the \emph{main diagonal}.
	\end{defn}
\end{frame}

\begin{frame}{\insertframenumber}
	\begin{exercise}
	What is the size of each of the following matrices:
		\begin{enumerate}[label=(\alph*)]
			\item $\begin{bmatrix}
				1&-1&2\\0&0&1
			\end{bmatrix}$
			\item $\begin{bmatrix}
			4&0\\-54&8
			\end{bmatrix}$
			\item $\begin{bmatrix}
			1\\2\\-1\\1
			\end{bmatrix}$
		\end{enumerate}
	\end{exercise}
\end{frame}

\begin{frame}{\insertframenumber}
	\begin{defn}
		There are two basic matrices that can be constructed using a linear system of equations:
			\begin{itemize}
				\item the \emph{coefficient matrix} which contains only the coefficients of the linear system, not the constant terms;
				\item the \emph{augmented matrix} which is the matrix that contains both the coefficients and the constant terms.
			\end{itemize}
	\end{defn}
	\begin{question}
		If we're solving a system of linear equations which of these two matrices are we going to use?
	\end{question}
\end{frame}


\begin{frame}{\insertframenumber}
	\begin{exercise}
		Write down the coefficient matrix and augmented matrix for the following system of linear equations:
			\[
				\begin{array}{rcrcrccl}
					x&&&-&2z&=&5\\
					2x&+&3y&&&=&-3\\
					-2x&+&y&-&2z&=&3
				\end{array}
			\]
	\end{exercise}
\end{frame}

\begin{frame}{\insertframenumber}
	\begin{defn}[Elementary Row Operations]
		There are three \emph{elementary row operations} that we use on a matrix:
			\begin{enumerate}[label=\textbf{\arabic*.}]
				\item Interchange two rows.
				\item Multiply an equation by a nonzero constant.
				\item Add a multiple of on equation to another equation.
			\end{enumerate}
	\end{defn}
	\begin{nb}
		Each of these operations corresponds to an equivalent action on the linear equations.
	\end{nb}
\end{frame}

\begin{frame}{\insertframenumber}
	\begin{exercise}
		Use elementary row operations on the augmented matrix and back substitution to solve the system of linear equations.
		\[\begin{array}{rcrcrcl}
		 	x&+&2y&-&4z&=&-5\\
		 	&&y&+&z&=&1\\
		 	-2x&-&y&-&z&=&5
		\end{array}\]
%		\[
%		\left[\begin{array}{rrrr}
%			1 & 2 & -4 & -5 \\
%			4 & -3 & -5 & 5 \\
%			-2 & 5 & -1 & 5
%		\end{array}\right]
%		\]
%		\left[\begin{array}{rrrr}
%			1 & 0 & -2 & 0 \\
%			0 & 1 & -1 & 0 \\
%			0 & 0 & 0 & 1
%		\end{array}\right]
	\end{exercise}
\end{frame}

\begin{frame}{\insertframenumber}
	\begin{defn}
	A matrix is in \emph{row-echelon form} if it satisfies the following properties:
		\begin{enumerate}[label=\arabic*.]
			\item Any rows consisting entirely of zeros occur at the bottom of the matrix.
			\item For each row that does not consist entirely of zeros, the first nonzero entry is 1 (called a \emph{leading 1}).
			\item For two successive (nonzero) rows, the leading 1 in the higher row is farther to the left than the leading 1 of the lower row.
		\end{enumerate}
	\end{defn}
	\begin{exa}
		The following matrix is in row echelon form, verify this
		\[\left[\begin{array}{rrrr}
		1 & 4 & 0 & 11 \\
		0 & 1 & 14 & 8 \\
		0 & 0 & 0 & 1
		\end{array}\right]\]
	\end{exa}
\end{frame}
\begin{frame}{\insertframenumber}
	\begin{exercise}
		Which of the following matrices is in row-echelon form?
		\begin{enumerate}[label=(\alph*)]
			\item $\left[\begin{array}{rr}
			1 & 4 \\
			0 & 1 \\
			0 & 0
			\end{array}\right]$
			\item $\left[\begin{array}{rrr}
			1 & 4 & -2 \\
			3 & 0 & -6
			\end{array}\right]$
			\item $\left[\begin{array}{rrrrrrrr}
			1 & 4 & -1 & 5 & -4 & -2 & -4 & 4
			\end{array}\right]$
			\item $\left[\begin{array}{rrrr}
				1 & 0 & 0 & 0 \\
				0 & 1 & 0 & 0 \\
				0 & 0 & 1 & 0 \\
				0 & 0 & 0 & 1 \\
				0 & 0 & 0 & 1
			\end{array}\right]$
		\end{enumerate}
	\end{exercise}
\end{frame}

\begin{frame}{\insertframenumber}
	\begin{question}
		Can we use elementary row operations to reduce a matrix to row echelon form?
	\end{question}
	\begin{question}
		How will row-echelon form help us solve linear systems of equations?
	\end{question}
\end{frame}

\begin{frame}{\insertframenumber}
	\begin{defn}
		A matrix is in \emph{reduced row-echelon form} if it is in row-echelon form and in every column with a leading 1 all other entries are zero.
	\end{defn}
	\begin{exercise}
		Which of the following matrices are in reduced row-echelon form?
		\begin{enumerate}[label=(\alph*)]
			\item $\left[\begin{array}{rr}
			1 & 0 \\
			0 & 0 \\
			0 & 1
			\end{array}\right]$
			\item $\left[\begin{array}{rrr}
			1 & 13 & 0 \\
			0 & 30 & 1
			\end{array}\right]$
			\item$\left[\begin{array}{rrr}
				1 & 3 & 0 \\
				0 & 0 & 1
			\end{array}\right]$
		\end{enumerate}
	\end{exercise}
\end{frame}

\begin{frame}{\insertframenumber}
	\begin{block}{\textbf{Gaussian Elimination with Back-Substitution}}
		\begin{enumerate}[label=\arabic*.]
			\item Write the augmented matrix of the system of linear equations.
			\item Use elementary row operations to rewrite the matrix in row-echelon form.
			\item Write the system of linear equations corresponding to the matrix in row-echelon form, and use back-substitution to find the solution.
		\end{enumerate}
	\end{block}
\end{frame}

\begin{frame}{\insertframenumber}
\begin{exa}
	In a chemical reaction, atoms reorganize in one or more substances.  For instance, when methane gas (CH$_4$) combines with oxygen (O$_2$) and burns, carbon dioxide (CO$_2$) and water (H$_2$O) form.  Chemists represent this process by a chemical equation of the form	
	\[(x_1){CH}_4+(x_2){O}_2\rightarrow(x_3){CO}_2 + (x_4){H}_2{O}.\]
	With hopes of solving, we can create a system of linear equations, one for each element:
	$$\begin{array}{rrcl}
		C:&1x_1+0x_2&=&1x_3+0x_4\\
		H:&4x_1+0x_2&=&0x_3+2x_4\\
		O:&0x_1+2x_2&=&2x_3+1x_4
	\end{array}
	$$
\end{exa}
\end{frame}

\begin{frame}{\fn}
	We then rearrange this system into the form we have been working with:
		$$\begin{array}{rrcl}
	C:&1x_1+0x_2&=&1x_3+0x_4,\\
	H:&4x_1+0x_2&=&0x_3+2x_4,\\
	O:&0x_1+2x_2&=&2x_3+1x_4,
	\end{array}$$
	becomes
	$$\begin{array}{rcrcrcrcl}
	x_1&&&-&x_3&&&=&0,\\
	4x_1&&&&&-&2x_4&=&0,\\
	&&2x_2&-&2x_3&-&x_4&=&0.
	\end{array}
	$$
	As an augmented matrix, we get the following matrix and reduced row echelon form:
	\[\begin{bmatrix}
	1&0&-1&0&0\\4&0&0&-2&0\\0&2&-2&-1&0
	\end{bmatrix}\rightarrow
	\begin{bmatrix}
	1 & 0 & 0 & -\frac{1}{2} & 0 \\
	0 & 1 & 0 & -1 & 0 \\
	0 & 0 & 1 & -\frac{1}{2} & 0
	\end{bmatrix}\]
	\begin{exercise}
		Write the solutions to the system in parametric form.
	\end{exercise}
\end{frame}

\begin{frame}
	\begin{exercise}
		Balance the chemical equation:	
			\[(x_1)Cu+(x_2)HNO_3\rightarrow (x_3)Cu(NO_3)_2+(x_4)NO+(x_5)H_2O\]
		That is, find a parametric representation of the solutions to the linear system as in the previous example.
		
		If you want to end up with 7,200 units of water, $H_2O$, how much of the initial compounds must you start with?
	\end{exercise}
\end{frame}

\begin{frame}[fragile]
	\frametitle{\insertframenumber}
\begin{exercise}
	Discuss a general method you would use to reduce a matrix to row echelon form.
\end{exercise}
\begin{exercise}
	Open \url{http://sagecell.sagemath.org} and use the following code to generate a random matrix:
	
	\begin{verbatim}
	matrix([[randint(-2,2) for i in [1..5]] for j in [1..4]])
	\end{verbatim}

	Do this a couple times until you have one with 3-6 zeros.
	
	Use elementary row operations to reduce this matrix to (reduced) row-echelon form.
\end{exercise}
\end{frame}
\end{document}

