\documentclass{beamer}
%\usepackage[margin=1in]{geometry}
\usepackage{amsthm,amsmath,amsfonts,hyperref,graphicx,color,multicol}
\usepackage{enumitem,tikz}

%%%%%%%%%%
%Beamer Template Customization
%%%%%%%%%%
\setbeamertemplate{navigation symbols}{}
\setbeamertemplate{theorems}[ams style]
\setbeamertemplate{blocks}[rounded]

\definecolor{Blu}{RGB}{43,62,133} % UWEC Blue
\setbeamercolor{structure}{fg=Blu} % Titles

%Unnumbered footnotes:
\newcommand{\blfootnote}[1]{%
	\begingroup
	\renewcommand\thefootnote{}\footnote{#1}%
	\addtocounter{footnote}{-1}%
	\endgroup
}


%%%%%%%%%%
%Custom Commands
%%%%%%%%%%
\newcommand{\R}{\mathbb{R}}
\newcommand{\veca}{\vec{a}}
\newcommand{\vecb}{\vec{b}}
\newcommand{\vece}{\vec{e}}
\newcommand{\vecu}{\vec{u}}
\newcommand{\vecv}{\vec{v}}
\newcommand{\vecw}{\vec{w}}
\newcommand{\vecx}{\vec{x}}
\newcommand{\zerovector}{\vec{0}}

\newcommand{\ds}{\displaystyle}

\newcommand{\fn}{\insertframenumber}

\newcommand{\rank}{\operatorname{rank}}
\newcommand{\adj}{\operatorname{adj}}

\newcommand{\blank}[1]{\underline{\hspace*{#1}}}


%%%%%%%%%%
%Custom Theorem Environments
%%%%%%%%%%
\theoremstyle{definition}
\newtheorem{exercise}{Exercise}
\newtheorem{question}[exercise]{Question}
\newtheorem*{defn}{Definition}
\newtheorem*{exa}{Example}
\newtheorem*{disc}{Group Discussion}
\newtheorem*{nb}{Note}
\newtheorem*{recall}{Recall}
\renewcommand{\emph}[1]{{\color{blue}\texttt{#1}}}

\definecolor{Gold}{RGB}{237, 172, 26}
%Statement block
\newenvironment{statementblock}[1]{%
	\setbeamercolor{block body}{bg=Gold!20}
	\setbeamercolor{block title}{bg=Gold}
	\begin{block}{\textbf{#1.}}}{\end{block}}



\begin{document}
	\title{Math 324: Linear Algebra}
	\subtitle{1.2: Gaussian Elimination and Gauss-Jordan Elimination}
	\author{Mckenzie West}
	\date{Last Updated: \today}
\begin{frame}
\maketitle
\end{frame}

\begin{frame}{\insertframenumber}
	\begin{block}{\textbf{Last Time.}}
	\begin{itemize}[label=--]
		\item $m\times n$ matrix
		\item coefficient and augmented matrix
		\item elementary row operations
		\item row-echelon form and leading 1's
		\item reduced row-echelon form
	\end{itemize}
	\end{block}
\begin{block}{\textbf{Today.}}
\begin{itemize}[label=--]
	\item an algorithm for row reducing
	\item Gauss--Jordan elimination
	\item homogeneous equations
\end{itemize}
\end{block}
\end{frame}

\begin{frame}{\insertframenumber}
\begin{exercise}
	What are all of the possible $2\times 3$ reduced row-echelon matrices? (Use a $*$ to indicate a position that can have any value.) 
	
	For example, 
		\[\begin{bmatrix}1&*&0\\0&0&1\end{bmatrix}.\]
\end{exercise}
\begin{exercise}
	For each of the reduced row-echelon forms determine if the corresponding linear system with that augmented matrix would have: one solution, infinitely many solutions, or no solution.
\end{exercise}
\end{frame}

\begin{frame}{\insertframenumber}
	\begin{statementblock}{A generic row reducing algorithm - Gaussian Elimination}
		Work column by column from left to right and top to bottom.
		\begin{enumerate}[label=\arabic*]
			\item Swap rows and/or scale a row so that there is a $1$ in the top left corner.
			\item Add a multiple of row 1 to each of the rows below to zero-out the rest of the first column.
			\item Repeat these steps, ignoring the first row and column. 
		\end{enumerate}
	\end{statementblock}
\end{frame}
\begin{frame}{\insertframenumber}
	\begin{exa}
		\begin{eqnarray*}
			\left[\begin{array}{rrrr}
			0 & -2 & 2 & 0 \\
			1 & 0 & 0 & 1 \\
			-2 & -1 & 2 & -3
		\end{array}\right]\pause
		& \xrightarrow{R_1\leftrightarrow R_2} & 
		\left[\begin{array}{rrrr}
			1 & 0 & 0 & 1 \\
			0 & -2 & 2 & 0 \\
			-2 & -1 & 2 & -3
		\end{array}\right]\\\pause
		&\xrightarrow{R_3+2R_1\rightarrow R_3}&
		\left[\begin{array}{rrrr}
		1 & 0 & 0 & 1 \\
		0 & -2 & 2 & 0 \\
		0 & -1 & 2 & -1
		\end{array}\right]\\\pause
		&\xrightarrow{-\frac{1}{2}R_2\rightarrow R_2}&
		\left[\begin{array}{rrrr}
		1 & 0 & 0 & 1 \\
		0 & 1 & -1 & 0 \\
		0 & -1 & 2 & -1
		\end{array}\right]\\\pause
		&\xrightarrow{R_3+R_2\rightarrow R_3}&
		\left[\begin{array}{rrrr}
		1 & 0 & 0 & 1 \\
		0 & 1 & -1 & 0 \\
		0 & 0 & 1 & -1
		\end{array}\right]
		\end{eqnarray*}
	\end{exa}
\end{frame}

\begin{frame}{\insertframenumber}
	\begin{exercise}
		Use this algorithm to find the reduced row echelon form of the matrix:
			$$\left[\begin{array}{rrrrr}
			3 & -2 & -1 & 2 & 0 \\
			-3 & 3 & -3 & 3 & 0 \\
			2 & -3 & 2 & -3 & -3 \\
			-1 & 2 & 3 & -2 & 0
			\end{array}\right].$$
%\begin{tabular}{rl|rl}
%	1.& $\left[\begin{array}{rrrrr}
%	-2 & -1 & 1 & 0 & -3 \\
%	-1 & -2 & 2 & -3 & 2 \\
%	-3 & -1 & -1 & 1 & 0 \\
%	3 & 3 & 2 & 3 & -1
%	\end{array}\right]$
%	%	\left(\begin{array}{rrrrr}
%	%		1 & 0 & 0 & -1 & 0 \\
%	%		0 & 1 & 0 & 2 & 0 \\
%	%		0 & 0 & 1 & 0 & 0 \\
%	%		0 & 0 & 0 & 0 & 1
%	%	\end{array}\right)
%	&2.& 
%	$\left[\begin{array}{rrrrr}
%	-3 & 3 & 2 & -2 & -2 \\
%	-2 & 3 & 0 & 2 & -2 \\
%	-1 & 2 & -1 & 0 & 2 \\
%	-3 & 3 & 0 & 3 & -3
%	\end{array}\right]$
%	%$\left[\begin{array}{rrrrr}
%	%	1 & 0 & 0 & 0 & 0 \\
%	%	0 & 1 & 0 & 0 & 0 \\
%	%	0 & 0 & 1 & 0 & -2 \\
%	%	0 & 0 & 0 & 1 & -1
%	%\end{array}\right]$
%	\\&&\\
%	3.& 
%	$\left[\begin{array}{rrrrr}
%	3 & -2 & -1 & 2 & 0 \\
%	-3 & 3 & -3 & 3 & 0 \\
%	2 & -3 & 2 & -3 & -3 \\
%	-1 & 2 & 3 & -2 & 0
%	\end{array}\right]$
%	%$\left[\begin{array}{rrrrr}
%	%	1 & 0 & 0 & 1 & 0 \\
%	%	0 & 1 & 0 & 1 & 0 \\
%	%	0 & 0 & 1 & -1 & 0 \\
%	%	0 & 0 & 0 & 0 & 1
%	%\end{array}\right]$
%	&4.& 
%	$\left[\begin{array}{rrrrr}
%	-2 & 0 & 2 & 2 & -3 \\
%	-2 & 0 & 2 & 2 & -2 \\
%	-1 & 2 & -1 & 1 & -1 \\
%	-2 & 1 & 3 & 2 & -2
%	\end{array}\right]$
%	%$\left[\begin{array}{rrrrr}
%	%	1 & 0 & 0 & -1 & 0 \\
%	%	0 & 1 & 0 & 0 & 0 \\
%	%	0 & 0 & 1 & 0 & 0 \\
%	%	0 & 0 & 0 & 0 & 1
%	%\end{array}\right]$
%	\\&&\\
%	5.&
%	$\left[\begin{array}{rrrrr}
%	-1 & 0 & 1 & 1 & -3 \\
%	-1 & -3 & 1 & 1 & -3 \\
%	0 & -1 & 0 & -2 & -1 \\
%	-3 & 0 & 3 & 3 & 3
%	\end{array}\right]$
%	%$\left[\begin{array}{rrrrr}
%	%	1 & 0 & -1 & 0 & 0 \\
%	%	0 & 1 & 0 & 0 & 0 \\
%	%	0 & 0 & 0 & 1 & 0 \\
%	%	0 & 0 & 0 & 0 & 1
%	%\end{array}\right]$
%	&6.& 
%	$\left[\begin{array}{rrrrr}
%	3 & 3 & 2 & -2 & 1 \\
%	-1 & -1 & 2 & -2 & 2 \\
%	1 & 2 & 2 & 3 & -1 \\
%	2 & 2 & -2 & 2 & 1
%	\end{array}\right]$
%	%$\left[\begin{array}{rrrrr}
%	%1 & 0 & 0 & -5 & 0 \\
%	%0 & 1 & 0 & 5 & 0 \\
%	%0 & 0 & 1 & -1 & 0 \\
%	%0 & 0 & 0 & 0 & 1
%	%\end{array}\right]$
%\end{tabular}
	\end{exercise}
\end{frame}

\begin{frame}{\insertframenumber}
	\begin{question}
		Can a matrix be reduced to two different row-echelon forms?
		
		Can a matrix be reduced to two different reduced row-echelon forms?
	\end{question}
	\begin{question}
		If a matrix is in row-echelon form, how would you go about reducing it to reduced row-echelon form?
	\end{question}
	\begin{block}{Facts}
		\begin{enumerate}[label=(\alph*)]
			\item Yes, every matrix can be reduced to a \textit{unique} reduced row echelon form.
			\item The process of translating a RE matrix to a RRE matrix is called \emph{Gauss-Jordan elimination}.
		\end{enumerate}
	\end{block}
\end{frame}

\begin{frame}{\insertframenumber}
	\begin{defn}
		A \emph{homogeneous system of linear equations} is a linear system of equations in which every constant term is zero.
		
		Every homogeneous system has the \emph{trivial} solution, the one where every variable is zero.
	\end{defn}
\end{frame}

\begin{frame}{\insertframenumber}
	\begin{exercise}
		Solve the homogeneous linear system with the given coefficient matrix:
		\[
		\left[\begin{array}{rrrr}
		0 & 1 & 0 & 0 \\
		1 & 1 & -1 & 1 \\
		0 & 0 & 0 & -1
		\end{array}\right]\]
	\end{exercise}
	\pause
	\begin{question}
		Why is it ok to just use the coefficient matrix to solve a homogeneous system but for other systems we use the augmented matrix?
	\end{question}
\end{frame}

\begin{frame}{\insertframenumber}
	\begin{exercise}
		Assume the given matrix is the augmented matrix of a system of linear equations, and (a) determine the number of equations and the number of variables, and (b) find the values of $k$ such that the system is consistent.  
	
		Then assume the matrix is the coefficient matrix of a homogeneous system of linear equations and repeat (a) and (b).
		
		\[A=\begin{bmatrix} 1&k&2\\-3&4&1\end{bmatrix}\]
	\end{exercise}
\end{frame}

\begin{frame}{\insertframenumber}
	\begin{exercise}
		Find the values of $a$, $b$ and $c$ (if possible) such that the system of linear equations has 	
		\begin{enumerate}[label=(\alph*)]
			\item a unique solution
			\item no solutions
			\item infinitely many solutions
		\end{enumerate}
		\[\begin{matrix}
			x&+&y&&&=&2\\
			&&y&+&z&=&2\\
			x&&&+&z&=&2\\
			ax&+&by&+&cz&=&0
		\end{matrix}\]
	\end{exercise}
\end{frame}

\begin{frame}[fragile]
	\frametitle{\insertframenumber}
	\begin{exercise}
		Get more practice!
		
		Open \url{http://sagecell.sagemath.org} and use the following code to generate a random matrix:
		
		
		\begin{verbatim}
		matrix([[randint(0,3) for i in [1..5]] for j in [1..4]])
		\end{verbatim}
		
		Do this a couple times until you have one with 3-6 zeros.
		
		Use elementary row operations to reduce this matrix to (reduced) row-echelon form.
	\end{exercise}
\end{frame}
\end{document}

