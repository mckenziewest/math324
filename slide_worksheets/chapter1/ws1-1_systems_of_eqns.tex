\documentclass{beamer}
%\usepackage[margin=1in]{geometry}
\usepackage{amsthm,amsmath,amsfonts,hyperref,graphicx,color,multicol}
\usepackage{enumitem,tikz}

%%%%%%%%%%
%Beamer Template Customization
%%%%%%%%%%
\setbeamertemplate{navigation symbols}{}
\setbeamertemplate{theorems}[ams style]
\setbeamertemplate{blocks}[rounded]

\definecolor{Blu}{RGB}{43,62,133} % UWEC Blue
\setbeamercolor{structure}{fg=Blu} % Titles

%Unnumbered footnotes:
\newcommand{\blfootnote}[1]{%
	\begingroup
	\renewcommand\thefootnote{}\footnote{#1}%
	\addtocounter{footnote}{-1}%
	\endgroup
}


%%%%%%%%%%
%Custom Commands
%%%%%%%%%%
\newcommand{\R}{\mathbb{R}}
\newcommand{\veca}{\vec{a}}
\newcommand{\vecb}{\vec{b}}
\newcommand{\vece}{\vec{e}}
\newcommand{\vecu}{\vec{u}}
\newcommand{\vecv}{\vec{v}}
\newcommand{\vecw}{\vec{w}}
\newcommand{\vecx}{\vec{x}}
\newcommand{\zerovector}{\vec{0}}

\newcommand{\ds}{\displaystyle}

\newcommand{\fn}{\insertframenumber}

\newcommand{\rank}{\operatorname{rank}}
\newcommand{\adj}{\operatorname{adj}}

\newcommand{\blank}[1]{\underline{\hspace*{#1}}}


%%%%%%%%%%
%Custom Theorem Environments
%%%%%%%%%%
\theoremstyle{definition}
\newtheorem{exercise}{Exercise}
\newtheorem{question}[exercise]{Question}
\newtheorem*{defn}{Definition}
\newtheorem*{exa}{Example}
\newtheorem*{disc}{Group Discussion}
\newtheorem*{nb}{Note}
\newtheorem*{recall}{Recall}
\renewcommand{\emph}[1]{{\color{blue}\texttt{#1}}}

\definecolor{Gold}{RGB}{237, 172, 26}
%Statement block
\newenvironment{statementblock}[1]{%
	\setbeamercolor{block body}{bg=Gold!20}
	\setbeamercolor{block title}{bg=Gold}
	\begin{block}{\textbf{#1.}}}{\end{block}}




\begin{document}
	\title{Math 324: Linear Algebra}
	\subtitle{1.1: Introduction to Systems of Linear Equations}
	\author{Mckenzie West}
	\date{Last Updated: \today}
\begin{frame}
\maketitle
\end{frame}
\begin{frame}{\insertframenumber}
	\begin{block}{\textbf{Today.}}
		\begin{itemize}[label=--]
			\item Linear Equations
			\item Solving Linear Equations and Parametrization
			\item Linear Systems of Equations
			\item Solving Linear Systems of Equations
			\item Consistent vs Inconsistent Systems
		\end{itemize}
	\end{block}
\end{frame}

\begin{frame}{\insertframenumber}
\begin{defn}
	A \emph{linear equation in $n$ variables} $x_1,x_2,\dots,x_n$ is an equation of the form
	\[a_1x_1+a_2x_2+\cdots+a_nx_n=b\]
where $a_1,a_2,\dots,a_n$ are real numbers called \emph{coefficients} and $b$ is a real number called the \emph{constant term}.\par
We call $a_1$ the \emph{leading coefficient} and $x_1$ the \emph{leading variable}.
\end{defn}
\end{frame}

\begin{frame}{\insertframenumber}
\begin{exercise}
	Which of the following equations are linear equations?  For the linear equations, identify the constant term, leading coefficient, and leading variable.
	\begin{multicols}{2}
	\begin{enumerate}[label=(\alph*)]
		\item $x-2y=1$
		\item $\sin x+\cos y=1$
		\item $x-xy+y=3$
		\item $5x_1=6+3x_2$
		\item $3(x_1+5)=2(-6-x_2)-2x_3$
		\item $\displaystyle\frac{2z}{z+3}=\frac{3}{z-10}+2$
	\end{enumerate}
	\end{multicols}
\end{exercise}
\end{frame}

\begin{frame}{\insertframenumber}
\begin{defn}
	A \emph{solution} to a linear equation in $n$ variables is a sequence of $n$ real numbers $s_1,s_2,\dots,s_n$ arranged so that when you substitute the values $x_1=s_1,x_2=s_2,\dots,x_n=s_n$ the equation is satisfied.
\end{defn}
\pause
\begin{exercise}
	Find a solution to $x-2y=1$.  (Be creative here.)
	
	How many solutions does this equation have?  Can you describe them all?
\end{exercise}
\end{frame}

\begin{frame}{\insertframenumber}
\begin{defn}
	The collection of all solutions of a linear equation is called a \emph{solution set}.  If you're asked to \emph{solve} a linear equation, you should find all of the solutions and describe them.
\end{defn}
\begin{exa}
	The solutions to the linear equation $x+y=4$ are all of the form $x=4-y$.
	We write these \emph{parametrically} as $y=t$, $x=4-t$. We call $y$ a \emph{free} variable because it is free to be whatever it wants to be.
	
	We call $t$ a \emph{parameter} and specific solutions can be found by assigning values to this parameter.
\end{exa}
\begin{block}{Note}
	Typically, leading variables are not used as free variables.
\end{block}
\end{frame}

\begin{frame}{\insertframenumber}
\begin{exercise}
	Write all solutions to $5x=6+3y$ in parametric form.
\end{exercise}
\begin{exercise}
	Write all solutions to $6x-2y+3z=1$ in parametric form.  (Note: you may need more than one parameter.)
\end{exercise}
\end{frame}

\begin{frame}{\insertframenumber}
\begin{defn}
	A \emph{system of $m$ linear equations in $n$ variables} is a set of $m$ equations each of which is linear in up to $n$ variables $x_1,\dots,x_n$.
\end{defn}
\begin{exa}
	A system of 2 equations in 3 variables:	
		\[\begin{array}{rcrcrcl}
			3x&-&2y&+&3z&=&4\\
			x&&&+&z&=&5\\
		\end{array}\]
\end{exa}
\end{frame}

\begin{frame}{\insertframenumber}
\begin{defn}
A \emph{solution} to a system of $m$ linear equations in $n$ variables is any sequence of $n$ real numbers that is a solution to every linear equation in the system. To \emph{solve} a linear system is to find all of the solutions to the system.\par
	A system is called \emph{consistent} if it has at least one solution and \emph{inconsistent} if it has none.
\end{defn}
\end{frame}

\begin{frame}{Slide \insertframenumber}
\begin{exercise}
	Find a solution to  the system
		\[\begin{array}{rcrcrcl}
		3x&-&2y&+&3z&=&4\\
		x&&&+&z&=&5\\
		\end{array}\]
\end{exercise}
\begin{exercise}
	Come up with a linear system of 2 equations and 2 variables that is inconsistent.
\end{exercise}
\end{frame}

\begin{frame}{\insertframenumber}
\begin{exercise}
For each of the following systems, graph the two lines in the $xy$-plane.  Where do they intersect? How many solutions does this system of linear equations have?
		\begin{enumerate}[label=(\alph*)]
			\item $\begin{array}{rcrcl}
			3x&-&y&=&1\\
			2x&-&y&=&0\\
			\end{array}$
			\vskip .3in
			\item $\begin{array}{rcrcl}
			3x&-&y&=&1\\
			3x&-&y&=&0\\
			\end{array}$
			\vskip .3in
			\item $\begin{array}{rcrcl}
			3x&-&y&=&1\\
			6x&-&2y&=&2\\
			\end{array}$
		\end{enumerate}
\end{exercise}
\end{frame}

\begin{frame}{\insertframenumber}
\begin{question}
	Is it possible for a system of linear equations in two variables to have two solutions?
\end{question}
\end{frame}

\begin{frame}{\insertframenumber}
\begin{exercise}
	Consider the system of equations:
	$$\begin{array}{rcrcrcrcr}
		-2x&-&2y&+&2z&-&w &=& 2\\
		&&2y&-&2z&+&w&=&0\\
		2x&&&&&&&=&2
		\end{array}$$
	Find all solutions to the system.
\end{exercise}
\end{frame}

\begin{frame}{\insertframenumber}
\begin{exercise}
	In a chemical reaction, atoms reorganize in one or more substances.  For instance, when methane gas (CH$_4$) combines with oxygen (O$_2$) and burns, carbon dioxide (CO$_2$) and water (H$_2$O) form.  Chemists represent this process by a chemical equation of the form	
		\[(x_1)\textup{CH}_4+(x_2)\textup{O}_2\rightarrow(x_3)\textup{CO}_2 + (x_4)\textup{H}_2\textup{O}.\]
	Write a system of linear equations in the four variables $x_1,x_2,x_3,x_4$ corresponding to the fact that the amount of each element--carbon, hydrogen, and oxygen--must be equal before and after the reaction.
\end{exercise}
\end{frame}
\end{document}