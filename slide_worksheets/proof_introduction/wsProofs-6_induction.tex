\documentclass{beamer}
%\usepackage[margin=1in]{geometry}
\usepackage{amsthm,amsmath,amsfonts,hyperref,graphicx,color,multicol}
\usepackage{enumitem,tikz}

%%%%%%%%%%
%Beamer Template Customization
%%%%%%%%%%
\setbeamertemplate{navigation symbols}{}
\setbeamertemplate{theorems}[ams style]
\setbeamertemplate{blocks}[rounded]

\definecolor{Blu}{RGB}{43,62,133} % UWEC Blue
\setbeamercolor{structure}{fg=Blu} % Titles

%Unnumbered footnotes:
\newcommand{\blfootnote}[1]{%
	\begingroup
	\renewcommand\thefootnote{}\footnote{#1}%
	\addtocounter{footnote}{-1}%
	\endgroup
}


%%%%%%%%%%
%Custom Commands
%%%%%%%%%%
\newcommand{\R}{\mathbb{R}}
\newcommand{\veca}{\mathbf{a}}
\newcommand{\vecb}{\mathbf{b}}
\newcommand{\vecu}{\mathbf{u}}
\newcommand{\vecv}{\mathbf{v}}
\newcommand{\vecw}{\mathbf{w}}
\newcommand{\vecx}{\mathbf{x}}
\newcommand{\zerovector}{\mathbf{0}}

\newcommand{\ds}{\displaystyle}

\newcommand{\fn}{\insertframenumber}

\newcommand{\rank}{\operatorname{rank}}

\newcommand{\blank}[1]{\underline{\hspace*{#1}}}


%%%%%%%%%%
%Custom Theorem Environments
%%%%%%%%%%
\theoremstyle{definition}
\newtheorem{exercise}{Exercise}
\newtheorem{question}[exercise]{Question}
\newtheorem*{defn}{Definition}
\newtheorem*{exa}{Example}
\newtheorem*{disc}{Group Discussion}
\newtheorem*{nb}{Note}
\newtheorem*{recall}{Recall}
\renewcommand{\emph}[1]{{\color{blue}\texttt{#1}}}

\definecolor{Gold}{RGB}{237, 172, 26}
%Statement block
\newenvironment{statementblock}[1]{%
	\setbeamercolor{block body}{bg=Gold!20}
	\setbeamercolor{block title}{bg=Gold}
	\begin{block}{\textbf{#1.}}}{\end{block}}





\begin{document}
	\title{Math 324: Linear Algebra}
	\subtitle{Induction}
	\author{Mckenzie West}
	\date{Last Updated: \today}
\begin{frame}
\maketitle
\end{frame}

\begin{frame}{\insertframenumber}
	\begin{block}{\textbf{Last Time.}}
	\begin{itemize}[label=--]
		\item Proof by contradiction
	\end{itemize}
	\end{block}
	\begin{block}{\textbf{Today.}}
		\begin{itemize}[label=--]
			\item Proof by Induction
		\end{itemize}
	\end{block}
\end{frame}

\begin{frame}{\fn}
	\begin{exercise}
		In the video, the speaker talked about what induction is and gave some examples.  Here is one.
			\begin{statementblock}{Proposition}
				For any positive integer $n$, $1+2+\cdots+n=\frac{n(n+1)}{2}$.
			\end{statementblock}
		Re-read the proof and discuss:
		\begin{enumerate}[label=(\alph*)]
			\item Why does this prove that $1+2+3=\frac{3(4)}{2}=6$?
			\item Why does this prove that $1+2+\cdots+10=\frac{10(11)}{2}=55$?
			\item Why does this prove the statement for $n=10000$?
			\item What pieces of the proof are essential for induction proofs?
			\item Share your proof from your pre-class exercise.
		\end{enumerate}
	\end{exercise}
\end{frame}

\begin{frame}{\fn}
	\begin{exercise}
		Use proof by induction to prove the statement that follows.
		\begin{statementblock}{Proposition}
			Let $n\geq 1$ be an integer.  If $A$ is an $m\times m$ matrix and $P$ is a $m\times m$ invertible matrix, then $(P^{-1}AP)^n=P^{-1}A^nP$.
		\end{statementblock}
	\end{exercise}
	\begin{exercise}
		Use proof by induction to prove the statement that follows.
		\begin{statementblock}{Proposition}
			For all integers $n\geq 0$, if $B=\begin{bmatrix}
			2&0\\-1&1
			\end{bmatrix}$, then $B^n=\begin{bmatrix}2^n&0\\1-2^n&1\end{bmatrix}$.
		\end{statementblock}
	\end{exercise}
\end{frame}

\begin{frame}{\fn}
	\begin{exercise}
		Use induction and Theorem 2.9 to prove the following.
	\begin{statementblock}{Proposition}
			Let $n\geq 2$ be a positive integer.
		If $A_1,A_2,\dots,A_n$ are invertible matrices of the same size, then
			\[(A_1A_2\cdots A_n)^{-1}=A_n^{-1}\cdots A_2^{-1}A_1^{-1}.\]
	\end{statementblock}
	\end{exercise}
\end{frame}

\begin{frame}{\fn}
\begin{exercise}
	Use induction and Theorem 2.3 Property 2 to prove the following.
	\begin{statementblock}{Proposition}
		Let $n\geq 2$ be a positive integer.
		If $A$ is an $m\times r$ matrix and $B_1,B_2,\dots,B_n$ are $r\times s$ matrices, then
		\[A(B_1+B_2+\cdots+B_n)=AB_1+AB_2+\cdots+AB_n.\]
	\end{statementblock}
\end{exercise}
\end{frame}

\begin{frame}{\fn}
	\begin{exercise}
		Use induction to prove the following.
	\begin{statementblock}{Proposition}
		If $n$ is a positive integer, then
			\[1(2)+2(3)+3(4)+\cdots+n(n+1)=\frac{n(n+1)(n+2)}{3}.\]
	\end{statementblock}
	\end{exercise}
	\begin{exercise}
		Use induction to prove the following.
		\begin{statementblock}{Proposition}
			If $n\geq 4$, then $n!>2^n$.
		\end{statementblock}
	\end{exercise}
\end{frame}


\end{document}

