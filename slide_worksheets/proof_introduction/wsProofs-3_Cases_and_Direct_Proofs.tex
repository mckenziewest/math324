\documentclass[handout]{beamer}
%\usepackage[margin=1in]{geometry}
\usepackage{amsthm,amsmath,amsfonts,hyperref,graphicx,color,multicol}
\usepackage{enumitem,tikz}

%%%%%%%%%%
%Beamer Template Customization
%%%%%%%%%%
\setbeamertemplate{navigation symbols}{}
\setbeamertemplate{theorems}[ams style]
\setbeamertemplate{blocks}[rounded]

\definecolor{Blu}{RGB}{43,62,133} % UWEC Blue
\setbeamercolor{structure}{fg=Blu} % Titles

%Unnumbered footnotes:
\newcommand{\blfootnote}[1]{%
	\begingroup
	\renewcommand\thefootnote{}\footnote{#1}%
	\addtocounter{footnote}{-1}%
	\endgroup
}


%%%%%%%%%%
%Custom Commands
%%%%%%%%%%
\newcommand{\R}{\mathbb{R}}
\newcommand{\veca}{\vec{a}}
\newcommand{\vecb}{\vec{b}}
\newcommand{\vece}{\vec{e}}
\newcommand{\vecu}{\vec{u}}
\newcommand{\vecv}{\vec{v}}
\newcommand{\vecw}{\vec{w}}
\newcommand{\vecx}{\vec{x}}
\newcommand{\zerovector}{\vec{0}}

\newcommand{\ds}{\displaystyle}

\newcommand{\fn}{\insertframenumber}

\newcommand{\rank}{\operatorname{rank}}
\newcommand{\adj}{\operatorname{adj}}

\newcommand{\blank}[1]{\underline{\hspace*{#1}}}


%%%%%%%%%%
%Custom Theorem Environments
%%%%%%%%%%
\theoremstyle{definition}
\newtheorem{exercise}{Exercise}
\newtheorem{question}[exercise]{Question}
\newtheorem*{defn}{Definition}
\newtheorem*{exa}{Example}
\newtheorem*{disc}{Group Discussion}
\newtheorem*{nb}{Note}
\newtheorem*{recall}{Recall}
\renewcommand{\emph}[1]{{\color{blue}\texttt{#1}}}

\definecolor{Gold}{RGB}{237, 172, 26}
%Statement block
\newenvironment{statementblock}[1]{%
	\setbeamercolor{block body}{bg=Gold!20}
	\setbeamercolor{block title}{bg=Gold}
	\begin{block}{\textbf{#1.}}}{\end{block}}




\begin{document}
	\title{Math 324: Linear Algebra}
	\subtitle{Direct Proofs and Cases}
	\author{Mckenzie West}
	\date{Last Updated: \today}
\begin{frame}
\maketitle
\end{frame}

\begin{frame}{\insertframenumber}
	\begin{block}{\textbf{Last Time.}}
	\begin{itemize}[label=--]
		\item Proper Proof Techniques
		\item Direct Proofs
	\end{itemize}
	\end{block}
\begin{block}{\textbf{Today.}}
	\begin{itemize}[label=--]
		\item Proofs using Cases
	\end{itemize}
\end{block}
\end{frame}
\begin{frame}{\fn}
	\begin{exercise}
		Take time to read the proof from your classmates. 
		
		Indicate (a) 2 things their proof does well and (b) 1-2 things that could be improved.  Then return the commented version to Dr.~West.
	\end{exercise}
\end{frame}
\begin{frame}{\fn}
	\begin{exa}
		Here is one possible proof the the statement from last time.
	\begin{block}{{Claim.}}
		If the product $AB$ is a square matrix, then the product $BA$ is defined.
	\end{block}
	\begin{proof} 
		Let $A$ and $B$ be matrices such that $AB$ is defined and square of size $n\times n$.  By definition of matrix multiplication, this means that $A$ has $n$ rows and $B$ has $n$ columns.  Then the middle dimensions for the product $BA$ match and so the product is defined.
	\end{proof}
		Do you think this is sufficient or would you add more?
	\end{exa}
\end{frame}
\begin{frame}{\fn}
	\begin{defn}
		You may have seen previously that if $A$ is a square matrix satisfying $A^2=A$, then we call $A$ \emph{idempotent}.
	\end{defn}
	\begin{exercise}[Exercise 5 from last time]
		Follow the steps to prove that if $A$ and $B$ are idempotent matrices satisfying $AB=BA$, then $AB$ is also idempotent.
		\begin{enumerate}[label=(\alph*)]
			\item Start with some scratch work, you want to show $(AB)^2=AB$.  
			What is $(AB)^2$? Where can you use the fact that $AB=BA$?
			\item Write down your assumptions.
			\item State what it means for $A$ and $B$ to be idempotent.
			\item Write down a sentence stating what you want to show.
			\item Go through the algebraic steps.  Note: You may want to display them as in guideline (9).
			\item Write a concluding sentence as suggested in guideline (13).
		\end{enumerate}
	\end{exercise}
\end{frame}

\begin{frame}{\fn}
	\begin{block}{\textbf{Brain Break.}}
		When you wake up, what is the very first thing you do?
		\begin{center}
			\includegraphics[width=1.5in]{images/morning}
		\end{center}
	\end{block}
\end{frame}

\begin{frame}{\fn}
\begin{statementblock}{Theorem 2.2 (3)}
	Let $A$ be an $m\times n$ matrix and $c$ a scalar.
	If $cA=O_{mn}$, then $c=0$ or $A=O_{m,n}$.
\end{statementblock}
\begin{exercise}
	Before getting to the proof of this theorem, let's think about how it's going to go.
	
	Because we see the word `or', we're going to want to use cases.  What might be some reasonable cases to consider? Note: There is more than one option.
\end{exercise}
\end{frame}

\begin{frame}{\fn}
\begin{statementblock}{Theorem 2.2 (3)}
	Let $A$ be an $m\times n$ matrix and $c$ a scalar.
	If $cA=O_{mn}$, then $c=0$ or $A=O_{m,n}$.
\end{statementblock}
\begin{exercise}Do some scratch work for each collection of cases:
			\begin{enumerate}[label=(\alph*)]
				\item $c=0$ or $c\neq 0$
				\item $A=O_{m,n}$ or $A\neq O_{m,n}$
			\end{enumerate}			
\end{exercise}
	\begin{nb}
		Keep in mind the fact that $A=O_{m,n}$ if every entry of $A$ is $0$.  On the other hand $A\neq O_{m,n}$ if there exists at least one entry of $A$ that is not $0$.
	\end{nb}
\end{frame}
\begin{frame}{\fn}
\begin{statementblock}{Theorem 2.2 (3)}
	Let $A$ be an $m\times n$ matrix and $c$ a scalar.
	If $cA=O_{m,n}$, then $c=0$ or $A=O_{m,n}$.
\end{statementblock}
\begin{exercise}
	Add some explanation to the proof using the given cases.
	\begin{quote}
		Let $A$ be an $m\times n$ matrix and $c$ a scalar such that $cA=O_{m,n}$.  We consider two cases, \fbox{$c=0$ or $c\neq 0$}.
		
		\underline{Case 1:} Assume $c=0$.  Then the statement is true. (\textit{why})
		
		\underline{Case 2:} Assume $c\neq 0$. We want to show that $A=O_{m,n}$. (\textit{explain why every entry of $A$ must be 0 -- this may include saying something like $ca_{ij}=0$, so $a_{ij}=0$.})
	\end{quote}
	
\end{exercise}
\end{frame}
\begin{frame}{\fn}
	\begin{statementblock}{Theorem 2.2 (3)}
		Let $A$ be an $m\times n$ matrix and $c$ a scalar.
		If $cA=O_{m,n}$, then $c=0$ or $A=O_{m,n}$.
	\end{statementblock}
	\begin{exercise}
		Let's start over and try the other set of cases.
		\begin{quote}
			Let $A$ be an $m\times n$ matrix and $c$ a scalar such that $cA=O_{m,n}$.  We consider two cases, \fbox{$A=O_{m,n}$ or $A\neq O_{m,n}$}.
			
			\underline{Case 1:} Assume $A=O_{m,n}$.  Then the statement is true. (\textit{why})
			
			\underline{Case 2:} Assume $A\neq O_{m,n}$. We want to show that $c=0$. (\textit{explain why $c$ must be 0 -- this may include saying something like there is an entry $a_{ij}\neq 0$ of $A$.....$ca_{ij}=0$, so $c=0$.})
		\end{quote}
		
	\end{exercise}
\end{frame}
\begin{frame}{\fn}
	\begin{exercise}
		Which of these two cases do you prefer?  Keep in mind there is no right answer but it will be useful to discuss this with your group.
	\end{exercise}
\end{frame}
\begin{frame}{\fn}
	\begin{exercise}
		Consider the following statement --- which we may have seen a few times previously.
		\begin{statementblock}{Claim}
			If $A^2=A$, then $A$ is singular or $A=I$.
		\end{statementblock}
		\begin{enumerate}[label=(\alph*)]
			\item To prove this we use cases.  What might you consider as the cases?
			\item Using the cases: $A$ is singular or $A$ is invertible, work out some scratch work.  Follow the foundation of the proofs in exercises 5 and 6 to prove the claim.
		\end{enumerate}
\end{exercise}
\end{frame}
\end{document}

