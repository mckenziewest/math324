\documentclass{beamer}
%\usepackage[margin=1in]{geometry}
\usepackage{amsthm,amsmath,amsfonts,hyperref,graphicx,color,multicol}
\usepackage{enumitem,tikz}

%%%%%%%%%%
%Beamer Template Customization
%%%%%%%%%%
\setbeamertemplate{navigation symbols}{}
\setbeamertemplate{theorems}[ams style]
\setbeamertemplate{blocks}[rounded]

\definecolor{Blu}{RGB}{43,62,133} % UWEC Blue
\setbeamercolor{structure}{fg=Blu} % Titles

%Unnumbered footnotes:
\newcommand{\blfootnote}[1]{%
	\begingroup
	\renewcommand\thefootnote{}\footnote{#1}%
	\addtocounter{footnote}{-1}%
	\endgroup
}


%%%%%%%%%%
%Custom Commands
%%%%%%%%%%
\newcommand{\R}{\mathbb{R}}
\newcommand{\veca}{\mathbf{a}}
\newcommand{\vecb}{\mathbf{b}}
\newcommand{\vecu}{\mathbf{u}}
\newcommand{\vecv}{\mathbf{v}}
\newcommand{\vecw}{\mathbf{w}}
\newcommand{\vecx}{\mathbf{x}}
\newcommand{\zerovector}{\mathbf{0}}

\newcommand{\ds}{\displaystyle}

\newcommand{\fn}{\insertframenumber}

\newcommand{\rank}{\operatorname{rank}}

\newcommand{\blank}[1]{\underline{\hspace*{#1}}}


%%%%%%%%%%
%Custom Theorem Environments
%%%%%%%%%%
\theoremstyle{definition}
\newtheorem{exercise}{Exercise}
\newtheorem{question}[exercise]{Question}
\newtheorem*{defn}{Definition}
\newtheorem*{exa}{Example}
\newtheorem*{disc}{Group Discussion}
\newtheorem*{nb}{Note}
\newtheorem*{recall}{Recall}
\renewcommand{\emph}[1]{{\color{blue}\texttt{#1}}}

\definecolor{Gold}{RGB}{237, 172, 26}
%Statement block
\newenvironment{statementblock}[1]{%
	\setbeamercolor{block body}{bg=Gold!20}
	\setbeamercolor{block title}{bg=Gold}
	\begin{block}{\textbf{#1.}}}{\end{block}}





\begin{document}
	\title{Math 324: Linear Algebra}
	\subtitle{Contradiction}
	\author{Mckenzie West}
	\date{Last Updated: \today}
\begin{frame}
\maketitle
\end{frame}
%\begin{frame}{\fn}
%\begin{statementblock}{Claim}
%	If $x$ and $y$ are integers then $x^2\neq 4y+3$.
%\end{statementblock}
%\end{frame}

\begin{frame}{\insertframenumber}
	\begin{block}{\textbf{Last Time.}}
	\begin{itemize}[label=--]
		\item If and only If Statements
		\item Contrapositive, Converse, and Inverse
	\end{itemize}
	\end{block}
	\begin{block}{\textbf{Today.}}
		\begin{itemize}[label=--]
		\item Proof by contradiction
		\end{itemize}
	\end{block}
\end{frame}
\begin{frame}{\fn}
	\begin{statementblock}{Proposition}
		If $x$ is a real number in $[0,\pi/2]$ then $\sin x+\cos x\geq 1$.
	\end{statementblock}
	
	\begin{exercise}
		Complete the proof in the handout.
	\end{exercise}
	
	The general format of proof by contradiction for a statement of the form $P\Rightarrow Q$
	is as follows:
	\begin{enumerate}[label=\arabic*.]
		\item Assume $P$ is true.
		\item Suppose \textbf{toward contradiction}, that $Q$ is false.  
		
		Yes, including ``toward contradiction'' is important, don't be lazy.
		\item Derive any contradiction.
		\item Conclusion: Therefore is impossible for both $P$ to be true and $Q$ to be false, so the claim is true. 
		
		Yes, write this line too, don't be lazy.
	\end{enumerate}
\end{frame}

\begin{frame}{\fn}
	\begin{exercise}
		Use proof by contradiction to prove:
		\begin{center}\begin{minipage}{.9\textwidth}
				If $a$ and $b$ are integers then $a^2-4b\neq 2$.
		\end{minipage}\end{center}
		You may use the fact that if $a^2$ is even then $a$ is even.
	\end{exercise}
	\begin{exercise}
	Use proof by contradiction to prove:
	\begin{center}\begin{minipage}{.9\textwidth}If $a$, $b$ and $c$ are integers satisfying $a^2+b^2=c^2$, then $a$ or $b$ is even.
	\end{minipage}\end{center}
	\end{exercise}
\end{frame}

\begin{frame}{\fn}
	\begin{exercise}
		Often for contradiction proofs, we may be negating statements of the form \emph{$\neg(A$ or $B)$}, which is logically equivalent to \emph{$\neg A$ and $\neg B$}.
		
		Consider the statements $A=$\texttt{my cat is not orange} and $B=$\texttt{my dog has whiskers}.
		\begin{enumerate}[label=(\alph*)] 
			\item In an English sentence, how would you write \texttt{$\neg(A$ or $B)$}?
			\item How would you write \texttt{$\neg A$ and $\neg B$}?
			\item Why are these logically equivalent.
		\end{enumerate}
	\end{exercise}
\end{frame}

\begin{frame}{\fn}
\begin{exercise}
	\begin{statementblock}{Claim}
	If $AB$ is singular then $A$ is singular or $B$ is singular.
	\end{statementblock}
	\begin{enumerate}[label=(\alph*)]
		\item Represent this statement as $P\Rightarrow Q$.  What is $P$? $Q$? $\neg Q$?
		\item What are the first two sentences of your contradiction proof of this claim?
		\item Find a contradiction. You may want to consider citing Theorem 2.9.
		\item Conclude the proof.
	\end{enumerate}
\end{exercise}
\end{frame}

\begin{frame}{\fn}
\begin{exercise}
	\begin{statementblock}{Claim}
		There are no positive integers $a$ and $b$ such that $18a+6b=1$.
	\end{statementblock}
	\begin{enumerate}[label=(\alph*)]
		\item Write this statement as: if $P$ then $18a+6b\neq 1$.
		\item What are the first two sentences of your contradiction proof of this claim?
		\item Find a contradiction.
		\item Conclude the proof.
	\end{enumerate}
\end{exercise}
\end{frame}

\begin{frame}{\fn}
\begin{exercise}
	\begin{statementblock}{Claim}
		The sum of a rational number and an irrational number is irrational.
	\end{statementblock}
	\begin{enumerate}[label=(\alph*)]
		\item Write this statement as: if $P$ then $Q$.  
		\item What is $P$? $Q$? $\neg Q$?
		\item What are the first two sentences of your contradiction proof of this claim?
		\item Find a contradiction.
		\item Conclude the proof.
	\end{enumerate}
\end{exercise}
\end{frame}
\end{document}

