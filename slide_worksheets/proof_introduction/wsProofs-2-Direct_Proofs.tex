\documentclass{beamer}
%\usepackage[margin=1in]{geometry}
\usepackage{amsthm,amsmath,amsfonts,hyperref,graphicx,color,multicol}
\usepackage{enumitem}

%\usepackage{setspace}
%\doublespacing
% or:
%\onehalfspacing

\newcommand{\R}{\mathbb{R}}
\newcommand{\veca}{\mathbf{a}}
\newcommand{\vecb}{\mathbf{b}}
\newcommand{\vecu}{\mathbf{u}}
\newcommand{\vecv}{\mathbf{v}}
\newcommand{\vecw}{\mathbf{w}}
\newcommand{\vecx}{\mathbf{x}}
\newcommand{\zerovector}{\mathbf{0}}

\newcommand{\fn}{\insertframenumber}

\newcommand{\rank}{\operatorname{rank}}

\newcommand{\blank}[1]{\underline{\hspace*{#1}}}

\theoremstyle{definition}
\newtheorem{exercise}{Exercise}
\newtheorem{question}[exercise]{Question}
\newtheorem*{defn}{Definition}
\newtheorem*{exa}{Example}
\newtheorem*{disc}{Group Discussion}
\newtheorem*{nb}{Note}
\newtheorem*{recall}{Recall}
\renewcommand{\emph}[1]{{\color{blue}\texttt{#1}}}

\setbeamertemplate{navigation symbols}{}
\setbeamertemplate{theorems}[ams style]


%Unnumbered footnotes:
\newcommand{\blfootnote}[1]{%
	\begingroup
	\renewcommand\thefootnote{}\footnote{#1}%
	\addtocounter{footnote}{-1}%
	\endgroup
}

\begin{document}
	\title{Math 324: Linear Algebra}
	\subtitle{Direct Proofs}
	\author{Mckenzie West}
	\date{Last Updated: \today}
\begin{frame}
\maketitle
\end{frame}

\begin{frame}{\insertframenumber}
	\begin{block}{\textbf{Last Time.}}
	\begin{itemize}[label=--]
		\item Statements
		\item Exploration and Counterexamples
		\item Quantifiers
		\item Conditional Statements
		\item Compound Statements
		\item Negation
	\end{itemize}
	\end{block}
\begin{block}{\textbf{Today.}}
	\begin{itemize}[label=--]
		\item Proper Proof Techniques
		\item Direct Proofs
	\end{itemize}
\end{block}
\end{frame}

\begin{frame}{\fn}
	\begin{exercise}
		\begin{itemize}[label=--]
			\item Read the bolded summaries on the handout, focus on those highlighted here.
			\item (1) For this class, who is your audience?
			\item (3) What is the purpose of stating assumptions?
			\item (4) Why are proofs written using ``we'' instead of $I$?
			\item (7) What's the difference between writing a proof and computing an answer?
			\item (9) Notice that even displayed equations end in a period because they should be considered a sentence.
			\item (11) No sentence can begin with a mathematical symbol.  Why?
			\item (15) Can we just write something and walk away?
		\end{itemize}
	\end{exercise}
\end{frame}

\begin{frame}{\fn}
	\begin{exercise}
		Examine your pre-class proof in the context of the handout.  What guidelines does it follow? Does it break any of the guidelines? 
		
		Share your thoughts with your table-mates.
	\end{exercise}
\end{frame}

\begin{frame}{\fn}
	\begin{exercise}
		Re-write the following ``proof'' of the claim that follows so that the proof is correct and matches the guidelines from the handout.
		\begin{block}{\textbf{Claim.}}
			If the product $AB$ is a square matrix, then the product $BA$ is defined.
		\end{block}
		\begin{proof}
			$AB$ is square so it is size $n\times n$. This means $A$ has $n$ columns and $B$ has $n$ columns. I can now multiply $BA$.
		\end{proof}
	\end{exercise}
\end{frame}
\begin{frame}{\fn}
	\begin{block}{\textbf{Brain Break.}}
		What food are you always running out of at your house or dorm?
		\begin{center}
			\includegraphics[width=2in]{../images/pantry}
		\end{center}
	\end{block}
\end{frame}
\begin{frame}{\fn}
	\vskip-.25in
	\begin{exercise}
		The proof of the following claim is long, wordy, and hard to understand.  Re-write a better proof.
		\begin{block}{\textbf{Claim.}}
			If the products $AB$ and $BA$ are both defined then $AB$ is a square matrix.
		\end{block}
		\begin{proof}
			Let $A$ and $B$ be matrices of size $m\times n$ and $r\times s$. Assume that both of the products $AB$ and $BA$ are defined.  Since $AB$ is defined the number of columns of $A$ must equal the number of rows of $B$.  Since $BA$ is defined the number of columns of $B$ must equal the number of rows of $A$.  This means that $n$ equals $r$ and $m$ equals $s$.  Since $A$ is size $m\times n$ and $B$ is size $r\times s$, $AB$ is size $m\times s$.  But $m$ equals $s$ so $AB$ is square because the number of rows of $AB$ is equal to the number of rows of $A$ which is $m$ and the number of columns of $AB$ is equal to the number of columns of $B$ which is $s$ which is equal to $m$, the number of rows of $AB$.
		\end{proof}
	\end{exercise}
\end{frame}
\begin{frame}{\fn}
	\begin{defn}
		You may have seen previously that if $A$ is a square matrix satisfying $A^2=A$, then we call $A$ \emph{idempotent}.
	\end{defn}
	\begin{exercise}
		Follow the steps to prove that if $A$ and $B$ are idempotent matrices satisfying $AB=BA$, then $AB$ is also idempotent.
		\begin{enumerate}[label=(\alph*)]
			\item Start with some scratch work, you want to show $(AB)^2=AB$.  
			Try multiplying, where do you use the fact that $AB=BA$?
			\item Write down your assumptions.
			\item State what it means for $A$ and $B$ to be idempotent.
			\item Write down a sentence stating what you want to show.
			\item Go through the algebraic steps.  Note: You may want to display them as in guideline (9).
			\item Write a concluding sentence as suggested in guideline (13).
		\end{enumerate}
	\end{exercise}
\end{frame}
%\begin{frame}{\fn}
%\begin{block}{\textbf{Theorem 2.2 property 3.}}
%	Let $A$ be an $m\times n$ matrix and $c$ a scalar.
%	If $cA=O_{mn}$, then $c=0$ or $A=O_{mn}$.
%\end{block}
%\begin{exercise}
%	\includegraphics[width=1.5in]{../images/stop}
%\end{exercise}
%\end{frame}
\end{document}

