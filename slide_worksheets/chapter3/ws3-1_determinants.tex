\documentclass[handout]{beamer}
%\usepackage[margin=1in]{geometry}
\usepackage{amsthm,amsmath,amsfonts,hyperref,graphicx,color,multicol}
\usepackage{enumitem,tikz}

%%%%%%%%%%
%Beamer Template Customization
%%%%%%%%%%
\setbeamertemplate{navigation symbols}{}
\setbeamertemplate{theorems}[ams style]
\setbeamertemplate{blocks}[rounded]

\definecolor{Blu}{RGB}{43,62,133} % UWEC Blue
\setbeamercolor{structure}{fg=Blu} % Titles

%Unnumbered footnotes:
\newcommand{\blfootnote}[1]{%
	\begingroup
	\renewcommand\thefootnote{}\footnote{#1}%
	\addtocounter{footnote}{-1}%
	\endgroup
}


%%%%%%%%%%
%Custom Commands
%%%%%%%%%%
\newcommand{\R}{\mathbb{R}}
\newcommand{\veca}{\vec{a}}
\newcommand{\vecb}{\vec{b}}
\newcommand{\vece}{\vec{e}}
\newcommand{\vecu}{\vec{u}}
\newcommand{\vecv}{\vec{v}}
\newcommand{\vecw}{\vec{w}}
\newcommand{\vecx}{\vec{x}}
\newcommand{\zerovector}{\vec{0}}

\newcommand{\ds}{\displaystyle}

\newcommand{\fn}{\insertframenumber}

\newcommand{\rank}{\operatorname{rank}}
\newcommand{\adj}{\operatorname{adj}}

\newcommand{\blank}[1]{\underline{\hspace*{#1}}}


%%%%%%%%%%
%Custom Theorem Environments
%%%%%%%%%%
\theoremstyle{definition}
\newtheorem{exercise}{Exercise}
\newtheorem{question}[exercise]{Question}
\newtheorem*{defn}{Definition}
\newtheorem*{exa}{Example}
\newtheorem*{disc}{Group Discussion}
\newtheorem*{nb}{Note}
\newtheorem*{recall}{Recall}
\renewcommand{\emph}[1]{{\color{blue}\texttt{#1}}}

\definecolor{Gold}{RGB}{237, 172, 26}
%Statement block
\newenvironment{statementblock}[1]{%
	\setbeamercolor{block body}{bg=Gold!20}
	\setbeamercolor{block title}{bg=Gold}
	\begin{block}{\textbf{#1.}}}{\end{block}}



\begin{document}
	\title{Math 324: Linear Algebra}
	\subtitle{Section 3.1: The Determinant of a Matrix}
	\author{Mckenzie West}
	\date{Last Updated: \today}
\begin{frame}
\maketitle
\end{frame}

\begin{frame}{\insertframenumber}
	\begin{block}{\textbf{Last Time.}}
	\begin{itemize}[label=--]
		\item Proof by Induction
	\end{itemize}
	\end{block}
	\begin{block}{\textbf{Today.}}
		\begin{itemize}[label=--]
			\item The determinant of a $2\times 2$ matrix
			\item Minors
			\item Cofactors
			\item The determinant of an $n\times n$ matrix.
		\end{itemize}
	\end{block}
\end{frame}
\begin{frame}{\fn}
	\begin{defn}
		Generally speaking, the \emph{determinant} of an $n\times n$ matrix is a real number that we can associate to it that indicates whether $A\vec x=\vec 0$ has nontrivial solutions.
	\end{defn}

	\begin{nb}
		Recall that $A\vec x=\vec 0$ has a unique solution if and only if $A$ is invertible.  Therefore, the determinant can also indicate whether $A$ is invertible.
	\end{nb}
	\begin{question}
		In light of this discussion, what do you think the determinant of the matrix $A=\begin{bmatrix}a_{11}&a_{12}\\a_{21}&a_{22}\end{bmatrix}$ is?
	\end{question}
\end{frame}

\begin{frame}{\fn}
	\begin{block}{\textbf{Notation.}}
		If $A$ is an $2\times 2$ matrix, denote the determinant of $A$ by $|A|$, $\det(A)$,\[\text{or } \begin{vmatrix}a_{11}&a_{12}\\a_{21}&a_{22}\end{vmatrix} = a_{11}a_{22}-a_{12}a_{21}.\]
	\end{block}
	\begin{exercise}
		Compute each of the following:
		\begin{multicols}{3}
			\begin{enumerate}[label=(\alph*)]
			\item $\left|\begin{array}{rr}
			-5 & 2 \\
			4 & -3
			\end{array}\right|$
			\item $\left|\begin{array}{rr}
			1 & -\frac{1}{3} \\
			-2 & -5
			\end{array}\right|$
			\item $\left|\begin{array}{rr}
			0 & 0 \\
			1 & -1
			\end{array}\right|$
		\end{enumerate}
		\end{multicols}
	\end{exercise}
	\begin{question}
		What are the possible values for $|A|$?
	\end{question}
\end{frame}
\begin{frame}{\fn}
	\begin{defn}
		If $A$ is a square matrix, then the \emph{$(i,j)$-minor}, $M_{ij}$, is the determinant of the matrix obtained by deleting the $i$th row and $j$th column of $A$.
	\end{defn}
	\begin{exercise}
		Determine the given minor:
		\begin{enumerate}[label=(\alph*)]
			\item $(1,2)$-minor of $\begin{bmatrix}
			0 & 3 & 3 \\
			4 & 3 & 5 \\
			-1 & 0 & 3
			\end{bmatrix}$
			\item $(3,3)$-minor of $\begin{bmatrix}
			-1 & 2 & 4 \\
			2 & -1 & -3 \\
			1 & 4 & 5
			\end{bmatrix}$
			\item $(2,1)$-minor of $\begin{bmatrix}
			a_{11}&a_{12}&a_{13}\\
			a_{21}&a_{22}&a_{23}\\
			a_{31}&a_{32}&a_{33}
			\end{bmatrix}$
		\end{enumerate}
	\end{exercise}
\end{frame}
\begin{frame}{\fn}
\begin{defn}
	If $A$ is a square matrix, then the \emph{$(i,j)$-cofactor}, $C_{ij}$, is a signed version of the $(i,j)$-minor given by $C_{ij}=(-1)^{i+j}M_{ij}$.
\end{defn}
\begin{exercise}
	Determine the given cofactor:
	\begin{enumerate}[label=(\alph*)]
		\item $(1,2)$-cofactor of $\begin{bmatrix}
		0 & 3 & 3 \\
		4 & 3 & 5 \\
		-1 & 0 & 3
		\end{bmatrix}$
		\item $(3,3)$-cofactor of $\begin{bmatrix}
		-1 & 2 & 4 \\
		2 & -1 & -3 \\
		1 & 4 & 5
		\end{bmatrix}$
		\item $(2,1)$-cofactor of $\begin{bmatrix}
		a_{11}&a_{12}&a_{13}\\
		a_{21}&a_{22}&a_{23}\\
		a_{31}&a_{32}&a_{33}
		\end{bmatrix}$
	\end{enumerate}
\end{exercise}
\end{frame}
\begin{frame}{\fn}
	\begin{nb}
		The sign of the cofactor alternates across rows and columns:
		\begin{center}
			\begin{tabular}{ccc}
			$\begin{bmatrix}+&-&+\\-&+&-\\+&-&+\end{bmatrix}$&
			
			$\begin{bmatrix}+&-&+&-\\-&+&-&+\\+&-&+&-\\-&+&-&+\end{bmatrix}$&
			
			$\begin{bmatrix}+&-&+&-&+&\cdots\\-&+&-&+&-&\cdots\\+&-&+&-&+&\cdots\\
			-&+&-&+&-&\cdots\\+&-&+&-&+&\cdots\\\vdots&\vdots&\vdots&\vdots&\vdots&\end{bmatrix}$\\
			$3\times 3$&$4\times 4$ & $n\times n$
		\end{tabular}
		\end{center}
	\end{nb}
\end{frame}
\begin{frame}{\fn}
	\begin{defn}
		If $A$ is an $n\times n$ matrix ($n\geq 2$), the \emph{determinant} of $A$ is the value
			\[\det(A)=|A|=a_{11}C_{11}+a_{12}C_{12}+\cdots+a_{1n}C_{1n}=\sum_{j=1}^n a_{1j}C_{1j}.\]
	\end{defn}
	\begin{nb}
		We call this an inductive definition because we use the determinant of one size smaller matrix to define the determinant of the next.
	\end{nb}
	\begin{exercise}
		Use the definition given here to compute	
			\[\begin{vmatrix}
			0 & 3 & 3 \\
			4 & 3 & 5 \\
			-1 & 0 & 3
			\end{vmatrix}\]
	\end{exercise}
\end{frame}
\begin{frame}{\fn}
	\begin{block}{\textbf{Brain Break.}}
		What band were you obsessed with in middle school?
		\begin{center}
			\includegraphics[width=2in]{../images/band}
		\end{center}
	\end{block}
\end{frame}
\begin{frame}{\fn}
	\begin{nb}
		Notice how convenient it was that one of the entries in the first row of the last matrix was 0.  We didn't actually have to compute the $(1,1)$-cofactor to compute the determinant.
		
		We can actually take advantage of rows (and columns!) that have a lot of zeros when we take determinants.
	\end{nb}
\end{frame}
\begin{frame}{\fn}
	\begin{statementblock}{Theorem 3.1 (Expansion by Cofactors)}
		Let $A$ be a square matrix of order $n$.  Then the determinant of $A$ is given by:
			\begin{enumerate}[label=(\alph*)]
				\item the \emph{$i$-th row expansion} $$\det(A)=|A|=a_{i1}C_{i1}+a_{i2}C_{i2}+\cdots+a_{in}C_{in}=\sum_{j=1}^n a_{ij}C_{ij}.$$
				\item or the \emph{$j$-th column expansion}
				$$\det(A)=|A|=c_{1j}C_{1j}+a_{2j}C_{2j}+\cdots+a_{nj}C_{nj}=\sum_{i=1}^n a_{ij}C_{ij}.$$
			\end{enumerate}
	\end{statementblock}
\end{frame}
\begin{frame}{\fn}
	\begin{exercise}
		Compute the determinant of the following matrix in two different ways: (a) the 3rd row expansion and (b) the 4th column expansion:
		$$\left[\begin{array}{rrrr}
			-4 & -2 & 5 & 0 \\
			-3 & -1 & 0 & 0 \\
			0 & 0 & -5 & 0 \\
			-5 & 0 & 0 & 3
		\end{array}\right].$$
		
	 	Hopefully you got the same answer in both cases. Why do you think that happened?
	\end{exercise}
\end{frame}
\begin{frame}{\fn}
	\begin{exercise}
		Use your choice of row/column cofactor expansion(s) to compute each of the following:
		\begin{multicols}{2}
			\begin{enumerate}[label=(\alph*)]
			\item $\begin{vmatrix}
			-2 & 1 & 0 \\
			4 & 0 & -2 \\
			-4 & -1 & 5
			\end{vmatrix}$
			\item $\begin{vmatrix}
			2 & 2 & -2 & 1 \\
			0 & -1 & 5 & -2 \\
			0 & 0 & -3 & 3 \\
			0 & 0 & 0 & 4
			\end{vmatrix}$
			\item $\begin{vmatrix}
				-5 & 5 & 0 & -4 \\
				5 & 4 & 0 & 0 \\
				-1 & -2 & 2 & -3 \\
				-4 & 0 & 0 & 2
			\end{vmatrix}$
			\item $\begin{vmatrix}
			1 & -3 & -2 & 0 \\
			-3 & 0 & 1 & 1 \\
			5 & 0 & 0 & 0 \\
			-5 & 0 & -4 & -4
			\end{vmatrix}$
		\end{enumerate}
		\end{multicols}
	\end{exercise}
\end{frame}
\begin{frame}{\fn}
	\begin{exercise}
		Determine the values of $\lambda$ for which the determinant of the given matrix is zero:
			\begin{enumerate}[label=(\alph*)]
				\item $\begin{bmatrix}
				\lambda+2&0\\-5&\lambda-3
				\end{bmatrix}$
				\item $\begin{bmatrix}
					\lambda&1\\7&\lambda+6
				\end{bmatrix}$
				\item $\begin{bmatrix}
				\lambda+1 & 0 & 0 \\
				1 & \lambda-2 & 1 \\
				1 & 0 & \lambda
				\end{bmatrix}$
			\end{enumerate}
	\end{exercise}
\end{frame}
\begin{frame}{\fn}
	\begin{exercise}
		Evaluate the determinant to verify the equation:
		\begin{enumerate}[label=(\alph*)]
			\item $\begin{vmatrix}a&b\\c&d\end{vmatrix}=-\begin{vmatrix}c&d\\a&b\end{vmatrix}$
			\item $\begin{vmatrix}
				1&x&x^2\\1&y&y^2\\1&z&z^2
			\end{vmatrix}=(y-x)(z-x)(z-y)$
	\end{enumerate}
	\end{exercise}
\end{frame}
\begin{frame}{\fn}
	\begin{exercise}
		Show that the system of linear equations
			$$\begin{array}{rcl}
				ax+by&=&e\\
				cx+dy&=&f
			\end{array}$$
		has a unique solution if and only if the determinant of the coefficient matrix is nonzero.
	\end{exercise}
\end{frame}
\end{document}

