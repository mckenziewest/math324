\documentclass{beamer}
%\usepackage[margin=1in]{geometry}
\usepackage{amsthm,amsmath,amsfonts,hyperref,graphicx,color,multicol}
\usepackage{enumitem,tikz}

%%%%%%%%%%
%Beamer Template Customization
%%%%%%%%%%
\setbeamertemplate{navigation symbols}{}
\setbeamertemplate{theorems}[ams style]
\setbeamertemplate{blocks}[rounded]

\definecolor{Blu}{RGB}{43,62,133} % UWEC Blue
\setbeamercolor{structure}{fg=Blu} % Titles

%Unnumbered footnotes:
\newcommand{\blfootnote}[1]{%
	\begingroup
	\renewcommand\thefootnote{}\footnote{#1}%
	\addtocounter{footnote}{-1}%
	\endgroup
}


%%%%%%%%%%
%Custom Commands
%%%%%%%%%%
\newcommand{\R}{\mathbb{R}}
\newcommand{\veca}{\vec a}
\newcommand{\vecb}{\vec b}
\newcommand{\vecu}{\vec{u}}
\newcommand{\vecv}{\vec{v}}
\newcommand{\vecw}{\vec{w}}
\newcommand{\vecx}{\vec{x}}
\newcommand{\zerovector}{\vec{0}}

\newcommand{\ds}{\displaystyle}

\newcommand{\fn}{\insertframenumber}

\newcommand{\rank}{\operatorname{rank}}

\newcommand{\blank}[1]{\underline{\hspace*{#1}}}


%%%%%%%%%%
%Custom Theorem Environments
%%%%%%%%%%
\theoremstyle{definition}
\newtheorem{exercise}{Exercise}
\newtheorem{question}[exercise]{Question}
\newtheorem*{defn}{Definition}
\newtheorem*{exa}{Example}
\newtheorem*{disc}{Group Discussion}
\newtheorem*{nb}{Note}
\newtheorem*{recall}{Recall}
\renewcommand{\emph}[1]{{\color{blue}\texttt{#1}}}

\definecolor{Gold}{RGB}{237, 172, 26}
%Statement block
\newenvironment{statementblock}[1]{%
	\setbeamercolor{block body}{bg=Gold!20}
	\setbeamercolor{block title}{bg=Gold}
	\begin{block}{\textbf{#1.}}}{\end{block}}





\begin{document}
	\title{Math 324: Linear Algebra}
	\subtitle{Section 3.1: The Determinant of a Matrix\\Section 3.2: Determinants and Elementary Operations}
	\author{Mckenzie West}
	\date{Last Updated: \today}
\begin{frame}
\maketitle
\end{frame}

\begin{frame}{\insertframenumber}
	\begin{block}{\textbf{Last Time.}}
	\begin{itemize}[label=--]
		\item The determinant of a $2\times 2$ matrix
		\item Minors
		\item Cofactors
		\item The determinant of an $n\times n$ matrix.
	\end{itemize}
	\end{block}
	\begin{block}{\textbf{Today.}}
		\begin{itemize}[label=--]
			\item Determinants of Triangular Matrices
			\item How Elementary Row Operations Affect Determinants
		\end{itemize}
	\end{block}
\end{frame}
\begin{frame}{\fn}
	\begin{defn}
		An \emph{upper triangular matrix} is a square matrix for which every entry below the main diagonal is zero.  A \emph{lower triangular matrix} is a square matrix for which every entry above the main diagonal is zero.
		
		A \emph{diagonal matrix} is a matrix that is both upper triangular and lower triangular.
	\end{defn}
	\begin{exercise}
		Compute the determinant of each of the following triangular matrices, using the appropriate cofactor expansion.
		
				 $\begin{bmatrix}
					-3 & 0  \\
					12 & 8 
					\end{bmatrix}$
				\hfill $\begin{bmatrix}
					-6 & 15 & -4 \\
					0 & -1 & 6 \\
					0 & 0 & 0
					\end{bmatrix}$
				\hfill $\begin{bmatrix}
					-3 & 0 & 0 &0 \\
					0 & -7 &0 & 0 \\
					0 & 0 & 2 & 0 \\
					0 & 0 & 0 & 3
					\end{bmatrix}$
	\end{exercise}
\end{frame}

\begin{frame}{\fn}
	\begin{statementblock}{Theorem 3.2}
		If $A$ is a triangular matrix of order $n$, then its determinant is the product of the entries on the main diagonal.  That is,	\[\det(A)=|A|=a_{11}a_{22}a_{33}\cdots a_{nn}.\]
	\end{statementblock}
	
	\begin{exercise}
		Complete the proof of this Theorem on your worksheet.
		
		Note that due to the inductive definition of determinants, we often use induction to prove properties of determinants.
	\end{exercise}
\end{frame}

\begin{frame}{\fn}
	\begin{exercise}
		Use Theorem 3.2 to compute:
		\begin{enumerate}[label=(\alph*)]
			\item $\det(I_n)$
			\item $\det(O_n)$
			\item $\det(4I_3)$
		\end{enumerate}
	\end{exercise}
\end{frame}

\begin{frame}{\fn}
	\begin{block}{\textbf{Brain Break.}}
		What fictional character would you like to meet?
			\begin{center}
				\includegraphics[width=2in]{images/meet}
			\end{center}
	\end{block}
\end{frame}

\begin{frame}{\fn}
	\begin{exercise}
		Each person at your table should claim one of these four matrices.
		
		\[\begin{bmatrix}5 & 4 \\0 & 1\end{bmatrix},\
		\begin{bmatrix}2 & -3 \\1 & 5\end{bmatrix},\
		\begin{bmatrix}1 & -3 \\-3 & 0\end{bmatrix},\text{ or }
		\begin{bmatrix}-2 & 1 \\	2 & -3\end{bmatrix}\!.\]
		
		\begin{enumerate}[label=(\alph*)]
			\item Compute the determinant of your matrix.
			\item Compute the determinant of your matrix after performing the elementary row operation $R_1\leftrightarrow R_2$.
			\item Reset your matrix, now compute the determinant after performing the elementary row operation $R_2-4R_1\rightarrow R_2$.
			\item Reset your matrix again, now compute the determinant after performing the elementary row operation $3 R_1\rightarrow R_1$.
		\end{enumerate}
	\end{exercise}
\end{frame}

\begin{frame}{\fn}
	\begin{exercise}
		Using your answers from Exercise 4, fill in the blanks of Theorem 3.3 that describe how elementary row operations affect determinants.
	\end{exercise}
	\begin{statementblock}{Theorem 3.3}
		Let $A$ and $B$ be square matrices.
		\begin{enumerate}[label=\textbf{\arabic*.}]
			\item When $B$ is obtained from $A$ by interchanging two rows of $A$, $\det(B) = \underline{\hspace*{.75in}}$.
			\item When $B$ is obtained from $A$ by adding a multiple of a row of $A$ to another row of $A$, $\det(B)=\underline{\hspace*{.75in}}$.
			\item When $B$ is obtained from $A$ by multiplying a row of $A$ by a nonzero constant $c$, $\det(B)=\underline{\hspace*{.75in}}$.
		\end{enumerate}
	\end{statementblock}
\end{frame}

\begin{frame}{\fn}
	\begin{exa}
		We now have another way to compute determinants, row reducing to upper triangular and keeping track of our path:
		
		\[
		\begin{bmatrix}0 & 4 & -2 \\0 & 1 & 3 \\-3 & 2 & 0\end{bmatrix}
		\xrightarrow{R_1\leftrightarrow R_3}
		\begin{bmatrix}-3 & 2 & 0\\0 & 1 & 3 \\0 & 4 & -2 \end{bmatrix}
		\xrightarrow{R_3-4R_2\rightarrow R_3}
		\begin{bmatrix}-3 & 2 & 0\\0 & 1 & 3 \\0 & 0 & -14 \end{bmatrix}.
		\]
		
		Therefore, the determinant of the starting matrix is
		\[\begin{array}{rl}\begin{vmatrix}0 & 4 & -2 \\0 & 1 & 3 \\-3 & 2 & 0\end{vmatrix}
		&=-
		\begin{vmatrix}-3 & 2 & 0\\0 & 1 & 3 \\0 & 4 & -2 \end{vmatrix}
		=-
		\begin{vmatrix}-3 & 2 & 0\\0 & 1 & 3 \\0 & 0 & -14 \end{vmatrix}\\\\
		&=
		-(-3)(1)(-14)=-72.
		\end{array}\]
	\end{exa}
\end{frame}
\begin{frame}{\fn}
	\begin{exercise}
		Use elementary row operations to compute:
			\[\begin{vmatrix}
			4 & 2 & 6 \\
			0 & 2 & 3 \\
			2 & 5 & 5
			\end{vmatrix}\]
	\end{exercise}
\end{frame}
\begin{frame}{\fn}
\begin{defn}
	The following are \emph{elementary column operations}:
	\begin{enumerate}[label=\textbf{\arabic*.}]
		\item swap two columns, denoted $C_i\leftrightarrow C_j$,
		\item add a multiple of one column to another, denoted $C_i+cC_j\rightarrow C_i$,
		\item multiply a column of $B$ by a nonzero scalar $c$, denoted $cC_i\rightarrow C_i$.
	\end{enumerate} 
	We call two matrices \emph{column equivalent} if one of them can be obtained from the other by elementary column operations.
\end{defn}
\begin{nb}
	Elementary \textbf{column} operations affect the determinant in the exact same way that their related elementary \textbf{row} operations do.
\end{nb}
\end{frame}
\begin{frame}{\fn}
\begin{exercise}
	Make use of elementary column operations to compute the determinant:
	\[\begin{vmatrix}
	-2 & 1 & 4 \\
	3 & 3 & -6 \\
	4 & 0 & -8
	\end{vmatrix}\]
\end{exercise}
\begin{exercise}
	Make use of elementary row and/or column operations to compute each determinant:
	\[\text{(a)}\ \begin{vmatrix}
	6 & 5 & 4 \\
	0 & 3 & 2 \\
	3 & 5 & 4
	\end{vmatrix}\hskip .5in \text{(b)}\ \begin{vmatrix}
	2 & 0 & 0 & 5 \\
	-2 & 3 & 3 & 4 \\
	1 & 3 & -3 & 4 \\
	-2 & 3 & -2 & 2
	\end{vmatrix}\]
\end{exercise}
\end{frame}
\end{document}

