\documentclass{beamer}
%\usepackage[margin=1in]{geometry}
\usepackage{amsthm,amsmath,amsfonts,hyperref,graphicx,color,multicol}
\usepackage{enumitem,tikz}

%%%%%%%%%%
%Beamer Template Customization
%%%%%%%%%%
\setbeamertemplate{navigation symbols}{}
\setbeamertemplate{theorems}[ams style]
\setbeamertemplate{blocks}[rounded]

\definecolor{Blu}{RGB}{43,62,133} % UWEC Blue
\setbeamercolor{structure}{fg=Blu} % Titles

%Unnumbered footnotes:
\newcommand{\blfootnote}[1]{%
	\begingroup
	\renewcommand\thefootnote{}\footnote{#1}%
	\addtocounter{footnote}{-1}%
	\endgroup
}


%%%%%%%%%%
%Custom Commands
%%%%%%%%%%
\newcommand{\R}{\mathbb{R}}
\newcommand{\veca}{\vec{a}}
\newcommand{\vecb}{\vec{b}}
\newcommand{\vecu}{\vec{u}}
\newcommand{\vecv}{\vec{v}}
\newcommand{\vecw}{\vec{w}}
\newcommand{\vecx}{\vec{x}}
\newcommand{\zerovector}{\vec{0}}

\newcommand{\ds}{\displaystyle}

\newcommand{\fn}{\insertframenumber}

\newcommand{\rank}{\operatorname{rank}}
\newcommand{\adj}{\operatorname{adj}}

\newcommand{\blank}[1]{\underline{\hspace*{#1}}}


%%%%%%%%%%
%Custom Theorem Environments
%%%%%%%%%%
\theoremstyle{definition}
\newtheorem{exercise}{Exercise}
\newtheorem{question}[exercise]{Question}
\newtheorem*{defn}{Definition}
\newtheorem*{exa}{Example}
\newtheorem*{disc}{Group Discussion}
\newtheorem*{nb}{Note}
\newtheorem*{recall}{Recall}
\renewcommand{\emph}[1]{{\color{blue}\texttt{#1}}}

\definecolor{Gold}{RGB}{237, 172, 26}
%Statement block
\newenvironment{statementblock}[1]{%
	\setbeamercolor{block body}{bg=Gold!20}
	\setbeamercolor{block title}{bg=Gold}
	\begin{block}{\textbf{#1.}}}{\end{block}}





\begin{document}
	\title{Math 324: Linear Algebra}
	\subtitle{Section 3.3: Properties of Determinants\\
	Section 3.4: Cramer's Rule}
	\author{Mckenzie West}
	\date{Last Updated: \today}
\begin{frame}
\maketitle
\end{frame}

\begin{frame}{\insertframenumber}
	\begin{block}{\textbf{Last Time.}}
	\begin{itemize}[label=--]
		\item Determinants of Products
	\end{itemize}
	\end{block}
	\begin{block}{\textbf{Today.}}
		\begin{itemize}[label=--]
			\item More Properties of Determinants
			\item Applications of Determinants
		\end{itemize}
	\end{block}
\end{frame}

\begin{frame}{\fn\ - Last Time}
	\begin{statementblock}{Theorem 3.5}
		If $A$ and $B$ are square matrices of order $n$, then \[\det(AB)=\det(A)\det(B).\]
	\end{statementblock}
	\begin{statementblock}{Theorem 3.8}
	If $A$ is an $n\times n$ invertible matrix, then $\ds \det(A^{-1})=\frac{1}{\det(A)}$.
	\end{statementblock}
\end{frame}
\begin{frame}{\fn\ - Last Time with a New Exercise}
	\begin{statementblock}{Equivalent Conditions for a Nonsingular Matrix}
		If $A$ is an $n\times n$ matrix, then the following are equivalent:
		\begin{enumerate}[label=\textbf{\arabic*.}]
			\item $A$ is invertible.
			\item $A\vecx=\vecb$ has a unique solution for every $n\times 1$ column matrix $\vecb$.
			\item $A\vecx=O$ has only the trivial solution.
			\item $A$ is row equivalent to $I_n$.
			\item $A$ can be written as the product of elementary matrices.
			\item $\det(A)\neq0$.  (\textbf{Theorem 3.7})
		\end{enumerate}
	\end{statementblock}
	\begin{exercise}
		Compute the determinants of each of the following to determine which are invertible.
		
			\begin{center}
				(a)\ $\begin{bmatrix}
			4 & 5 & 4 \\
			0 & 3 & 2 \\
			0 & 0 & 3
			\end{bmatrix}$
			\hfill
			(b)\ $\begin{bmatrix}
			3 & 0 & 0 \\
			4 & 3 & 0 \\
			5 & 4 & 0
			\end{bmatrix}$
			\hfill
			(c)\ $\begin{bmatrix}
				0 & 0 & 0 & 1 \\
				1 & 1 & -1 & -1 \\
				0 & 1 & 1 & -1 \\
				-1 & 1 & 0 & -1
			\end{bmatrix}$
			\end{center}
	\end{exercise}
\end{frame}

\begin{frame}[fragile]\frametitle{\fn}
	\begin{exercise}
%		Use the following Sage code to produce a random $2\times 2$ matrix for each person at the table: (\url{https://sagecell.sagemath.org/})
%		{\footnotesize \begin{verbatim}
%			A = matrix([[randint(1,5) for j in [1,2]] for i in [1,2]])
%			print("A=\n",A)
%			B = matrix([[randint(-3,3) for j in [1,2]] for i in [1,2]])
%			print("B=\n",B)
%			C = matrix([[randint(-5,-1) for j in [1,2]] for i in [1,2]])
%			print("C=\n",C)
%			\end{verbatim}}
		Each person claim one of the following three matrices,
			\[
			\left[\begin{array}{rr}
			5 & 2 \\
			1 & 0
			\end{array}\right],\ 
			\left[\begin{array}{rr}
			3 & 0 \\
			-2 & -1
			\end{array}\right],
			\text{ or } 
			\left[\begin{array}{rr}
			0 & 5 \\
			1 & -1
			\end{array}\right]\]
		For your matrix $A$, using the definition or properties of the determinant find:
			\begin{center} (a) $|A^T|$, (b) $|A^2|$, (c) $|AA^T|$, (d) $|2A|$, and (e) $|A^{-1}|$.\end{center}
	\end{exercise}
\end{frame}
\begin{frame}
	\begin{exercise}
		Fill in the blanks of the Theorem:
		\begin{statementblock}{Theorems 3.6 and 3.9}
			Let $A$ be an $n\times n$ matrix.
			\begin{enumerate}[label= ]
				\item[3.6:] If $c$ is a scalar, then $\det(cA)=\underline{\hspace*{.75in}}$.
				\item[3.9:] Moreover $\det(A^T)=\underline{\hspace*{.75in}}$.
				\item[3.?:] If $k$ is a positive integer, then $\det(A^k)=\underline{\hspace*{.75in}}$.
			\end{enumerate}
		\end{statementblock}
	\end{exercise}
\end{frame}
\begin{frame}{\fn}
	\begin{recall}
		The $(i,j)$-cofactor of the $n\times n$ matrix $A=[a_{ij}]$ is
			\[C_{ij}=(-1)^{i+j}\det(M_{ij}),\]
		where $M_{ij}$ is the matrix obtained by removing the $i$-th row and $j$-th column of $A$.
	\end{recall}
	\begin{defn}
		The \emph{matrix of cofactors} of the $n\times n$ matrix $A$ is the $n\times n$ matrix given by:
			\[\begin{bmatrix} C_{11} & C_{12} & \cdots & C_{1n}\\ C_{21} & C_{22} & \cdots & C_{2n}\\
			\vdots&\vdots&&\vdots\\ C_{n1} & C_{n2} & \cdots & C_{nn}\end{bmatrix}.\]
	\end{defn}
\end{frame}
\begin{frame}{\fn}
	\begin{defn}
		The \emph{adjoin} of the $n\times n$ matrix $A$ is the transpose of the matrix of cofactors of $A$ and is denoted $\adj(A)$.  In particular:
		
			\[\adj(A)=\begin{bmatrix} C_{11} & C_{21} & \cdots & C_{n1}\\ C_{12} & C_{22} & \cdots & C_{n2}\\
			\vdots&\vdots&&\vdots\\ C_{1n} & C_{2n} & \cdots & C_{nn}\end{bmatrix}\]
	\end{defn}
\end{frame}
\begin{frame}{\fn}
	\begin{statementblock}{Theorem 3.10}
		If $A$ is an $n\times n$ invertible matrix then $\ds A^{-1}=\frac{1}{\det(A)}\adj(A)$.
	\end{statementblock}
	\begin{exercise}
		Find $\adj(A)$ for \[A=\begin{bmatrix}1 & -2 & -1 \\
		2 & 1 & 0 \\
		0 & 0 & 2\end{bmatrix}.\]
		
		Compute $A\adj(A)$.
		
		What is $A^{-1}$?
	\end{exercise}
\end{frame}
\begin{frame}{\fn}
	\begin{exercise}
		Proof IDEA for Theorem 3.10.\vskip .25in
		
		Take $A=\begin{bmatrix}
			a_{11} & a_{12} & a_{13} \\
			a_{21} & a_{22} & a_{23} \\
			a_{31} & a_{32} & a_{33} \\
		\end{bmatrix}$ and 
		$\adj(A)=\begin{bmatrix}
		C_{11}&C_{21}&C_{31}\\
		C_{12}&C_{22}&C_{32}\\
		C_{13}&C_{23}&C_{33}
		\end{bmatrix}$.
		
		Then the (1,1)-entry of $A\adj(A)$ is
			\[a_{11}C_{11}+a_{12}C_{12}+a_{13}C_{13}=\blank{2in}.\]
		Then the (2,2)-entry of $A\adj(A)$ is
			\[a_{21}C_{21}+a_{22}C_{22}+a_{23}C_{23}=\blank{2in}.\]
		Then the (3,3)-entry of $A\adj(A)$ is
			\[a_{31}C_{31}+a_{32}C_{32}+a_{33}C_{33}=\blank{2in}.\]
	\end{exercise}
\end{frame}
\begin{frame}{\fn}
	\addtocounter{exercise}{-1}
	\begin{exercise}[Cont.]
		On the other hand the (1,2)-entry of $A\adj(A)$ is
		\vskip .25in
			$a_{11}C_{21}+a_{12}C_{22}+a_{13}C_{23}$
			$$\begin{array}{l}=
			-a_{11}\begin{vmatrix}a_{12}&a_{13}\\a_{32}&a_{33}\end{vmatrix}
			+a_{12}\begin{vmatrix}a_{11}&a_{13}\\a_{31}&a_{33}\end{vmatrix}
			-a_{13}\begin{vmatrix}a_{11}&a_{12}\\a_{31}&a_{32}\end{vmatrix}\\
			=\text{ middle row cofactor expansion for }\begin{vmatrix}
			a_{11} & a_{12} & a_{13} \\
			a_{11} & a_{12} & a_{13} \\
			a_{31} & a_{32} & a_{33} \\
			\end{vmatrix}\\=0\text{\ \ \ (why:)\blank{2in}}.\end{array}$$
		\vskip .25in
		What are the $(1,3)$, $(2,1)$, $(2,3)$, $(3,1)$, and $(3,2)$-entries of the product $A\adj(A)$?
	\end{exercise}
\end{frame}
\begin{frame}{\fn}
	\begin{block}{\textbf{Brain Break.}}
		Today the Jewish celebration of Purim will end.  Purim is an annual holiday commemorating the salvation of the Jewish people from the ancient Persian Empire, as written in the Megillah (book of Esther). On Purim, observers celebrate friendship and community by sending gifts of food and drink to friends. Who would you send such gifts to and why?
		\vskip .5in
		* Today is also the Hindu holiday of Holi and the Sikh holiday of Hola Mohalla.
		\vskip .5in
		{\footnotesize Information found at \url{https://www.chabad.org/holidays/purim/article_cdo/aid/645309/jewish/What-Is-Purim.htm}\\ and \url{https://www.interfaith-calendar.org/2020.htm}}
	\end{block}
\end{frame}
\begin{frame}{\fn}
	\begin{block}{\textbf{Cramer's Rule}}
		Consider the system of equations 
		\begin{eqnarray*}
			a_{11}x_1+a_{12}x_2&=&b_1\\a_{21}x_1+a_{22}x_2&=&b_2
		\end{eqnarray*}
		which has coefficient matrix $A=\begin{bmatrix}a_{11}&a_{12}\\a_{21}&a_{22}\end{bmatrix}$.
		\emph{Cramer's rule} says the following: 
			\begin{enumerate}[label=\textbf{\arabic*.}]
				\item Check that $A$ is invertible.
				\item Let $A_1=\begin{bmatrix} b_1 & a_{12}\\b_2&a_{22}\end{bmatrix}$ and $A_2=\begin{bmatrix}a_{11}&b_1\\a_{21}&b_2\end{bmatrix}$, obtained by replacing the given columns of $A$ by the constant terms.
				\item Compute the determinants $|A|$, $|A_1|$, and $|A_2|$.
				\item Then $\ds x_1=\frac{|A_1|}{|A|}$ and $\ds x_2=\frac{|A_2|}{|A|}$ is the unique solution to the system.
			\end{enumerate}
	\end{block}
\end{frame}
\begin{frame}{\fn}
	\begin{exercise}
		Use Cramer's rule to solve the system:
			\begin{eqnarray*}
				x+y&=& 2\\
				3x-y&=&0.
			\end{eqnarray*}
	\end{exercise}
\end{frame}
\begin{frame}{\fn}
	\begin{statementblock}{Theorem 3.11: Cramer's Rule}
		If a system of $n$ linear equations in $n$ variables has a coefficient matrix $A$ with a nonzero determinant $|A|$, then the unique solution to the system is
			\[x_1=\frac{\det(A_1)}{\det(A)},\ x_2=\frac{\det(A_2)}{\det(A)},\ 
			\dots,\ x_n=\frac{\det(A_n)}{\det(A)}\]
		where the $i$-th column of $A_i$ is the column of constants in the system of equations, all other columns of $A_i$ equal those of $A$.
	\end{statementblock}
\end{frame}
\begin{frame}{\fn}
	\begin{exercise}
		Use Cramer's Rule to solve the system of linear equations for $\vecx$.
		\[\begin{array}{rcrcrcr}
			2x&+&y&-&3z&=&4\\
			4x&&&+&2z&=&0\\
			-2x&+&3y&&&=&8
		\end{array}\]
	\end{exercise}
\end{frame}
\begin{frame}{\fn}
	\begin{statementblock}{Proposition}
		Let $A$ be an invertible matrix with integer entries.
		Then $\det(A)=\pm 1$ if and only if all of the entries of $A^{-1}$ are integers.
	\end{statementblock}
	\begin{exercise}
		Prove the proposition using the following steps:
		\begin{enumerate}[label=(\alph*)]
			\item Forward direction.
				\begin{enumerate}[label=(\roman*)]
					\item Assume $|A|=\pm 1$.
					\item Use the formula $A^{-1}=\frac{1}{\det(A)}\adj(A)$ and the definition of cofactors to explain why $A^{-1}$ has integer entries.
				\end{enumerate}
			\item Reverse Direction
				\begin{enumerate}[label=(\roman*)]
					\item Assume that $A$ and $A^{-1}$ have integer entries.
					\item Use Theorem 3.5 to explain why $|A||A^{-1}|=1$.
					\item Use the definition of the determinant to explain why $|A|$ and $|A^{-1}|$ are both integers.
					\item Conclude that $|A|=|A^{-1}|=\pm 1$.
				\end{enumerate}
		\end{enumerate}
	\end{exercise}
\end{frame}

\begin{frame}{\fn}
	\begin{exercise}
		Use Cramer's Rule to solve the system of linear equations for $x$ and $y$ in terms of $k$.  Then determine which values of $k$ will make the system inconsistent.
			\[\begin{array}{rcrcr}
				kx&+&(1-k)y&=&1\\
				(1-k)x &+& ky&=&3
			\end{array}\]
	\end{exercise}
\end{frame}
%
%\begin{frame}{\fn}
%	\begin{exercise}
%		Sketch a triangle whose sides are labeled $a,b,c$ and whose angles opposite each side are labeled $A,B,C$ respectively.
%		
%		Heading back to trigonometry, we get
%		
%			\[\begin{array}{rcrcrcr}
%				&&c\cos B &+& b\cos C &=& a\\
%				c\cos A&&&+&a\cos C &=&b\\
%				b\cos A&+&a\cos B&&&=&c
%			\end{array}\]
%		
%		Use Cramer's Rule to solve for $\cos C$, and use the result to verify the Law of Cosines,
%			\[c^2=a^2+b^2-2ab\cos C.\]
%	\end{exercise}
%\end{frame}

\begin{frame}{\fn}
	\begin{exercise}
		The table shows annual personal health care expenditure (in billions of dollars) in the United States from 2007 through 2009.  
	
		\begin{center}
			\begin{tabular}{l|ccc}
				\hline
				Year, $t$ &2007&2008&2009\\
				Amount, $y$ & 1465&1547&1623\\
				\hline
			\end{tabular}
		\end{center}
	\begin{enumerate}[label=(\alph*)]
		\item Create a system of linear equations for the data to fit the curve $y=at^2+bt+c$ where $t=7$ corresponds to 2007, and $y$ is the amount of expenditure.
		\item Use Cramer's Rule to solve the system.
		\item Graph the curve and the data.
		\item Briefly describe how well the polynomial fits the data.
	\end{enumerate}
	\end{exercise}
\end{frame}
\end{document}

