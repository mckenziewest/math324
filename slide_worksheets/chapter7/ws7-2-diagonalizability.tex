\documentclass{beamer}
%\usepackage[margin=1in]{geometry}
\usepackage{amsthm,amsmath,amsfonts,hyperref,graphicx,color,multicol}
\usepackage{enumitem,tikz}
\usepackage{booktabs}
%%%%%%%%%%
%Beamer Template Customization
%%%%%%%%%%
\setbeamertemplate{navigation symbols}{}
\setbeamertemplate{theorems}[ams style]
\setbeamertemplate{blocks}[rounded]

\definecolor{Blu}{RGB}{43,62,133} % UWEC Blue
\setbeamercolor{structure}{fg=Blu} % Titles

%Unnumbered footnotes:
\newcommand{\blfootnote}[1]{%
	\begingroup
	\renewcommand\thefootnote{}\footnote{#1}%
	\addtocounter{footnote}{-1}%
	\endgroup
}


%%%%%%%%%%
%Custom Commands
%%%%%%%%%%
\newcommand{\R}{\mathbb{R}}
\newcommand{\veca}{\vec{a}}
\newcommand{\vecb}{\vec{b}}
\newcommand{\vece}{\vec{e}}
\newcommand{\vecu}{\vec{u}}
\newcommand{\vecv}{\vec{v}}
\newcommand{\vecw}{\vec{w}}
\newcommand{\vecx}{\vec{x}}
\newcommand{\zerovector}{\vec{0}}

\newcommand{\ds}{\displaystyle}

\newcommand{\fn}{\insertframenumber}

\newcommand{\col}{\operatorname{col}}
\newcommand{\row}{\operatorname{row}}
\newcommand{\rank}{\operatorname{rank}}
\newcommand{\nullity}{\operatorname{nullity}}
\newcommand{\adj}{\operatorname{adj}}
\newcommand{\proj}{\operatorname{proj}}
\newcommand{\ip}[2]{\left\langle #1,#2\right\rangle}

\newcommand{\blank}[1]{\underline{\hspace*{#1}}}

\newcommand{\dotp}{\,{\boldsymbol{\cdot}\hspace*{.01in}}}

%%%%%%%%%%
%Custom Theorem Environments
%%%%%%%%%%
\theoremstyle{definition}
\newtheorem{exercise}{Exercise}
\newtheorem{question}[exercise]{Question}
\newtheorem*{defn}{Definition}
\newtheorem*{exa}{Example}
\newtheorem*{disc}{Group Discussion}
\newtheorem*{nb}{Note}
\newtheorem*{recall}{Recall}
\renewcommand{\emph}[1]{{\color{blue}\texttt{#1}}}

\definecolor{Gold}{RGB}{237, 172, 26}
%Statement block
\newenvironment{statementblock}[1]{%
	\setbeamercolor{block body}{bg=Gold!20}
	\setbeamercolor{block title}{bg=Gold}
	\begin{block}{\textbf{#1.}}}{\end{block}}





\begin{document}
	\title{Math 324: Linear Algebra}
	\subtitle{Section 7.1: Eigenvalues and Eigenvectors\\ Section 7.2: Diagonalizability}
	\author{Mckenzie West}
	\date{Last Updated: \today}
\begin{frame}[fragile]
\maketitle	
\end{frame}

\begin{frame}{\insertframenumber}
	\begin{block}{\textbf{Last Time.}}
	\begin{itemize}[label=--]
		\item Eigenvalues and Eigenvectors
		\item Characteristic Polynomials and Equations
	\end{itemize}
	\end{block}
	\begin{block}{\textbf{Today.}}
		\begin{itemize}[label=--]
			\item Eigenvalues of Triangular Matrices
			\item Eigenvalues of Linear Transformations
		\end{itemize}
	\end{block}
\end{frame}
\begin{frame}{\fn}
	\begin{recall}
		\begin{statementblock}{Theorem 7.2}
		Let $A$ be an $n\times n$ matrix.
			\begin{enumerate}[label=\textbf{\arabic*.}]
				\item An eigenvalue of $A$ is a scalar $\lambda$ such that $\det(\lambda I-A)=0$.
				\item The eigenvectors of $A$ corresponding to $\lambda$ are the nonzero solutions of $(\lambda I-A)\vec x=\vec0$.
				\item The eigenspace of $A$ corresponding to $\lambda$ is the nullspace of the matrix $\lambda I-A$.
			\end{enumerate}
	\end{statementblock}
	\end{recall}
\begin{question}
What is the maximum number of eigenvalues that an $n\times n$ matrix can have?
\end{question}
\end{frame}
\begin{frame}{\fn}
	\begin{statementblock}{Theorem 3.2}
		If $A$ is a triangular matrix of order $n$, then its determinant is the product of the entries on the main diagonal.  That is,	
			\[\det(A)=a_{11}a_{22}a_{33}\cdots a_{nn}.\]
	\end{statementblock}
	\begin{nb}
		We saw this back in Chapter 3.  Triangular matrices have easy determinants.
		
		Additionally, if $A$ is triangular, so is $\lambda I-A$.
	\end{nb}
	\begin{question}
		If $A$ is triangular with diagonal entries $a_{11},a_{22},\dots,a_{nn}$, what are the diagonal entries of $\lambda I-A$?
	\end{question}
\end{frame}
\begin{frame}{\fn}
	\begin{statementblock}{Theorem 7.3}
		If $A$ is an $n\times n$ triangular matrix, then its eigenvalues are the entries on its main diagonal.
	\end{statementblock}
	\begin{exercise}
		Find the eigenvalues and eigenspaces of
		\[A=\begin{bmatrix}
		1&2&0&1\\
		0&0&3&1\\
		0&0&1&0\\
		0&0&0&1
		\end{bmatrix}\]
	\end{exercise}
\end{frame}
\begin{frame}{\fn}
	\begin{exercise}
		Find the eigenvalues and eigenspaces of
			\[A=\begin{bmatrix}
				-1&0&0&0&0\\
				0&2&0&0&0\\
				0&0&2&0&0\\
				0&0&0&10&0\\
				0&0&0&0&5
			\end{bmatrix}\]
	\end{exercise}
\end{frame}
\begin{frame}{\fn}
\begin{block}{\textbf{Brain Break.}}
	If you won the lottery but only had one day to spend the money, what would you buy?
	\begin{center}
		\includegraphics[width=2in]{../images/sleepy_pep}
	\end{center}
	Pepper says ``blankets!''
\end{block}
\end{frame}
\begin{frame}{\fn}
\begin{defn}
	A square matrix $A$ is \emph{diagonalizable} if there is a diagonal matrix $D$ and and invertible matrix $P$ such that $D=P^{-1}AP$.
\end{defn}
\begin{exercise}\label{first}
Let $A=\begin{bmatrix}-4&-15\\2&7\end{bmatrix}$ and $P=\begin{bmatrix}3&-5\\-1&2\end{bmatrix}$.

Verify that $P^{-1}AP$ is diagonal, and confirm that $A$ is diagonalizable.
\end{exercise}
\begin{exercise}
	Is $\begin{bmatrix}
	3&0\\0&2
	\end{bmatrix}$ diagonalizable?
\end{exercise}
\end{frame}
\begin{frame}{\fn}
\begin{statementblock}{Theorem 7.5}
	An $n\times n$ matrix $A$ is diagonalizable if and only if it has $n$ linearly independent eigenvectors.
	
	Moreover if $P=\begin{bmatrix}\vec v_1&\vec v_2&\cdots&\vec v_n\end{bmatrix}$ is a matrix whose columns are these $n$ linearly independent eigenvectors, then $P^{-1}AP$ is diagonal.
\end{statementblock}
\begin{exercise}
The matrix $A=\begin{bmatrix}0&2\\5&-3\end{bmatrix}$ has eigenvalues $\lambda_1=2$ and $\lambda_2=-5$.  There are eigenvectors $\vec v_1=\begin{bmatrix}1\\1\end{bmatrix}$ and $\vec v_2=\begin{bmatrix}2\\-5\end{bmatrix}$.
\begin{enumerate}[label=(\alph*)]
	\item Let $P=\begin{bmatrix}
	\vec v_1 & \vec v_2
	\end{bmatrix}=\begin{bmatrix}
	1&2\\1&-5
	\end{bmatrix}$ and compute $P^{-1}AP$.
	\item Let $Q=\begin{bmatrix}
	\vec v_2 & \vec v_1
	\end{bmatrix}=\begin{bmatrix}
	2&1\\-5&1
	\end{bmatrix}$ and compute $Q^{-1}AQ$.
\end{enumerate}
\end{exercise}
\end{frame}
\begin{frame}[fragile]
\frametitle{\fn}
\vskip -.25in
\begin{exercise}
	Determine whether the matrix is diagonalizable. (Find all eigenspaces and add up their dimensions.)
	\begin{multicols}{2}
		\begin{enumerate}[label=(\alph*)]
			\item $A=\begin{bmatrix}
			4 & 0 & 1 \\
			2 & -2 & 0 \\
			0 & 0 & 5
			\end{bmatrix}$
			\item $A = \begin{bmatrix}1&1&1\\0&1&1\\0&0&1\end{bmatrix}$
		\end{enumerate}
	\end{multicols}
\end{exercise}
\begin{nb}
	It may be helpful to use Sage for this:
	\begin{verbatim}
	A = matrix([[1,0],[3,4]])
	print(A.eigenvalues())
	print(A.right_eigenspaces())
	\end{verbatim}
	Notice that the eigenspaces command indicates a `right' eigenspace, as in we're multiplying on the right.  
	
	And the output includes each of the eigenvalues followed by their corresponding eigenspaces with basis.
\end{nb}
\end{frame}
%\begin{frame}{\fn}
%	\begin{defn}
%		Let $T:V\to W$ be a  linear transformation.
%		A number $\lambda$ is an \emph{eigenvalue} of $T$ if there is some nonzero vector $\vec v\in V$ such that $T(\vec v)=\lambda \vec v$.  In this case, we call $\vec v$ an \emph{eigenvector} of $T$ corresponding to $\lambda$.  The collection of all vectors satisfying $T(\vec v)=\lambda\vec v$ is called the \emph{eigenspace} of $T$ corresponding to $\lambda$.
%	\end{defn}
%	\begin{exercise}
%		Consider the transformation $T:P_2\to P_2$ defined by $T(ax^2+bx+c)=(3a+2b+c)x^2+(c-b)x+(2b)$.  
%		\begin{enumerate}[label=(\alph*)]
%			\item Show that $x^2-5x+5$ is an eigenvector of $T$, and find the corresponding eigenvalue.
%			\item Show that $x^2$ is also an eigenvector of $T$, and find the corresponding eigenvalue.
%		\end{enumerate}
%	\end{exercise}
%\end{frame}
%\begin{frame}{\fn}
%	\begin{exercise}
%		Let's try to find all eigenvalues and eigenvectors of the $T$ on the previous slide.
%		\begin{enumerate}[label=(\alph*)]
%			\item If $\lambda$ is an eigenvalue, then $T(ax^2+bx+c)=\lambda(ax^2+bx+c)$ for some lambda.  That is,
%				\[(3a+2b+c)x^2+(c-b)x+(2b)=\lambda ax^2+\lambda b x+\lambda c.\]
%			\item Use the equation above to create 3 equations by extracting the coefficients.\pause
%				\only<1-2>{\begin{eqnarray}
%				3a+2b+c&=&\lambda a\\
%				c-b&=&\lambda b\\
%				2b &=&\lambda c
%				\end{eqnarray}}
%			\item Use equations (2) and (3) to determine that either $c=0$ or $\lambda ^2+\lambda-2=0$.\pause
%			\item First solve the quadratic equation to find that $\lambda = 1$ and $\lambda =-2$ are two of the eigenvalues of $T$. (Check they work with equation (1).)
%			\item Next, plug $c=0$ into the equations to hopefully find a possible $\lambda$.
%		\end{enumerate}
%	\end{exercise}
%\end{frame}
%\begin{frame}{\fn}
%	\begin{exercise}
%		Finding eigenvalues (Option 2).  Again, use $T:P_2\to P_2$ defined by
%			\[T(ax^2+bx+c)=(3a+2b+c)x^2+(c-b)x+(2b).\]
%		\begin{enumerate}[label=(\alph*)]
%			\item Associate $ax^2+bx+c$ with the vector $(a,b,c)\in \R^3$. And rewrite $T$ as a transformation $T:\R^3\to\R^3$.
%			\item Find the standard matrix for $T$.
%			\item Find the eigenvalues and corresponding eigenspaces for the standard matrix of $T$.
%		\end{enumerate}
%	\end{exercise}
%\end{frame}
%\begin{frame}{\fn}
%	\begin{exercise}
%		Find the eigenvalues (no need to find eigenspaces) for $T:M_{2,2}\to M_{2,2}$ defined by
%			\[T\left(\begin{bmatrix}a&b\\c&d\end{bmatrix}\right)=\begin{bmatrix}
%			a&2a+2b\\b+3c&2a-d
%			\end{bmatrix}\]
%	\end{exercise}%[3, 2, 1, -1]
%\end{frame}
\end{document}

