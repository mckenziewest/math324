\documentclass{beamer}
%\usepackage[margin=1in]{geometry}
\usepackage{amsthm,amsmath,amsfonts,hyperref,graphicx,color,multicol}
\usepackage{enumitem,tikz}

%%%%%%%%%%
%Beamer Template Customization
%%%%%%%%%%
\setbeamertemplate{navigation symbols}{}
\setbeamertemplate{theorems}[ams style]
\setbeamertemplate{blocks}[rounded]

\definecolor{Blu}{RGB}{43,62,133} % UWEC Blue
\setbeamercolor{structure}{fg=Blu} % Titles

%Unnumbered footnotes:
\newcommand{\blfootnote}[1]{%
	\begingroup
	\renewcommand\thefootnote{}\footnote{#1}%
	\addtocounter{footnote}{-1}%
	\endgroup
}


%%%%%%%%%%
%Custom Commands
%%%%%%%%%%
\newcommand{\R}{\mathbb{R}}
\newcommand{\veca}{\vec{a}}
\newcommand{\vecb}{\vec{b}}
\newcommand{\vece}{\vec{e}}
\newcommand{\vecu}{\vec{u}}
\newcommand{\vecv}{\vec{v}}
\newcommand{\vecw}{\vec{w}}
\newcommand{\vecx}{\vec{x}}
\newcommand{\zerovector}{\vec{0}}

\newcommand{\ds}{\displaystyle}

\newcommand{\fn}{\insertframenumber}

\newcommand{\col}{\operatorname{col}}
\newcommand{\row}{\operatorname{row}}
\newcommand{\rank}{\operatorname{rank}}
\newcommand{\nullity}{\operatorname{nullity}}
\newcommand{\adj}{\operatorname{adj}}

\newcommand{\blank}[1]{\underline{\hspace*{#1}}}


%%%%%%%%%%
%Custom Theorem Environments
%%%%%%%%%%
\theoremstyle{definition}
\newtheorem{exercise}{Exercise}
\newtheorem{question}[exercise]{Question}
\newtheorem*{defn}{Definition}
\newtheorem*{exa}{Example}
\newtheorem*{disc}{Group Discussion}
\newtheorem*{nb}{Note}
\newtheorem*{recall}{Recall}
\renewcommand{\emph}[1]{{\color{blue}\texttt{#1}}}

\definecolor{Gold}{RGB}{237, 172, 26}
%Statement block
\newenvironment{statementblock}[1]{%
	\setbeamercolor{block body}{bg=Gold!20}
	\setbeamercolor{block title}{bg=Gold}
	\begin{block}{\textbf{#1.}}}{\end{block}}





\begin{document}
	\title{Math 324: Linear Algebra}
	\subtitle{Section 5.1: Length and Dot Product in $\R^n$}
	\author{Mckenzie West}
	\date{Last Updated: \today}
\begin{frame}
\maketitle
\end{frame}

\begin{frame}{\insertframenumber}
	\begin{block}{\textbf{Last Time.}}
	\begin{itemize}[label=--]
		\item Nullspace of a matrix
		\item Dimension of solution spaces
		\item Solutions of systems of equations
	\end{itemize}
	\end{block}
	\begin{block}{\textbf{Today.}}
		\begin{itemize}[label=--]
			\item Dot Product
			\item Length
			\item Cauchy-Schwarz Inequality
			\item Orthogonality
			\item Triangle Inequality
			\item Pythagorean Theorem
		\end{itemize}
	\end{block}
\end{frame}
\begin{frame}{\fn}
	\begin{defn}
		Let $\vec u=(u_1,u_2,\dots,u_n)$ and $\vec v=(v_1,v_2,\dots,v_n)$ be two vectors in $\R^n$.  Define the \emph{dot product} of $\vec u$ and $\vec v$ to be the \textbf{scalar}
			\[\vec u\cdot\vec v=u_1v_1+u_2v_2+\cdots+u_nv_n.\]
	\end{defn}
	\begin{exercise}[Warmup]
		Let $\vec u=(1,2)$ and $\vec v=(3,-1)$, compute $\vec u\cdot \vec v$.  
		
		What is $(2\vecu)\cdot\vec v$?
	\end{exercise}
\end{frame}
\begin{frame}{\fn}
	\begin{statementblock}{Theorem 5.3}
		If $\vec u$, $\vec v$, and $\vec w$ are vectors in $\R^n$ and $c$ is a scalar, then the following are true:
			\begin{enumerate}[label=\textbf{\arabic*.}]
				\item $\vec u\cdot\vec v=\vec v\cdot \vec u$
				\item $\vec u\cdot(\vec v+\vec w)=\vec u\cdot \vec v+\vec u\cdot\vec w$
				\item $c(\vec u\cdot \vec v)=(c\vec u)\cdot \vec v=\vec u\cdot(c\vec v)$
			\end{enumerate}
	\end{statementblock}
	\begin{exercise}
		Prove property \textbf{1.} of Theorem 5.3 using $\vec u=(u_1,u_2,\dots,u_n)$ and $\vec v=(v_1,v_2,\dots,v_n)$.  Make sure to be precise about what $\vec u\cdot \vec v$ and $\vec v\cdot\vec u$ are.
	\end{exercise}
\end{frame}
\begin{frame}{\fn}
	\begin{defn}
		The \emph{length} of a vector $\vec v =(v_1,v_2,\dots,v_n)$ in $\R^n$ is the scalar
			\[\|\vec v\|=\sqrt{v_1^2+v_2^2+\cdots+v_n^2}.\]
	\end{defn}
	\begin{statementblock}{Theorem 5.3}
		If $\vec v$ is a vector in $\R^n$, then the following are true:
		\begin{enumerate}[label=.]
			\item[\textbf{4.}] $\vec v\cdot\vec v=\|\vec v\|^2$
			\item[\textbf{5.}] $\vec v\cdot \vec v\geq 0$ and $\vec v\cdot\vec v=0$ if and only if $\vec v=\vec 0$
		\end{enumerate}
	\end{statementblock}
\end{frame}
\begin{frame}{\fn}
	\begin{statementblock}{\textbf{Theorem 5.1}}
		Let $\vec v$ be a vector in $\R^n$ and let $c$ be a scalar.  Then
			\[\|c\vec v\|=|c|\|\vec v\|,\]
		where $|c|$ is the absolute value of the scalar $c$.
	\end{statementblock}
	\begin{proof}[Alternate Proof Using Dot Products]
		Let $\vec v$ be a vector in $\R^n$ and $c$ a scalar.  Then
			\begin{eqnarray*}
				\|c\vec v\|&=&\sqrt{(c\vec v)\cdot(c\vec v)}\\
				&=&\sqrt{c^2(\vec v\cdot\vec v)}\\
				&=&|c|\sqrt{\vec v\cdot\vec v}\\
				&=&|c|\|\vec v\|.
			\end{eqnarray*}
		Therefore, $\|c\vec v\|=|c|\|\vec v\|$, as desired.
	\end{proof}
\end{frame}
\begin{frame}{\fn}
	\begin{exercise}
		Let $\vec u=(1,0,3)$ and $\vec v=(-2,-1,2)$.
		
		Compute each of the following:
		\begin{enumerate}[label=(\alph*)]
			\item $\vec u\cdot\vec v$
			\item $\|\vec u\|$
			\item $\|\vec u-\vec v\|$
			\item\label{unitvector} $\vec w=\frac{\vec u}{\|\vec u\|}$ and $\|\vec w\|$
			\item $\vec w\cdot\vec u$ (same $\vec w$ as part \ref{unitvector})
		\end{enumerate}
	\end{exercise}
\end{frame}
\begin{frame}{\fn}
	\begin{defn}
		A vector $\vec v$ in $\R^n$ is called a \emph{unit vector} if $\|\vec v\|=1$.
	\end{defn}
	\begin{statementblock}{Theorem 5.2}
		If $\vec v$ is a nonzero vector in $\R^n$, then the vector
			\[\vec u=\frac{\vec v}{\|\vec v\|}\]
		has length 1 and has the same direction as $\vec v$.  The vector $\vec u$ is called the \emph{unit vector in the direction of $\vec v$}.
	\end{statementblock}
\end{frame}
\begin{frame}{\fn}
	\begin{exercise}
		Let $\vec u=(1,2)$ and $\vec v=(3,1)$.  What is the distance between the endpoints of these vectors?
		
		Repeat the process for generic $\vec u=(x_1,y_1)$ and $\vec v=(x_2,y_2)$.
	\end{exercise}
\end{frame}
\begin{frame}{\fn}
	\begin{defn}
		The \emph{distance} between two vectors $\vec u$ and $\vec v$ in $\R^n$ is defined to be
			\[d(\vec u,\vec v)=\|\vec u-\vec v\|.\]
	\end{defn}
	\begin{exercise}
	Let $\vec u=(1,3,-2)$ and $\vec v=(-1,0,-1)$.  Compute $d(\vec u,\vec v)$.
	\end{exercise}
\end{frame}
\begin{frame}{\fn}
	\begin{exercise}
		Verify that for all vectors $\vec u$ and $\vec v$ in $\R^n$, 
		\begin{enumerate}[label=\textbf{\arabic*}]
			\item $d(\vec u,\vec v)\geq 0$
			\item $d(\vec u,\vec v)=0$ if and only if $\vec u=\vec v$
			\item $d(\vec u,\vec v)=d(\vec v,\vec u)$.
		\end{enumerate}
		Explain they this makes sense.
	\end{exercise}
\end{frame}
\begin{frame}{\fn}
	\begin{exercise}
		Use the dot product version of length to verify that 
			\[\|\vec v-\vec u\|^2=\|\vec u\|^2+\|\vec v\|^2-2(\vec u\cdot\vec v).\]
	\end{exercise}
	\begin{defn}
		The \emph{angle between the vectors $\vec u$ and $\vec v$} is the angle $0\leq\theta\leq \pi$ satisfying 
			\[\cos(\theta)=\frac{\vec u\cdot\vec v}{\|\vec u\|\|\vec v\|}.\]
	\end{defn}
	\begin{exercise}
		Find the angle between $\vec u=(3,-1,0,2)$ and $\vec v=(-6,2,0,-4)$.
	\end{exercise}
\end{frame}
\begin{frame}{\fn}
	Warning, for our definition of the angle between the curves to work, we need to know that 
		\[\left|\frac{\vec u\cdot\vec v}{\|\vec u\|\|\vec v\|}\right|=\frac{|\vec u\cdot\vec v|}{\|\vec u\|\|\vec v\|}\leq 1.\]
	\begin{statementblock}{Theorem 5.4: The Cauchy--Schwarz Inequality}
		If $\vec u$ and $\vec v$ are vectors in $\R^n$, then 
			\[|\vec u\cdot\vec v|\leq\|\vec u\|\|\vec v\|.\]
	\end{statementblock}
	\begin{nb}
		The proof of this is a neat application of the quadratic formula and is in the book.
	\end{nb}
	\begin{exercise}
		Verify the Cauchy--Schwarz inequality using $\vec u=(1,1,3)$ and $\vec v=(3,-2,1)$.
	\end{exercise}
\end{frame}
\begin{frame}{\fn}
	\begin{block}{\textbf{Brain Break.}}
		What is your favorite word?
		
		\begin{center}
			\includegraphics[width=2in]{images/word}
		\end{center}
	\end{block}
\end{frame}
\begin{frame}{\fn}
	\begin{defn}
		Two vectors $\vec u$ and $\vec v$ in $\R^n$ are \emph{orthogonal} if $\vec u\cdot\vec v=0$.
	\end{defn}
	\begin{exercise}
		Verify that $\vec u=(1,0,0)$ and $\vec v=(0,2,3)$ are orthogonal.
	\end{exercise}
	\begin{exercise}
		Verify that $\vec u=(1,2,3,4)$ and $\vec v=(1,0,1,-1)$ are orthogonal.
	\end{exercise}
\end{frame}
%\begin{frame}{\fn}
%	\begin{exercise}
%		Consider the vector $\vec u=(3,-2)$.  Find all vectors orthogonal to $\vec u$:
%		\begin{enumerate}[label=(\alph*)]
%			\item Let $\vec v=(v_1,v_2)$ be a vector orthogonal to $\vec u$.  This means that
%				\[\vec u\cdot\vec v=\underline{\hspace*{1in}}=0.\]
%			\item Thus $3\vec v_1=2\vec v_2$, so every vector orthogonal to $\vec u$ is of the form:
%				\[\vec v=(t,\underline{\hspace*{.5in}})=t(1,\underline{\hspace*{.5in}}).\]
%		\end{enumerate}
%	\end{exercise}
%\end{frame}
\begin{frame}{\fn}
	\begin{statementblock}{Theorem 5.5: The Triangle Inequality}
		If $\vec u$ and $\vec v$ are vectors in $\R^n$, then
			\[\|\vec u+\vec v\|\leq\|\vec u\|+\|\vec v\|.\]
	\end{statementblock}
	\begin{exercise}
		Let $\vec u=(-2,3)$ and $\vec v=(1,1)$.  Draw a picture of these vectors as well as $\vec u+\vec v$.
		
		Explain why the Triangle Inequality is true in this case. 
		
		Geometrically speaking, what does the Triangle Inequality say?
	\end{exercise}
\end{frame}
\begin{frame}{\fn}
	\begin{statementblock}{Theorem 5.6: The Pythagorean Theorem}
		If $\vec u$ and $\vec v$ are vectors in $\R^n$, then $\vec u$ and $\vec v$ are orthogonal if and only if
			\[\|\vec u+\vec v\|^2=\|\vec u\|^2+\|\vec v\|^2.\]
	\end{statementblock}
\end{frame}
\begin{frame}{\fn}
	If we consider each vector $\vec u$ and $\vec v$ in $\R^n$ as $n\times 1$ column vectors, then
	
		\[\vec u\cdot\vec v=\vec u^T\vec v=\begin{bmatrix}u_1&u_2&\cdots&u_n\end{bmatrix}\begin{bmatrix}v_1\\v_2\\\vdots\\v_n\end{bmatrix}=\begin{bmatrix}u_1v_1+u_2v_2+\cdots+u_nv_n\end{bmatrix}.\]
	\begin{exercise}
		Use this method to compute the dot product of \[\vec u=\begin{bmatrix}3\\-5\end{bmatrix}\text{ and }\vec v=\begin{bmatrix}6\\1\end{bmatrix}.\]
	\end{exercise}
\end{frame}
\end{document}

