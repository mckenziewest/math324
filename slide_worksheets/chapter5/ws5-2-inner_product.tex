\documentclass{beamer}
%\usepackage[margin=1in]{geometry}
\usepackage{amsthm,amsmath,amsfonts,hyperref,graphicx,color,multicol}
\usepackage{enumitem,tikz}

%%%%%%%%%%
%Beamer Template Customization
%%%%%%%%%%
\setbeamertemplate{navigation symbols}{}
\setbeamertemplate{theorems}[ams style]
\setbeamertemplate{blocks}[rounded]

\definecolor{Blu}{RGB}{43,62,133} % UWEC Blue
\setbeamercolor{structure}{fg=Blu} % Titles

%Unnumbered footnotes:
\newcommand{\blfootnote}[1]{%
	\begingroup
	\renewcommand\thefootnote{}\footnote{#1}%
	\addtocounter{footnote}{-1}%
	\endgroup
}


%%%%%%%%%%
%Custom Commands
%%%%%%%%%%
\newcommand{\R}{\mathbb{R}}
\newcommand{\veca}{\vec{a}}
\newcommand{\vecb}{\vec{b}}
\newcommand{\vece}{\vec{e}}
\newcommand{\vecu}{\vec{u}}
\newcommand{\vecv}{\vec{v}}
\newcommand{\vecw}{\vec{w}}
\newcommand{\vecx}{\vec{x}}
\newcommand{\zerovector}{\vec{0}}

\newcommand{\ds}{\displaystyle}

\newcommand{\fn}{\insertframenumber}

\newcommand{\col}{\operatorname{col}}
\newcommand{\row}{\operatorname{row}}
\newcommand{\rank}{\operatorname{rank}}
\newcommand{\nullity}{\operatorname{nullity}}
\newcommand{\adj}{\operatorname{adj}}
\newcommand{\proj}{\operatorname{proj}}
\newcommand{\ip}[2]{\left\langle #1,#2\right\rangle}

\newcommand{\blank}[1]{\underline{\hspace*{#1}}}

\newcommand{\dotp}{\,{\boldsymbol{\cdot}\hspace*{.01in}}}

%%%%%%%%%%
%Custom Theorem Environments
%%%%%%%%%%
\theoremstyle{definition}
\newtheorem{exercise}{Exercise}
\newtheorem{question}[exercise]{Question}
\newtheorem*{defn}{Definition}
\newtheorem*{exa}{Example}
\newtheorem*{disc}{Group Discussion}
\newtheorem*{nb}{Note}
\newtheorem*{recall}{Recall}
\renewcommand{\emph}[1]{{\color{blue}\texttt{#1}}}

\definecolor{Gold}{RGB}{237, 172, 26}
%Statement block
\newenvironment{statementblock}[1]{%
	\setbeamercolor{block body}{bg=Gold!20}
	\setbeamercolor{block title}{bg=Gold}
	\begin{block}{\textbf{#1.}}}{\end{block}}





\begin{document}
	\title{Math 324: Linear Algebra}
	\subtitle{Section 5.2: Inner Product Spaces}
	\author{Mckenzie West}
	\date{Last Updated: \today}
\begin{frame}
\maketitle
\end{frame}

\begin{frame}{\insertframenumber}
	\begin{block}{\textbf{Last Time.}}
	\begin{itemize}[label=--]
		\item Dot Product
		\item Length
		\item Cauchy-Schwarz Inequality
		\item Triangle Inequality
	\end{itemize}
	\end{block}
	\begin{block}{\textbf{Today.}}
		\begin{itemize}[label=--]
			\item Orthogonal Complement
			\item Inner Products
		\end{itemize}
	\end{block}
\end{frame}
\begin{frame}{\fn}
	You  may have seen the following definition last time.
	\begin{defn}
		Two vectors $\vec u$ and $\vec v$ in $\R^n$ are said to be \emph{orthogonal} if $\vec u\dotp\vec v=0$.
	\end{defn}
	\begin{recall}
		The angle between $\vec u$ and $\vec v$ satisfies $$\cos(\theta)=\frac{\vec u\dotp \vec v}{\|\vec u\|\|\vec v\|}.$$
		
		Thus orthogonal vectors have angle $\theta$, with $\cos(\theta)=0$.  Or $\theta=\frac{\pi}{2}$, and are thus perpendicular.
	\end{recall}
\end{frame}
\begin{frame}[fragile]
	\frametitle{\fn}
	\begin{exercise}
		Find all vectors $(x,y,z)$ that are orthogonal to $(-2,1,0)$.
		\begin{enumerate}[label=(\alph*)]
			\item Compute $(x,y,z)\dotp (-2,1,0)$.
			\item Set the dot product equal to zero and parameterize.\pause\\
				The equation $-2x+y=0$ is equivalent to $y=2x$, so if $x$ and $z$ are given parameters, the orthogonal complement of $(-2,1,0)$ is the set \begin{equation}\label{orthcomp}
					\{(t,2t,s)\ :\ s,t\in\R\}.
				\end{equation}
			\item Verify that no matter the values of $s$ and $t$, $(t,2t,s)\dotp(-2,1,0)=0$.\pause
			\item The set in \eqref{orthcomp} is called the \emph{orthogonal complement} of $(-2,1,0)$.  Plot $(-2,1,0)$ and the orthogonal complement in Sage (\url{https://sagecell.sagemath.org/}) using:
			{\tiny
				\begin{verbatim}
					s,t = var("s,t")
					P = parametric_plot3d((t,2*t,s), (t,-2,2),(s,-2,2))
					P += arrow((0,0,0), (-2,1,0), width=3, color="red")
					P.show(aspect_ratio=1)
					\end{verbatim}
			}
		\end{enumerate}
	\end{exercise}
\end{frame}
\begin{frame}{\fn}
	\begin{recall}
		The four main properties of the dot product are:
		\begin{enumerate}[label=\textbf{\arabic*.}]
			\item $\vec u\dotp\vec v=\vec v\dotp\vec u$.
			\item $\vec u\dotp(\vec v+\vec w)=\vec u\dotp\vec v+\vec u\dotp \vec w$.
			\item $c(\vec u\dotp \vec v)=(c\vec u)\dotp\vec v$.
			\item $\vec v\cdot\vec v\geq 0$ and $\vec v\cdot\vec v=0$ if and only if $\vec v=\vec 0$.
		\end{enumerate}
	\end{recall}
\end{frame}
\begin{frame}{\fn}
	\begin{defn}
		Let $\vec u,\vec v$ and $\vec w$ be vectors in a vector space $V$, and let $c$ be any scalar.  An \emph{inner product} on $V$ is a function that associates a real number $\ip{\vecu}{\vec v}$ for each pair of vectors $\vec u$ and $\vec v$ and satisfies the following axioms.
		\begin{enumerate}[label=\textbf{\arabic*.}]
			\item $\ip{\vec u}{\vec v}=\ip{\vec v}{\vec u}$
			\item $\ip{\vec u}{\vec v+\vec w}=\ip{\vec u}{\vec v}+\ip{\vec u}{\vec w}$
			\item $c\ip{\vec u}{\vec v}=\ip{c\vec u}{\vec v}$
			\item $\ip{\vec v}{\vec v}\geq 0$ and $\ip{\vec v}{\vec v}=0$ if and only if $\vec v=\vec 0$.
		\end{enumerate}
	
		A vector space with an inner product is called an \emph{inner product space}.
	\end{defn}
	\begin{exa}
		The dot product is an inner product in $\R^n$, so $\R^n$ is an inner product space.
	\end{exa}
\end{frame}
\begin{frame}{\fn}
	\begin{exercise}
		Let $\vec u=(u_1,u_2)$ and $\vec v=(v_1,v_2)$ be vectors in $\R^2$.  Verify that
			\[\ip{\vec u}{\vec v}=3u_1v_1+2u_2v_2\]
		is an inner product on $\R^2$.  
		\begin{enumerate}[label=(\alph*)]
			\item Is $\ip{\vec{u}}{\vec{v}}$ a real number?
			\item Does $\ip{\vec u}{\vec v}=\ip{\vec v}{\vec u}$?
			\item Does $\ip{\vec u}{\vec v+\vec w}=\ip{\vec u}{\vec v}+\ip{\vec u}{\vec w}$?
			\item Does $c\ip{\vec u}{\vec v}=\ip{c\vec u}{\vec v}$?
			\item Is it true that $\ip{\vec v}{\vec v}\geq 0$ and $\ip{\vec v}{\vec v}=0$ if and only if $\vec v=\vec 0$?
		\end{enumerate}
	\end{exercise}
\end{frame}
\begin{frame}{\fn}
	\begin{exercise}
		Let $\vec u=(u_1,u_2)$ and $\vec v=(v_1,v_2)$ be vectors in $\R^2$.  Use a specific example to show that
		\[\ip{\vec u}{\vec v}=u_1v_1-u_2v_2\]
		is NOT an inner product on $\R^2$.  
		
		That is, pick out an explicit (numerical values) example of $\vec{u}$ and $\vec{v}$ for which one of the four inner product axioms fails.
	\end{exercise}
\end{frame}
\begin{frame}{\fn}
	\begin{block}{\textbf{Brain Break.}}
		How many people are in your family? 
		
		(Define family how you wish. To me, this means parents and siblings.  Maybe you want to count aunts/uncles/cousins/grandparents/etc. Let me know what you decide.)
		
		\begin{center}
			\includegraphics[width=2in]{../images/emory_grad}
			
			This is my family in town for my graduate school graduation.
		\end{center}
	
		
	\end{block}
\end{frame}
\begin{frame}{\fn}
	\begin{exercise}\label{MatInnerProd}
		Let $A=\begin{bmatrix} a_{11}&a_{12}\\a_{21}&a_{22}\end{bmatrix}$ and $B=\begin{bmatrix}b_{11}&b_{12}\\b_{21}&b_{22}\end{bmatrix}$ be matrices in the vector space $M_{2,2}$.  Show that the function
			\[\ip{A}{B}=a_{11}b_{11}+a_{12}b_{12}+a_{21}b_{21}+a_{22}b_{22}\]
		is an inner product on $M_{2,2}$.
	\end{exercise}
	\begin{nb}
		For a finite dimensional vectors space, inner products essentially always look like dot products with positive scalars.  Infinite dimensional vector spaces are not always so straight forward.
	\end{nb}
\end{frame}
\begin{frame}{\fn}
	\begin{exercise}
		Let $f$ and $g$ be real valued continuous functions in the vector space $C[0,1]$.  I claim that the following is an inner product,\[\ip{f}{g}=\int_0^1 f(x)g(x)\ dx.\]
		
		Compute $\ip{3x+2}{e^x}$. 
		
		(Yes, you can---and should---use your calculator or an online integral calculator).
	\end{exercise}
\end{frame}
\begin{frame}{\fn}
	\begin{defn}
		Let $\vec u$ and $\vec v$ be vectors in an inner product space $V$.  Then similarly to in the case of the dot product, we can make similar definitions for length, distance, angle, orthogonal, and orthogonal projection.
		\begin{enumerate}[label=\textbf{\arabic*.}]
			\item The \emph{length} (or \emph{norm}) of $\vec u$ is $\|\vec u\|=\sqrt{\ip{\vec u}{\vec u}}$.
			\item The \emph{distance} between $\vec u$ and $\vec v$ is $d(\vec u,\vec v)=\|\vec u-\vec v\|$.
			\item The \emph{angle} between $\vec u$ and $\vec v$ is given by
				\[\cos\theta=\frac{\ip{\vec u}{\vec v}}{\|\vec u\|\|\vec v\|},\hspace*{.1in}0\leq\theta\leq \pi.\]
			\item We say $\vec u$ and $\vec v$ are \emph{orthogonal} if $\ip{\vec u}{\vec v}=0$.
		\end{enumerate}
	\end{defn}
\end{frame}
\begin{frame}{\fn}
	\begin{exercise}
		Consider the inner product on $M_{2,2}$ from exercise \ref{MatInnerProd}:
			\[\ip{A}{B}=a_{11}b_{11}+a_{12}b_{12}+a_{21}b_{21}+a_{22}b_{22}.\]
		Let $A=\begin{bmatrix}1&2\\0&-1\end{bmatrix}$ and $B=\begin{bmatrix}0&5\\2&-2\end{bmatrix}$.
		Compute
		\begin{enumerate}[label=(\alph*)]
			\item $\|A\|^2$
			\item $d(A,B)^2$
			\item Cosine of the angle between $A$ and $B$.
			\item Are $A$ and $B$ orthogonal?
		\end{enumerate}
	\end{exercise}
\end{frame}
\begin{frame}{\fn}
	\begin{nb}
		The inequalities also persist for inner products. (Note that in proving the Triangle Inequality for example, we didn't actually use the definition of the dot product.)
	\end{nb}
	\begin{statementblock}{Theorem 5.8}
		Let $\vec u$ and $\vec v$ be vectors in an inner product space $V$.
		\begin{enumerate}[label=\textbf{\arabic*.}]
			\item Cauchy--Schwarz Inequality: $|\ip{\vec u}{\vec v}|\leq \|\vec u\|\|\vec v\|$
			\item Triangle Inequality: $\|\vec u+\vec v\|\leq\|\vec u\|+\|\vec v\|$
			\item Pythagorean Theorem: $\vec u$ and $\vec v$ are orthogonal if and only if 
				\[\|\vec u+\vec v\|^2=\|\vec u\|^2+\|\vec v\|^2.\]
		\end{enumerate}
	\end{statementblock}
\end{frame}
\begin{frame}{\fn}
	\begin{nb}
		One of the great things that we can use this for is to prove these relationships for integrals without actually talking about integrals!
	\end{nb}
	\begin{exercise}
		Let $f$ and $g$ be real valued continuous functions in the vector space $C[0,1]$. Use the inner product,\[\ip{f}{g}=\int_0^1 f(x)g(x)\ dx.\]
		
		Verify the Cauchy-Schwarz Inequality for $f=3x+2$ and $g=e^x$. 
		
		(Yes, you can---and should---use your calculator).
	\end{exercise}
\end{frame}
\end{document}

