\documentclass{beamer}
%\usepackage[margin=1in]{geometry}
\usepackage{amsthm,amsmath,amsfonts,hyperref,graphicx,color,multicol}
\usepackage{enumitem,tikz}
\usepackage{booktabs}
%%%%%%%%%%
%Beamer Template Customization
%%%%%%%%%%
\setbeamertemplate{navigation symbols}{}
\setbeamertemplate{theorems}[ams style]
\setbeamertemplate{blocks}[rounded]

\definecolor{Blu}{RGB}{43,62,133} % UWEC Blue
\setbeamercolor{structure}{fg=Blu} % Titles

%Unnumbered footnotes:
\newcommand{\blfootnote}[1]{%
	\begingroup
	\renewcommand\thefootnote{}\footnote{#1}%
	\addtocounter{footnote}{-1}%
	\endgroup
}


%%%%%%%%%%
%Custom Commands
%%%%%%%%%%
\newcommand{\R}{\mathbb{R}}
\newcommand{\veca}{\vec{a}}
\newcommand{\vecb}{\vec{b}}
\newcommand{\vece}{\vec{e}}
\newcommand{\vecu}{\vec{u}}
\newcommand{\vecv}{\vec{v}}
\newcommand{\vecw}{\vec{w}}
\newcommand{\vecx}{\vec{x}}
\newcommand{\zerovector}{\vec{0}}

\newcommand{\ds}{\displaystyle}

\newcommand{\fn}{\insertframenumber}

\newcommand{\col}{\operatorname{col}}
\newcommand{\row}{\operatorname{row}}
\newcommand{\rank}{\operatorname{rank}}
\newcommand{\nullity}{\operatorname{nullity}}
\newcommand{\adj}{\operatorname{adj}}
\newcommand{\proj}{\operatorname{proj}}
\newcommand{\ip}[2]{\left\langle #1,#2\right\rangle}

\newcommand{\blank}[1]{\underline{\hspace*{#1}}}

\newcommand{\dotp}{\,{\boldsymbol{\cdot}\hspace*{.01in}}}

%%%%%%%%%%
%Custom Theorem Environments
%%%%%%%%%%
\theoremstyle{definition}
\newtheorem{exercise}{Exercise}
\newtheorem{question}[exercise]{Question}
\newtheorem*{defn}{Definition}
\newtheorem*{exa}{Example}
\newtheorem*{disc}{Group Discussion}
\newtheorem*{nb}{Note}
\newtheorem*{recall}{Recall}
\renewcommand{\emph}[1]{{\color{blue}\texttt{#1}}}

\definecolor{Gold}{RGB}{237, 172, 26}
%Statement block
\newenvironment{statementblock}[1]{%
	\setbeamercolor{block body}{bg=Gold!20}
	\setbeamercolor{block title}{bg=Gold}
	\begin{block}{\textbf{#1.}}}{\end{block}}





\begin{document}
	\title{Math 324: Linear Algebra}
	\subtitle{Section 6.2: The Kernel and Range of a Linear Transformation}
	\author{Mckenzie West}
	\date{Last Updated: \today}
\begin{frame}[fragile]
\maketitle	
\end{frame}

\begin{frame}{\insertframenumber}
	\begin{block}{\textbf{Last Time.}}
	\begin{itemize}[label=--]
		\item Matrix Transformations
		\item The Kernel of a Transformation
		\item The Range of a Transformation
		\item Rank and Nullity
	\end{itemize}
	\end{block}
	\begin{block}{\textbf{Today.}}
		\begin{itemize}[label=--]
			\item One-to-One Transformations
			\item Onto Transformations
			\item Isomorphisms
		\end{itemize}
	\end{block}
\end{frame}
\begin{frame}{\fn}
	\begin{recall}
			The \emph{kernel} of a linear transformation $T\colon V\to W$ is the collection of all vectors in $V$ that map to the zero vector in $W$.  We denote this by	\[\ker(T)=\{\vec v\in V:T(\vec v)=\vec 0_W\}\]
			
			The \emph{range} of $T$ is all of the vectors in $W$ that are `hit' by $T$.  This is can be denoted using either of the following
				\begin{eqnarray*}
				\operatorname{range}(T)&=&\{T(\vec v):\vec v\in V\}\\
				&=&\{\vec w\in W:T(\vec v)=\vec w\ \exists\  \vec v\in V\}.
				\end{eqnarray*}
			(Note: Here we read the $\exists$ symbol as `for some'.)
	\end{recall}
\end{frame}
\begin{frame}{\fn}
\begin{recall}
	We also saw last time that if $T$ is a matrix transformation with matrix $A$, then $\ker(T)=N(A)$ and $\operatorname{range}(T)=\col(A)$.
	
	Some natural questions that may arise are whether or not 
	\begin{itemize}[label=\textbullet]
		\item the kernel/nullspace have only the trivial solution;
		\item the range/column space is all of the codomain. (That is, the columns span the space.)
	\end{itemize}
	The first of these ideas we will call \emph{one-to-one} and the second is called \emph{onto}.  They will be precisely defined in this worksheet.
	
\end{recall}
\end{frame}
\begin{frame}{\fn}
	\begin{defn}
		A linear transformation, $T\colon V\to W$, is \emph{one-to-one} if the preimage of every vector in $\operatorname{range}(T)$ has exactly one vector.
		
		That is, for every $\vec u, \vec v\in V$, if $T(\vec u)=T(\vec v)$ then $\vec u=\vec v$.
	\end{defn}
	\begin{exercise}
		Show that the linear transformation $T: \R^2\to \R^3$ defined by $T(x,y)=(x+y,x-y,0)$ is one-to-one.
			\begin{enumerate}[label=(\alph*)]
				\item Let $\vec v_1=(x_1,y_1)$ and $\vec v_2=(x_2,y_2)$ be two vectors in $\R^2$ such that $T(\vec v_1)=T(\vec v_2)$.
				\item Write out what it means for $T(\vec v_1)=T(\vec v_2)$ using the definition of $T$.
				\item Show that $x_1=x_2$ and $y_1=y_2$.  Thus proving $\vec v_1=\vec v_2$.
				\item Conclude that $T$ is one-to-one.
			\end{enumerate}
	\end{exercise}
\end{frame}
\begin{frame}{\fn}
	\begin{statementblock}{Theorem 6.6}
		Let $T:V\to W$ be a linear transformation.  Then $T$ is one-to-one if and only if $\ker(T)=\{\vec 0\}$.
	\end{statementblock}
	\begin{nb}
		Notice how powerful this is.  It says that knowing how many things map to $\vec 0$ immediately tells you how many things map to everything else in the range.
	\end{nb}

\end{frame}
\begin{frame}{\fn}
	\begin{exercise}
		Use Theorem 6.6 to show that the linear transformation $T: \R^2\to \R^3$ defined by $T(x,y)=(x+y,x-y,0)$ is one-to-one.
		\begin{enumerate}[label=(\alph*)]
			\item Let $\vec v=(x,y)$ be a vector in $\R^2$ such that $T(\vec v)=\vec 0$.
			\item Carefully write out what $T(\vec v)$ is and show that in fact $\vec v=\vec 0$ must be true.
		\end{enumerate}
	\end{exercise}
%	\begin{exercise}
%		Determine whether the transformation $T:M_{2,2}\to M_{2,2}$ defined by $T(A)=A-A^T$ is one-to-one.
%	\end{exercise}
	\begin{exercise}
		Determine whether the transformation $T: \R^3\to\R^2$ defined by $A=\begin{bmatrix}1&2&0\\0&1&1\end{bmatrix}$ is one-to-one.
	\end{exercise}
\end{frame}
\begin{frame}{\fn}
	\begin{question}
		If $T:V\to W$ is a one-to-one linear transformation, how does $\dim(V)$ compare to $\dim(W)$?
			\begin{itemize}[label=--]
				\item Can a transformation $T:\R^3\to\R^2$ be one-to-one?
				\item Can a transformation $T:\R^2\to\R^3$ be one-to-one?
				\item Can a transformation $T:\R^3\to\R^3$ be one-to-one? 
			\end{itemize}
	\end{question}
\end{frame}
\begin{frame}{\fn}
	\begin{defn}
		A linear transformation $T:V\to W$ is \emph{onto} if $\operatorname{range}(T)=W$.
	\end{defn}
	\begin{statementblock}{Theorem 6.7}
		Let $T:V\to W$ be a linear transformation where $W$ is finite dimensional.  Then $T$ is onto if and only if $\rank(T)=\dim(W)$.
	\end{statementblock}
	\begin{exercise}
		Discuss this theorem, make sure you agree and understand the difference between $T$ being onto and $\rank(T)=\dim(W)$.
	\end{exercise}
\end{frame}
\begin{frame}{\fn}
	\begin{block}{\textbf{Brain Break.}}
		What would you go viral for on the Internet?
		\begin{center}
			\includegraphics[height=.5\textheight]{../images/internet_famous.jpg}
		\end{center}
	\end{block}
\end{frame}
\begin{frame}{\fn}
	\begin{statementblock}{Theorem 6.8}
		Let $T:V\to W$ be a lienar transformation with vector space $V$ and $W$, both dimension $n$.  Then $T$ is one-to-one if and only if $T$ is onto.
	\end{statementblock}
	\begin{exercise}
		Prove Theorem 6.8,
			\begin{enumerate}[label=(\alph*)]
				\item Assume first that $T$ is one-to-one.
					\begin{enumerate}[label=\roman*.]
						\item What does theorem 6.6 say about $\ker (T)$?
						\item You know $\dim(V)=\dim(W)=n$ and $n=\rank(T)+\nullity(T)$.  Use this and part (i) to conclude that $\rank(T)=n$.
						\item Use Theorem 6.7 to conclude $T$ is onto.
					\end{enumerate}
				\item Now, reset and assume $T$ is onto.  Reverse the arguments from (a) with the starting point of $\rank(T)=n$ to get to $\ker(T)=\{\vec 0\}$.
			\end{enumerate}
	\end{exercise}
\end{frame}
\begin{frame}{\fn}
	\begin{defn}
		A linear transformation $T:V\to W$ that is one-to-one and onto is called an \emph{isomorphism}.  Moreover, if there exists at least one isomorphism between two vector spaces $V$ and $W$, then we call $V$ and $W$ \emph{isomorphic}.
	\end{defn}
	\begin{nb}
		Note that isomorphic means ``same''.  It often does not mean ``equal''.
	\end{nb}
	\begin{exa}
		The vector spaces $P_2$ and $\R^3$ are isomorphic because $T:P_2\to\R^3$ defined by $T(ax^2+bx+c)=(a,b,c)$ is an isomorphism.  (Check this.)
	\end{exa}
\end{frame}
\begin{frame}{\fn}
	\begin{exercise}
		For (a)-(d), consider the linear transformation $T:\R^n\to\R^m$ defined by $T(\vec x)=A\vec x$. 
			\begin{enumerate}[label=\roman*.]
				\item Determine $n$ and $m$.
				\item Is $T$ one-to-one, onto, or neither?
				\item Is $T$ an isomorphism?
			\end{enumerate} 
			\begin{multicols}{2}
				\begin{enumerate}[label=(\alph*)]
					\item $A=\begin{bmatrix}1&0&3\\0&1&2\\0&0&0\end{bmatrix}$
					\item $A=\begin{bmatrix}1&0&3\\0&1&2\end{bmatrix}$
					\item $A=\begin{bmatrix}1&0\\0&1\\3&2\end{bmatrix}$
					\item $A=\begin{bmatrix} 1&0&3\\0&1&2\\0&0&1\end{bmatrix}$
				\end{enumerate}
			\end{multicols}
	\end{exercise}
\end{frame}
\begin{frame}{\fn}
	\begin{statementblock}{Theorem 6.9}
		Two finite-dimensional vector spaces $V$ and $W$ are isomorphic if and only if they are of the same dimension.
	\end{statementblock}
	\begin{nb}
		Holy schmoley is this awesome! Theorem 6.9 says so many things we know, e.g.~$R^2$ is essentially the same as $M_{2,1}$, $M_{1,2}$, $P_1$, a plane in $\R^3$, a plane in $\R^4$,.... the list goes on.
	\end{nb}
	\begin{exercise}
		List at least 4 vector spaces that are isomorphic to $\R^6$.
	\end{exercise}
\end{frame}
\begin{frame}{\fn}
	\begin{question}
		If $T:\R^n\to\R^m$ is a linear transformation defined by the matrix $A$, what do each of the following tell you about the dimensions and the invertibility of $A$?
			\begin{enumerate}[label=(\alph*)]
				\item $T$ is neither one-to-one nor onto
				\item $T$ is one-to-one
				\item $T$ is onto
				\item $T$ is an isomorphism
			\end{enumerate}
	\end{question}
\end{frame}
%\begin{frame}{\fn}
%	\begin{exercise}
%		Let $B$ be a fixed invertible $n\times n$ matrix.  
%		Prove that the linear transformation $T:M_{n,n}\to M_{n,n}$ defined by $T(A)=AB$ is an isomorphism.
%			\begin{enumerate}[label=(\alph*)]
%				\item Show that $\ker(T)=\{O_{n,n}\}$.
%				\item State why this implies $T$ is one-to-one.
%				\item State why it is sufficient to show that $T$ is one-to-one in order to prove that it is an isomorphism.
%			\end{enumerate}
%	\end{exercise}
%	\begin{question}
%		Similarly, Theorem 6.8 also says that it would be sufficient to show that $T$ is onto.
%		It is often easier to show that a linear transformation is one-to-one than to show it is onto.  Why might this be the case?
%	\end{question}
%\end{frame}
\end{document}

