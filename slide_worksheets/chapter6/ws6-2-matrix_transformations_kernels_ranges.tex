\documentclass{beamer}
%\usepackage[margin=1in]{geometry}
\usepackage{amsthm,amsmath,amsfonts,hyperref,graphicx,color,multicol}
\usepackage{enumitem,tikz}
\usepackage{booktabs}
%%%%%%%%%%
%Beamer Template Customization
%%%%%%%%%%
\setbeamertemplate{navigation symbols}{}
\setbeamertemplate{theorems}[ams style]
\setbeamertemplate{blocks}[rounded]

\definecolor{Blu}{RGB}{43,62,133} % UWEC Blue
\setbeamercolor{structure}{fg=Blu} % Titles

%Unnumbered footnotes:
\newcommand{\blfootnote}[1]{%
	\begingroup
	\renewcommand\thefootnote{}\footnote{#1}%
	\addtocounter{footnote}{-1}%
	\endgroup
}


%%%%%%%%%%
%Custom Commands
%%%%%%%%%%
\newcommand{\R}{\mathbb{R}}
\newcommand{\veca}{\vec{a}}
\newcommand{\vecb}{\vec{b}}
\newcommand{\vece}{\vec{e}}
\newcommand{\vecu}{\vec{u}}
\newcommand{\vecv}{\vec{v}}
\newcommand{\vecw}{\vec{w}}
\newcommand{\vecx}{\vec{x}}
\newcommand{\zerovector}{\vec{0}}

\newcommand{\ds}{\displaystyle}

\newcommand{\fn}{\insertframenumber}

\newcommand{\col}{\operatorname{col}}
\newcommand{\row}{\operatorname{row}}
\newcommand{\rank}{\operatorname{rank}}
\newcommand{\nullity}{\operatorname{nullity}}
\newcommand{\adj}{\operatorname{adj}}
\newcommand{\proj}{\operatorname{proj}}
\newcommand{\ip}[2]{\left\langle #1,#2\right\rangle}

\newcommand{\blank}[1]{\underline{\hspace*{#1}}}

\newcommand{\dotp}{\,{\boldsymbol{\cdot}\hspace*{.01in}}}

%%%%%%%%%%
%Custom Theorem Environments
%%%%%%%%%%
\theoremstyle{definition}
\newtheorem{exercise}{Exercise}
\newtheorem{question}[exercise]{Question}
\newtheorem*{defn}{Definition}
\newtheorem*{exa}{Example}
\newtheorem*{disc}{Group Discussion}
\newtheorem*{nb}{Note}
\newtheorem*{recall}{Recall}
\renewcommand{\emph}[1]{{\color{blue}\texttt{#1}}}

\definecolor{Gold}{RGB}{237, 172, 26}
%Statement block
\newenvironment{statementblock}[1]{%
	\setbeamercolor{block body}{bg=Gold!20}
	\setbeamercolor{block title}{bg=Gold}
	\begin{block}{\textbf{#1.}}}{\end{block}}





\begin{document}
	\title{Math 324: Linear Algebra}
	\subtitle{Section 6.1: Linear Transformations\\
	Section 6.2: The Kernel and Range of a Linear Transformation}
	\author{Mckenzie West}
	\date{Last Updated: \today}
\begin{frame}
\maketitle
\end{frame}

\begin{frame}{\insertframenumber}
	\begin{block}{\textbf{Last Time.}}
	\begin{itemize}[label=--]
		\item Linear Transformations
		\item Properties of Linear Transformations
	\end{itemize}
	\end{block}
	\begin{block}{\textbf{Today.}}
		\begin{itemize}[label=--]
			\item Matrix Transformations
			\item The Kernel of a Transformation
			\item The Range of a Transformation
			\item Rank and Nullity
		\end{itemize}
	\end{block}
\end{frame}
\begin{frame}{\fn}
	\begin{recall}
		If $V$ and $W$ are vector spaces, a \emph{linear transformation} from $V$ to $W$ is a mapping $T\colon V\to W$ that satisfies
			\begin{enumerate}[label=(\alph*)]
				\item $T(\vec v_1+\vec v_2)=T(\vec v_1)+T(\vec v_2)$ for all $\vec v_1,\vec v_2\in V$,
				\item and $T(c\vec v)=cT(\vec v)$ for all scalars $c$ and $\vec v\in V$.
			\end{enumerate}
	\end{recall}
	\begin{exa}
	A very important linear transformation is a \emph{matrix transformation}, $T\colon \R^n\to\R^m$ given by $T(\vec x)=A\vec x$ where $A$ is some fixed $m\times n$ matrix.
	\end{exa}
\end{frame}
%\begin{frame}{\fn}
%	\begin{exercise}
%		Define the linear transformation $T:\R^n\to\R^m$ by $T(\vec v)=\begin{bmatrix}
%		1&1&0&0\\1&0&2&0\\1&0&0&1	\end{bmatrix}\vec v$.
%		\begin{enumerate}[label=(\alph*)]
%			\item What are $n$ and $m$?
%			\item What is $T(1,1,1,1)$?
%			\item What is the preimage of $(1,1,1)$?\\
%				(Hint: You're trying to find vectors $\vec v\in\R^n$ such that $T(\vec v)=(1,1,1)$, what is $T(\vec v)$?)
%			\item What is the preimage of $\vec 0$?
%		\end{enumerate}
%	\end{exercise}
%\end{frame}
\begin{frame}{\fn}
	\begin{defn}
		The \emph{kernel} of the linear transformation $T\colon V\to W$ is the collection of all vectors $\vec v$ in $V$ such that $T(\vec v)=\vec 0$.  The notation for this is,	\[\ker(T)=\{\vec v\in V: T(\vec v)=\vec 0\}.\]
	\end{defn}
	\begin{nb}
		The kernel of a linear transformation is the preimage of $\vec 0$.
	\end{nb}
	\begin{exercise}
		Let $T\colon \R^3\to\R^3$ be defined by $T(x,y,z)=(0,y,z)$.  What is $\ker(T)$?
	\end{exercise}
\end{frame}
\begin{frame}{\fn}
	\begin{defn}
		The \emph{identity transformation} $T\colon V\to V$ is the map $T(\vec v)=\vec v$.  Sometimes we denote this map by $Id_V$.
		
		The \emph{zero transformation} $T\colon V\to W$ is the map $T(\vec v)=\vec 0$.
	\end{defn}
	\begin{exercise}
		What is $\ker(Id_V)$?
		
		What is the kernel of the zero transformation $T\colon V\to W$?
	\end{exercise}
\end{frame}
\begin{frame}{\fn}
	\begin{exercise}
		Consider the transformation $T\colon\R^3\to\R^2$ defined by the matrix
			\[A=\begin{bmatrix}
			-1 & -2 & 4 \\
			2 & 3 & 4
			\end{bmatrix}\]
		What is $\ker(T)$? Carefully write out what it means for $T(\vec v)=\vec 0$.  Answer this question by finding a basis for the kernel.
	\end{exercise}\pause
	\begin{statementblock}{The Kernel of a Matrix Transformation}
		If $T:\R^n\to\R^m$ is a matrix transformation defined by the $m\times n$ matrix $A$, then $\ker(T)=\operatorname{null}(A)$.
	\end{statementblock}
\end{frame}
\begin{frame}{\fn}
	\begin{statementblock}{Theorem 6.3}
		The kernel of a linear transformation $T\colon V\to W$ is a subspace of the domain, $V$.
	\end{statementblock}
	\begin{exercise}
		Prove Theorem 6.3 via the subspace test.	
		\begin{enumerate}[label=(\alph*)]
			\item State why $\ker(T)$ contains $\vec 0$.
			\item Show that if $\vec u$ and $\vec v$ are in $\ker(T)$---meaning $T(\vec u)=\vec 0$ and $T(\vec v)=\vec 0$---then $\vec u+\vec v\in\ker(T)$ too.
			\item Show that if $c$ is a scalar, then $c\vec u$ is also in $\ker(T)$.
		\end{enumerate}
	\end{exercise}
\end{frame}
\begin{frame}{\fn}
\begin{block}{\textbf{Brain Break.}}
	When you were a kid, what did you want to be when you grew up?
	\begin{center}
		\includegraphics[height=.3\textheight]{../images/dream_job}
		
		\tiny Image source \url{https://www.flexjobs.com/blog/post/childhood-dream-job-pay-well/}
	\end{center}
\end{block}
\end{frame}
\begin{frame}{\fn}
	\begin{recall}
		The range of a transformation $T\colon V\to W$ is the collection of all possible images under the transformation.  We write
			\[\operatorname{range}(T)=\{T(\vec v): \vec v\in V\}.\]
	\end{recall}
	\begin{statementblock}{Theorem 6.4}
		The range of a linear transformation $T\colon V\to W$ is a subspace of the codomain $W$.
	\end{statementblock}
	\begin{block}{Proof Idea.}
		Take two generic elements of the range of $T$, $T(\vec u)$ and $T(\vec v)$.  Then by the definition of linear transformation,
			\[T(\vec u)+T(\vec v)=T(\vec u+\vec v).\]
		Therefore the sum of the two generic elements is also in the form of $\operatorname{range}(T)$.  Similarly, this can be one with scalar multiples.
	\end{block}
\end{frame}
\begin{frame}{\fn}
	\begin{statementblock}{The Range of a Matrix Transformation}
		If $T\colon \R^n\to\R^m$ is a matrix transformation defined by the $m\times n$ matrix $A$, then $\operatorname{range}(T)=\operatorname{col}(A)$.
	\end{statementblock}
	\begin{exercise}
	Consider the transformation $T\colon \R^3\to\R^2$ defined by the matrix
	\[A=\begin{bmatrix}
	-1 & -2 & 4 \\
	2 & 3 & 4
	\end{bmatrix}\]
	What is $\operatorname{range}(T)$? Answer this question by finding a basis for the range.
	\end{exercise}
\end{frame}
\begin{frame}{\fn}
	\begin{defn}
		Let $T\colon V\to W$ be a linear transformation.  The \emph{nullity} of $T$, denoted $\operatorname{nullity}(T)$, is the dimension of $\ker(T)$.
		
		The \emph{rank} of $T$, denoted $\rank(T)$, is the dimension of $\operatorname{range}(T)$.
	\end{defn}
	\begin{exercise}
		Consider the derivative transformation, $D\colon P_2\to P_1$.  What is the nullity and rank of $D$?
	\end{exercise}
\end{frame}
\begin{frame}{\fn}
	\begin{statementblock}{Theorem 6.5}
		Let $T\colon V\to W$ be a linear transformation from an $n$-dimensional vector space $V$ into a vector space $W$.  Then 
			\[\dim(V)=\rank(T)+\operatorname{nullity}(T).\]
	\end{statementblock}
	\begin{exercise}
		Does this match up with the fact that if $A$ is an $m\times n$ matrix then \[n=\rank(A)+\operatorname{nullity}(A)?\]
	\end{exercise}
\end{frame}
\begin{frame}{\fn}
	\begin{exercise}
		Find the rank and nullity of the matrix transformation $T\colon \R^3\to\R^3$ defined by	
			\[A=\begin{bmatrix}
			1 & 0 & -6 \\
			0 & 1 & -4\\
			0&0&0
			\end{bmatrix}\]
	\end{exercise}
\end{frame}
\end{document}

