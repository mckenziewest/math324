\documentclass[handout]{beamer}
%\usepackage[margin=1in]{geometry}
\usepackage{amsthm,amsmath,amsfonts,hyperref,graphicx,color,multicol}
\usepackage{enumitem,tikz}

%%%%%%%%%%
%Beamer Template Customization
%%%%%%%%%%
\setbeamertemplate{navigation symbols}{}
\setbeamertemplate{theorems}[ams style]
\setbeamertemplate{blocks}[rounded]

\definecolor{Blu}{RGB}{43,62,133} % UWEC Blue
\setbeamercolor{structure}{fg=Blu} % Titles

%Unnumbered footnotes:
\newcommand{\blfootnote}[1]{%
	\begingroup
	\renewcommand\thefootnote{}\footnote{#1}%
	\addtocounter{footnote}{-1}%
	\endgroup
}


%%%%%%%%%%
%Custom Commands
%%%%%%%%%%
\newcommand{\R}{\mathbb{R}}
\newcommand{\veca}{\vec{a}}
\newcommand{\vecb}{\vec{b}}
\newcommand{\vece}{\vec{e}}
\newcommand{\vecu}{\vec{u}}
\newcommand{\vecv}{\vec{v}}
\newcommand{\vecw}{\vec{w}}
\newcommand{\vecx}{\vec{x}}
\newcommand{\zerovector}{\vec{0}}

\newcommand{\ds}{\displaystyle}

\newcommand{\fn}{\insertframenumber}

\newcommand{\rank}{\operatorname{rank}}
\newcommand{\adj}{\operatorname{adj}}

\newcommand{\blank}[1]{\underline{\hspace*{#1}}}


%%%%%%%%%%
%Custom Theorem Environments
%%%%%%%%%%
\theoremstyle{definition}
\newtheorem{exercise}{Exercise}
\newtheorem{question}[exercise]{Question}
\newtheorem*{defn}{Definition}
\newtheorem*{exa}{Example}
\newtheorem*{disc}{Group Discussion}
\newtheorem*{nb}{Note}
\newtheorem*{recall}{Recall}
\renewcommand{\emph}[1]{{\color{blue}\texttt{#1}}}

\definecolor{Gold}{RGB}{237, 172, 26}
%Statement block
\newenvironment{statementblock}[1]{%
	\setbeamercolor{block body}{bg=Gold!20}
	\setbeamercolor{block title}{bg=Gold}
	\begin{block}{\textbf{#1.}}}{\end{block}}


\begin{document}
	\title{Math 324: Linear Algebra}
	\subtitle{2.1: Operations with Matrices}
	\author{Mckenzie West}
	\date{Last Updated: \today}
\begin{frame}
\maketitle
\end{frame}

\begin{frame}{\insertframenumber}
	\begin{block}{\textbf{Last Time.}}
	\begin{itemize}[label=--]
		\item Matrix Equality
		\item Matrix Addition, Subtraction and Scalar Multiplication
		\item Matrix Multiplication
	\end{itemize}
	\end{block}
\begin{block}{\textbf{Today.}}
	\begin{itemize}[label=--]
		\item Row and Column Matrices
		\item Linear Combinations
		\item Linear Systems as Matrix Products
	\end{itemize}
\end{block}
\end{frame}

\begin{frame}{\fn}
	\begin{defn}
		A matrix that has only one column is called a \emph{column matrix} or \emph{column vector}.
		
		A matrix that has only one row is called a \emph{row matrix} or \emph{row vector}.
	\end{defn}
	\begin{nb}
		\begin{itemize}[label = --]
			\item Column matrices are often denoted by either bold lowercase letters $\mathbf{a}$ or by lowercase letters with an arrow over the top $\vec{a}$.
			\item Every $m\times n$ matrix $A$ consists of $m$ row vectors and $n$ column vectors which may be denoted by $\mathbf{a}_i$.  It is essential to specify \textit{row} or \textit{column} when doing this. 
			\item When using $\mathbf{a}_j$, $1\leq j\leq n$, to denote column vectors, we can write
			\[A=\begin{bmatrix}
			\mathbf{a}_1&\mathbf{a}_2&\cdots & \mathbf{a}_n
			\end{bmatrix}\]
			\item What would it look like if we used $\mathbf{a}_i$ to denote the rows of $A$?
		\end{itemize}
	\end{nb}
\end{frame}


\begin{frame}{\fn}
	\begin{exercise}
		Let $A=\begin{bmatrix}
		1&2\\3&4
		\end{bmatrix}$ and let $\mathbf{a}_1 $ and $\mathbf{a}_2$ be the columns of $A$.  
		
		What are $\mathbf{a}_1$ and $\mathbf{a}_2$ explicitly?
	\end{exercise}
	\begin{exercise}
		Compute the product
			\[\begin{bmatrix}
			1&2&-3\\
			7&-5&2
			\end{bmatrix}
			\begin{bmatrix}
			x\\y\\z
			\end{bmatrix}\]
	\end{exercise}
\end{frame}
\begin{frame}{\fn}
	\begin{nb} 
		From the exercise, we see that the linear system of equations:
			\begin{eqnarray*}
			x+2y-3z&=&3,\\7x-5y+2z&=&0,
			\end{eqnarray*} 
		can be written as a matrix equation:
	\[\begin{bmatrix}
		1&2&-3\\
		7&-5&2
		\end{bmatrix}
		\begin{bmatrix}
		x\\y\\z
		\end{bmatrix}
		=
		\begin{bmatrix}
		3\\0
		\end{bmatrix}\!.
		\]
		
	Let $A$ be the coefficient matrix of a system, $\vec x$ the column vector of variables, and $\vecb$ the column matrix containing the constant terms.  Then a linear system of equations can be written as the equality
	\[A\vecx=\vecb.\]
	\end{nb}
\end{frame}

\begin{frame}{\fn}
	\begin{exercise}
		Write the system of linear equations in the form $A\vecx=\vecb$ and solve the matrix equation for $\vecx$.
		\[
		\begin{array}{rcrcrcr}
			x_1&-&2x_2&+&3x_3&=&9\\
			-x_1&+&3x_2&-&x_3&=&-6\\
			2x_1&-&5x_2&+&5x_3&=&17
		\end{array}
		\]
	\end{exercise}
\end{frame}
\begin{frame}{\fn}
	\begin{block}{\textbf{Brain Break.}}
		What sport is your favorite to watch?
		\begin{center}
			\includegraphics[width=2in]{../images/sports}
		\end{center}
	\end{block}
\end{frame}
\begin{frame}{\fn}
	\begin{defn}
		A \emph{linear combination} of the column matrices $\veca_1,\veca_2,\dots,\veca_n$ with coefficients $x_1,x_2,\dots,x_n$ is the vector $\vecb$ given by 
		\[x_1\veca_1+x_2\veca_2+\cdots+x_n\veca_n=\vecb\]
	\end{defn}
	\begin{exa}
		The column matrix $\vecb=\begin{bmatrix}10\\8\end{bmatrix}$ is a linear combination of $\veca_1=\begin{bmatrix}3\\2\end{bmatrix}$ and $\veca_2=\begin{bmatrix}-2\\0\end{bmatrix}$ because
		\[
		4\begin{bmatrix}3\\2\end{bmatrix}+
		1\begin{bmatrix}-2\\0\end{bmatrix}
		=\begin{bmatrix}
		10\\8
		\end{bmatrix}
		\]
	\end{exa}
\end{frame}

\begin{frame}{\fn}
	\begin{exercise}
		Can $\vecb=\begin{bmatrix}0\\3\\6\end{bmatrix}$ be written as a linear combination of \[\veca_1=\begin{bmatrix}1\\3\\2\end{bmatrix}, \veca_2=\begin{bmatrix}2\\0\\5\end{bmatrix},\textup{ and } \veca_3=\begin{bmatrix}-1\\2\\3\end{bmatrix}?\]
		
		Hint: Work entry by entry - think in terms of matrix equality and linear equations.
	\end{exercise}
\end{frame}
\end{document}

