\documentclass[handout]{beamer}
%\usepackage[margin=1in]{geometry}
\usepackage{amsthm,amsmath,amsfonts,hyperref,graphicx,color,multicol}
\usepackage{enumitem,tikz}

%%%%%%%%%%
%Beamer Template Customization
%%%%%%%%%%
\setbeamertemplate{navigation symbols}{}
\setbeamertemplate{theorems}[ams style]
\setbeamertemplate{blocks}[rounded]

\definecolor{Blu}{RGB}{43,62,133} % UWEC Blue
\setbeamercolor{structure}{fg=Blu} % Titles

%Unnumbered footnotes:
\newcommand{\blfootnote}[1]{%
	\begingroup
	\renewcommand\thefootnote{}\footnote{#1}%
	\addtocounter{footnote}{-1}%
	\endgroup
}


%%%%%%%%%%
%Custom Commands
%%%%%%%%%%
\newcommand{\R}{\mathbb{R}}
\newcommand{\veca}{\vec{a}}
\newcommand{\vecb}{\vec{b}}
\newcommand{\vece}{\vec{e}}
\newcommand{\vecu}{\vec{u}}
\newcommand{\vecv}{\vec{v}}
\newcommand{\vecw}{\vec{w}}
\newcommand{\vecx}{\vec{x}}
\newcommand{\zerovector}{\vec{0}}

\newcommand{\ds}{\displaystyle}

\newcommand{\fn}{\insertframenumber}

\newcommand{\rank}{\operatorname{rank}}
\newcommand{\adj}{\operatorname{adj}}

\newcommand{\blank}[1]{\underline{\hspace*{#1}}}


%%%%%%%%%%
%Custom Theorem Environments
%%%%%%%%%%
\theoremstyle{definition}
\newtheorem{exercise}{Exercise}
\newtheorem{question}[exercise]{Question}
\newtheorem*{defn}{Definition}
\newtheorem*{exa}{Example}
\newtheorem*{disc}{Group Discussion}
\newtheorem*{nb}{Note}
\newtheorem*{recall}{Recall}
\renewcommand{\emph}[1]{{\color{blue}\texttt{#1}}}

\definecolor{Gold}{RGB}{237, 172, 26}
%Statement block
\newenvironment{statementblock}[1]{%
	\setbeamercolor{block body}{bg=Gold!20}
	\setbeamercolor{block title}{bg=Gold}
	\begin{block}{\textbf{#1.}}}{\end{block}}

\begin{document}
	\title{Math 324: Linear Algebra}
	\subtitle{2.4: Elementary Matrices}
	\author{Mckenzie West}
	\date{Last Updated: \today}
\begin{frame}
\maketitle
\end{frame}

\begin{frame}{\insertframenumber}
	\begin{block}{\textbf{Last Time.}}
	\begin{itemize}[label=--]
		\item Properties of Inverses
		\item Using Inverses to Solve System of Equations
	\end{itemize}
	\end{block}
\begin{block}{\textbf{Today.}}
	\begin{itemize}[label=--]
		\item Theorem 2.15
		\item Elementary Matrices
		\item Row Equivalence
		\item Writing a Matrix as a Product of Elementary Matrices
		\item Elementary Matrices and Inverses
	\end{itemize}
\end{block}
\end{frame}


\begin{frame}{\fn}
	\begin{exercise}[Warm-up]
		Let $A=\begin{bmatrix}
		1&-2\\3&1
		\end{bmatrix}$.
		Use $A^{-1}$ to solve the system of equations \[A\begin{bmatrix}x\\y\end{bmatrix}=\begin{bmatrix}-3\\0\end{bmatrix}.\]
	\end{exercise}
\end{frame}
\begin{frame}{\fn}
	\begin{statementblock}{Theorem 2.15 (1-4)}
		If $A$ is an $n\times n$ matrix, then the following statements are equivalent:
		\begin{enumerate}[label=\arabic*.]
			\item $A$ is invertible.
			\item $A\vecx=\vecb$ has a unique solution for every $n\times 1$ column matrix $\vecb$.
			\item $A\vecx=\zerovector$ has only the trivial solution.
			\item $A$ is \emph{row-equivalent} to $I_n$, meaning there are elementary row operations that transform $A$ to $I_n$.
		\end{enumerate}
	\end{statementblock}
	\begin{exercise}
		What Theorem says (1) implies (2)?
		
		Why does (2) imply (3)?
		
		Why does (3) imply (4)?
		
		Why does (4) imply (1)? (Cite the method used here.)
		
		Given these, there is a loop of implications implying equivalence:\vskip -.15in
		\[(1)\Rightarrow(2)\Rightarrow(3)\Rightarrow(4)\Rightarrow(1)\]
	\end{exercise}
\end{frame}
\begin{frame}{\fn}
	\begin{defn}
		An $n\times n$ \emph{elementary matrix} is any matrix that can be obtained by performing a single elementary row operation on the identity matrix.
	\end{defn}
	\begin{exercise}
		Which of the following matrices are elementary matrices?
		\begin{multicols}{2}
			\begin{enumerate}[label=(\alph*)]
				\item $\begin{bmatrix}1&0&0\\0&3&0\\0&0&1\end{bmatrix}$
				\item $\begin{bmatrix}1&0&0\\0&1&0\end{bmatrix}$
				\item $\begin{bmatrix}1&0&0\\0&1&0\\0&0&0\end{bmatrix}$
				\item $\begin{bmatrix}1&0&0\\0&0&1\\0&1&0\end{bmatrix}$
				\item $\begin{bmatrix}1&0\\2&1\end{bmatrix}$
				\item $\begin{bmatrix}1&0&0\\0&2&0\\0&0&-1\end{bmatrix}$
			\end{enumerate}
		\end{multicols}
	\end{exercise}
\end{frame}
\begin{frame}{\fn}
	\begin{exercise}
		Compute each of the products
		\begin{enumerate}[label=(\alph*)]
			\item $\begin{bmatrix}1&0&0\\0&0&1\\0&1&0\end{bmatrix}\left[\begin{array}{rrr}
			-1 & 1 & -1 \\
			1 & -2 & 2 \\
			-1 & -4 & -3
			\end{array}\right]$
			\item $\begin{bmatrix}1&0&0\\0&3&0\\0&0&1\end{bmatrix}\left[\begin{array}{rr}
			4 & 4 \\
			3 & -5 \\
			0 & 3
			\end{array}\right]$
			\item $\begin{bmatrix}1&0\\2&1\end{bmatrix} \left[\begin{array}{rrrrr}
			3 & -1 & -5 & -5 & 0 \\
			1 & 5 & 4 & -3 & 5
			\end{array}\right]$
		\end{enumerate}
		How can you read the products from the elementary matrix without actually having to go through the process of matrix multiplication?
	\end{exercise}
\end{frame}

\begin{frame}{\fn}
	\begin{statementblock}{Theorem 2.12}
		Let $E$ be the elementary matrix obtained by performing an elementary row operation on $I_m$.  If that same elementary row operation is performed on an $m\times n$ matrix $A$, then the resulting matrix is given by the product $EA$.
	\end{statementblock}
	\begin{defn}
		We say $A$ is \emph{row equivalent} to $B$ if there exist elementary matrices $E_1,\dots,E_k$ such that $E_k\cdots E_2E_1A=B$.
	\end{defn}
\end{frame}
\begin{frame}{\fn}
	\begin{exercise}
		Let $A=\begin{bmatrix}
		0 & -1 & 3 \\
		5 & 4 & -2
		\end{bmatrix}$
		\begin{enumerate}[label=(\alph*)]
			\item Manually, without technology, use elementary row operations to reduce $A$ to row-echelon form.
			\item Encode each of those elementary row operations as an elementary matrix.
			\item Multiply the elementary matrices you just found by $A$ in the correct order to verify that you got the right sequence.
		\end{enumerate}
	\end{exercise}
\end{frame}

\begin{frame}{\fn}
	\begin{block}{\textbf{Brain Break.}}
		$\sim$NAMES$\sim$
		
		What song always gets stuck in your head?
		
		\begin{center}
			\includegraphics[width=2in]{images/song}
		\end{center}
	\end{block}
\end{frame}
\begin{frame}{\fn}
	\begin{nb}
		Elementary row operations can always be undone.
	\end{nb}
	\begin{exercise}
		What would you do to undo the following ERO's?
		\begin{multicols}{3}\begin{enumerate}[label=(\alph*)]
			\item $R_2\leftrightarrow R_3$
			\item $7R_3$
			\item $R_1+5R_3\rightarrow R_1$
		\end{enumerate}
		\end{multicols}
	\end{exercise}
\begin{exercise}
	What is the inverse of each of the following elementary matrices?
	\begin{multicols}{3}
		\begin{enumerate}[label=(\alph*)]
		\item $\begin{bmatrix}1&0&0\\0&0&1\\0&1&0\end{bmatrix}$
		\item $\begin{bmatrix}1&0&0\\0&1&0\\0&0&7\end{bmatrix}$
		\item $\begin{bmatrix}1&0&5\\0&1&0\\0&0&1\end{bmatrix}$
	\end{enumerate}
\end{multicols}
	\end{exercise}
\end{frame}

\begin{frame}{\fn}
	\begin{statementblock}{Theorem 2.13}
		If $E$ is an elementary matrix, then $E^{-1}$ exists and is also an elementary matrix.
	\end{statementblock}
	\begin{statementblock}{Theorem 2.14}
		An $n\times n$ matrix $A$ is invertible if and only if it can be written as a product of elementary matrices.
	\end{statementblock}
\end{frame}
\begin{frame}{\fn}
\begin{exercise}\label{productOfElementary}
	Let $A=\begin{bmatrix}
	0 & 1  \\
	5 & 4
	\end{bmatrix}$
	\begin{enumerate}[label=(\alph*)]
		\item Manually, without technology, use elementary row operations to transform $A$ to reduced row-echelon form. (Note: $A$ is invertible so it can be reduced to $I_2$)
		\item Encode each of those elementary row operations as an elementary matrix.
		\item Multiply the elementary matrices you just found by $A$ in the correct order to verify that you got the right sequence.
		\item Taking inverses of the ERO's in the correct order, solve for $A$.  Thus writing $A$ as a product of elementary matrices.
	\end{enumerate}
\end{exercise}
\begin{question}
	Using Exercise \ref{productOfElementary} as motivation, why is Theorem 2.14 true?
\end{question}
\end{frame}

\begin{frame}{\fn}
	\begin{statementblock}{Theorem 2.15}
		If $A$ is an $n\times n$ matrix, then the following statements are equivalent:
		\begin{enumerate}[label=\arabic*.]
			\item $A$ is invertible.
			\item $A\vecx=\vecb$ has a unique solution for every $n\times 1$ column matrix $\vecb$.
			\item $A\vecx=\zerovector$ has only the trivial solution.
			\item $A$ is row-equivalent to $I_n$.
			\item $A$ can be written as a product of elementary matrices.
		\end{enumerate}
	\end{statementblock}
\end{frame}

\begin{frame}{\fn}
	\begin{exercise}
		We can also find inverses by multiplying elementary matrices using the same process as we did to write $A$ as a product of elementary matrices.
		
		Again, by hand, writing down each step, determine the reduced row-echelon form of $A$:
			\[A=\begin{bmatrix}1&2&3\\2&5&6\\1&3&4\end{bmatrix}\]
		Then write the product $(E_k\cdots E_2E_1)A=I_3$.
		
		What do you think $A^{-1}$ is?
		
		Multiply $A(E_k\cdots E_2E_1)$.  What did you get?
	\end{exercise}
\end{frame}

\begin{frame}{\fn}
\begin{exercise}
	(Yes-this was part of the pre-class homework. Make sure you all agree on the answers.)
	
	Determine whether each statement is True or False.
	
	A statement is \emph{True} if it is always true no matter what.
	
	A statement is \emph{False} if you can write down some example where it fails.
	
	For every true statement, explain why it is true.
	For every false statement, give an example of when it fails.
	
	\begin{enumerate}[label=(\alph*)]
		\item The identity matrix is an elementary matrix.
		\item If $E$ is an elementary matrix, then $2E$ is an elementary matrix.
		\item The inverse of an elementary matrix is an elementary matrix.
	\end{enumerate}
\end{exercise}
\end{frame}

\begin{frame}{\fn}
	\begin{question}
		If $A$ is row equivalent to $B$ and $B$ is row equivalent to $C$, does this mean that $A$ is row equivalent to $C$? Why or why not?
	\end{question}
	\begin{question}
		If $A$ is row equivalent to $B$ and $A$ is nonsingular, does this mean that $B$ is also nonsingular? Why or why not?
	\end{question}
\end{frame}

\begin{frame}{\fn}
	\begin{recall}
		Last time we saw that if $A^2=A$ and $A$ is invertible then $A=I$.  
		
		In general if $A^2=A$, then we call $A$ \emph{idempotent}.
	\end{recall}
	\begin{exercise}
		Which of the following matrices are idempotent?
		\begin{multicols}{2}
			\begin{enumerate}[label=(\alph*)]
				\item $\begin{bmatrix}1&0\\0&0\end{bmatrix}$
				\item $\begin{bmatrix}0&1\\1&0\end{bmatrix}$
				\item $\begin{bmatrix}0&0&1\\0&1&0\\1&0&0\end{bmatrix}$
				\item $\begin{bmatrix}0&1&0\\1&0&0\\0&0&1\end{bmatrix}$
			\end{enumerate}
		\end{multicols}
	\end{exercise}
\end{frame}

\begin{frame}{\fn}
\begin{exercise}
	Determine $a$ and $b$ such that $A$ is idempotent.
	\[A=\begin{bmatrix}1&0\\a&b\end{bmatrix}\]
\end{exercise}
\begin{question}
	If $A$ is idempotent and invertible, is $A^{-1}$ also idempotent? Why or why not?
\end{question}
\begin{question}
	If $A$ and $B$ are idempotent, is $AB$ also idempotent? Why or why not?
\end{question}
\end{frame}
\end{document}

 