\documentclass[handout]{beamer}
%\usepackage[margin=1in]{geometry}
\usepackage{amsthm,amsmath,amsfonts,hyperref,graphicx,color,multicol}
\usepackage{enumitem,tikz}

%%%%%%%%%%
%Beamer Template Customization
%%%%%%%%%%
\setbeamertemplate{navigation symbols}{}
\setbeamertemplate{theorems}[ams style]
\setbeamertemplate{blocks}[rounded]

\definecolor{Blu}{RGB}{43,62,133} % UWEC Blue
\setbeamercolor{structure}{fg=red} % Titles

%Unnumbered footnotes:
\newcommand{\blfootnote}[1]{%
	\begingroup
	\renewcommand\thefootnote{}\footnote{#1}%
	\addtocounter{footnote}{-1}%
	\endgroup
}


%%%%%%%%%%
%Custom Commands
%%%%%%%%%%
\newcommand{\R}{\mathbb{R}}
\newcommand{\veca}{\vec{a}}
\newcommand{\vecb}{\vec{b}}
\newcommand{\vece}{\vec{e}}
\newcommand{\vecu}{\vec{u}}
\newcommand{\vecv}{\vec{v}}
\newcommand{\vecw}{\vec{w}}
\newcommand{\vecx}{\vec{x}}
\newcommand{\zerovector}{\vec{0}}

\newcommand{\ds}{\displaystyle}

\newcommand{\fn}{\insertframenumber}

\newcommand{\rank}{\operatorname{rank}}
\newcommand{\adj}{\operatorname{adj}}

\newcommand{\blank}[1]{\underline{\hspace*{#1}}}


%%%%%%%%%%
%Custom Theorem Environments
%%%%%%%%%%
\theoremstyle{definition}
\newtheorem{exercise}{Exercise}
\newtheorem{question}[exercise]{Question}
\newtheorem*{defn}{Definition}
\newtheorem*{exa}{Example}
\newtheorem*{disc}{Group Discussion}
\newtheorem*{nb}{Note}
\newtheorem*{recall}{Recall}
\renewcommand{\emph}[1]{{\color{red}\texttt{#1}}}

\definecolor{Gold}{RGB}{237, 172, 26}
%Statement block
\newenvironment{statementblock}[1]{%
	\setbeamercolor{block body}{bg=pink!20}
	\setbeamercolor{block title}{bg=pink}
	\begin{block}{\textbf{#1.}}}{\end{block}}

\begin{document}
	\title{Math 324: Linear Algebra}
	\subtitle{2.3: The Inverse of a Matrix}
	\author{Mckenzie West}
	\date{Last Updated: \today}
\begin{frame}
\maketitle
\begin{picture}(0,0)
\put(250,-25){\includegraphics[width=1in]{../images/penguin_hearts}}
\end{picture}
\end{frame}

\begin{frame}{\insertframenumber}
	\begin{block}{\textbf{Last Time.}}
	\begin{itemize}[label=--]
		\item Properties of Matrix Addition
		\item Additive and Multiplicative Identities
		\item Properties of Matrix Multiplication
		\item Transposes and Properties of
	\end{itemize}
	\end{block}
\begin{block}{\textbf{Today.}}
	\begin{itemize}[label=--]
		\item Inverses of Matrices
		\item Gauss-Jordan Elimination
		\item Inverses of $2\times2$ matrices
	\end{itemize}
\end{block}
\end{frame}

\begin{frame}{\fn}
	\begin{exercise}[Warm-up]
		Compute the product of the matrices:
			\begin{enumerate}[label=(\alph*)]
				\item $\begin{bmatrix}-1&-3\\2&4\end{bmatrix}
				\begin{bmatrix}2\\-1\end{bmatrix}$
				\item $\begin{bmatrix}-1&-3\\2&4\end{bmatrix}
				\begin{bmatrix}a&b\\c&d\end{bmatrix}$
			\end{enumerate}
	\end{exercise}
\end{frame}

\begin{frame}{\fn}
	\begin{nb}
		We are NEVER going to be able to divide by matrices.
		
		But maybe we can multiply to get the identity matrix (also known as 1 in matrix land).
	\end{nb}
	\begin{defn}
		An $n\times n$ matrix $A$ is \emph{invertible} (or \emph{nonsingular}) when there is an $n\times n$ matrix $B$ satisfying 
			\[AB=BA=I_n.\]
		The matrix $B$ is called an \emph{inverse} of $A$.
		
		If $A$ does not have an inverse, we call it \emph{noninvertible} (or \emph{singular}).
	\end{defn}
\end{frame}
\begin{frame}{\fn}
	\begin{exa}
		The matrix $A=\begin{bmatrix}
		13&1\\-1&0
		\end{bmatrix}$ is invertible because
			\[A\begin{bmatrix}
			0&-1\\1&13
			\end{bmatrix}=
			\begin{bmatrix}
			1&0\\0&1
			\end{bmatrix}\textup{ and }
			\begin{bmatrix}
			0&-1\\1&13
			\end{bmatrix}A=
			\begin{bmatrix}
			1&0\\0&1
			\end{bmatrix}\]
	\end{exa}
	\begin{exercise}
		The following matrices are singular.  Is there a pattern?
			\[\left[\begin{array}{rr}
			-2 & 1 \\
			-6 & 3
			\end{array}\right]
			\left[\begin{array}{rr}
			3 & -5 \\
			\frac{1}{2} & -\frac{5}{6}
			\end{array}\right]
			\left[\begin{array}{rr}
			-5 & 0 \\
			17 & 0
			\end{array}\right]
			\left[\begin{array}{rr}
			5 & -5 \\
			-1 & 1
			\end{array}\right]\]
	\end{exercise}
\end{frame}

\begin{frame}{\fn}
	\begin{nb}
		An invertible matrix only has one inverse.
	\end{nb}
	\begin{statementblock}{Theorem 2.7}
		If $A$ is an invertible matrix, then its inverse is unique and can be denoted by $A^{-1}$.
	\end{statementblock}
	\begin{center}
		\begin{picture}(100,0)
	\put(0,-60){\includegraphics[width=1in]{../images/snoopy_heart}}
	\end{picture}
	\end{center}
\end{frame}

\begin{frame}{\fn}
	\begin{statementblock}{The Gauss-Jordan Method for Inverses}
		Let $A$ be a square matrix of order $n$.
		\begin{enumerate}[label=\arabic*.]
			\item Write the $n\times 2n$ matrix that consists of the given matrix $A$ on the left and the $n\times n$ identity matrix $I$ on the right to obtain $\begin{bmatrix}A&\vdots& I\end{bmatrix}$.  This process is called \emph{adjoining} $I$ to $A$.
			\item If possible, row reduce $A$ to $I$ using elementary row operations on the entire matrix $\begin{bmatrix}A&\vdots& I\end{bmatrix}$.  The result will be the matrix $\begin{bmatrix}I&\vdots& A^{-1}\end{bmatrix}$.  If this is not possible, then $A$ is noninvertible (or singular).
			\item Check your work by multiplying $AA^{-1}=I=A^{-1}A$.
		\end{enumerate}
	\end{statementblock}
\end{frame}

\begin{frame}{\fn}
	\begin{exercise}
		By hand, without a calculator, determine whether each of the following are invertible. If so, what is the inverse?
		\begin{enumerate}[label=(\alph*)]
			\item $A=\begin{bmatrix}1 & 1 & 0 \\-1 & 0 & 1 \\0 & 0 & -1\end{bmatrix}$
			\item $B=\begin{bmatrix}1 & -1 & 0 \\0 & 0 & 1 \\-1 & 1 & 0\end{bmatrix}$
			\item $C=\begin{bmatrix}1 & 0 & -1 \\-1 & 0 & 0 \\-1 & 0 & 1\end{bmatrix}$
		\end{enumerate}
	\end{exercise}
\end{frame}
\begin{frame}{\fn}
	\begin{exercise}
		Use the Gauss-Jordan method to find the inverse of the matrix if it exists.  You should use a calculator or Sage to row reduce.
		\begin{enumerate}[label=(\alph*)]
			\item $A=\left[\begin{array}{rrr}
			2 & -4 & 8 \\
			-7 & -3 & 10 \\
			10 & -3 & 2
			\end{array}\right]$
			\item $B=
			\left[\begin{array}{rrr}
			-3 & 5 & -2 \\
			3 & -5 & 1 \\
			-4 & 7 & -6
			\end{array}\right]$
			\item $C=\left[\begin{array}{rrrr}
			5 & 8 & 5 & 6 \\
			1 & -10 & 9 & -6 \\
			-3 & -2 & 1 & -5 \\
			6 & -4 & -7 & -2
			\end{array}\right]$
		\end{enumerate}	
	\end{exercise}
\end{frame}

\begin{frame}{\fn}
	\begin{block}{\textbf{Brain Break.}}
		Always start these by reminding your group members of your name.  Names are hard.
		
		What acquired skill have you always wanted to learn?
		\begin{center}
			\includegraphics[width=2in]{../images/skill_tree}
		\end{center}
	\end{block}
\end{frame}

\begin{frame}{\fn}
	\begin{exercise}
		Let's discover the nice formula for the inverse of a $2\times 2$ matrix.
		\begin{enumerate}[label=(\alph*)]
			\item Take a generic $2\times 2$ matrix 
			$A=\begin{bmatrix}a&b\\c&d\end{bmatrix}$, and adjoin it by $I_2$.
			\pause
			\item Do the following elementary row operations:
				\begin{enumerate}[label=\roman*.]
					\item $aR_2\rightarrow R_2$
					\item $R_2-cR_1\rightarrow R_2$
				\end{enumerate}
			\pause
			\item At this stage, what must be true in order for $A$ to have an inverse?
		\end{enumerate}
	\end{exercise}
\end{frame}
\begin{frame}{\fn}
	\begin{block}{\textbf{Exercise 4. (Continued)}}
		\begin{enumerate}[label=(\alph*)]
			\item[(d)] Do the following elementary row operations:
			\begin{enumerate}[label=\roman*.]
				\item[iii.] $\frac{1}{ad-bc}R_2\rightarrow R_2$
				\item[iv.] $R_1-bR_2\rightarrow R_1$
				\item[v.] $\frac{1}{a}R_1\rightarrow R_1$
			\end{enumerate}
			\pause
			\item[(e)] Simplify the (1,1)-entry of the inverse.  Did you get:\vskip -.15in
				\[\frac{1}{ad-bc}\begin{bmatrix}d&-b\\-c&a\end{bmatrix}?\]
		\end{enumerate}
	\end{block}
	\pause
	\begin{block}{\textbf{Warning/Challenge.}}
		We cheated on step i of our elementary row operations by scaling by $a$ --- $a$ could have been 0. 
		
		Find the inverse of the matrix $\begin{bmatrix}
		0&b\\c&d
		\end{bmatrix}$.
		
		Hint: The same formula works. Why?
	\end{block}
\end{frame}
\begin{frame}{\fn}
	\begin{statementblock}{Proposition}
		If $A$ is a $2\times 2$ matrix given by
			\[A=\begin{bmatrix} a&b\\c&d\end{bmatrix},\]
		then $A$ is invertible if and only if $ad-bc\neq 0$.  Moreover if $ad-bc\neq 0$, then the inverse is given by
		\[A^{-1}=\frac{1}{ad-bc}\begin{bmatrix}d&-b\\-c&a\end{bmatrix}.\]
	\end{statementblock}
\end{frame}
\begin{frame}{\fn}
	\begin{exercise}
		Which is invertible?  Find the inverse of those that are:
		\begin{enumerate}[label=(\alph*)]
			\item $\left[\begin{array}{rr}
				8 & -2 \\
				-1 & -9
			\end{array}\right]$
			\item $\left[\begin{array}{rr}
			8 & -2 \\
			4 & -1
			\end{array}\right]$
			\item $\left[\begin{array}{rr}
			2 & 7 \\
			0 & 7
			\end{array}\right]$
			\item $\left[\begin{array}{rr}
			-9 & -5 \\
			8 & -10
			\end{array}\right]$
		\end{enumerate}
	\end{exercise}
\end{frame}
\begin{frame}{\fn}
	\begin{exercise}
		Find all $x$ and $y$ that make the matrix singular
			$\begin{bmatrix}
				2&x\\-3&y
			\end{bmatrix}$
	\end{exercise}
	\begin{exercise}
		Find $x$ so that $A=A^{-1}$ if $A=\left[\begin{array}{rr}
		-2 & x \\
		3 & 2
		\end{array}\right]$
	\end{exercise}
\end{frame}
\begin{frame}{\fn}
	\begin{exercise}
		Under what conditions will the diagonal matrix
			\[A=\begin{bmatrix}a_{11}&0&0&\cdots&0\\
				0&a_{22}&0&\cdots&0\\
				\vdots&\vdots&\vdots&&\vdots\\
				0&0&0&\cdots&a_{nn}\end{bmatrix}\]
		be invertible? 
		
		In the case that $A$ is invertible, what is its inverse?
	\end{exercise}
\end{frame}
\begin{frame}{\fn}
	\begin{exercise}
		Let $A=\left[\begin{array}{rr}
	-3 & -2 \\
	3 & -1
	\end{array}\right]$.
	\begin{enumerate}[label=(\alph*)]
		\item Verify that $A^2+4A+9I=O$.
		\item Verify that $A^{-1}=\frac{-1}{9}(A+4I)$.
		\item Show that for any square matrix satisfying $A^2+4A+9I=O$, the inverse of $A$ is given by $A^{-1}=\frac{-1}{9}(A+4I)$.
	\end{enumerate}
	\end{exercise}
\end{frame}
\end{document}

 