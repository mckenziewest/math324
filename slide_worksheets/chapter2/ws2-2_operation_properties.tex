\documentclass[handout]{beamer}
%\usepackage[margin=1in]{geometry}
\usepackage{amsthm,amsmath,amsfonts,hyperref,graphicx,color,multicol}
\usepackage{enumitem,tikz}

%%%%%%%%%%
%Beamer Template Customization
%%%%%%%%%%
\setbeamertemplate{navigation symbols}{}
\setbeamertemplate{theorems}[ams style]
\setbeamertemplate{blocks}[rounded]

\definecolor{Blu}{RGB}{43,62,133} % UWEC Blue
\setbeamercolor{structure}{fg=Blu} % Titles

%Unnumbered footnotes:
\newcommand{\blfootnote}[1]{%
	\begingroup
	\renewcommand\thefootnote{}\footnote{#1}%
	\addtocounter{footnote}{-1}%
	\endgroup
}


%%%%%%%%%%
%Custom Commands
%%%%%%%%%%
\newcommand{\R}{\mathbb{R}}
\newcommand{\veca}{\vec{a}}
\newcommand{\vecb}{\vec{b}}
\newcommand{\vece}{\vec{e}}
\newcommand{\vecu}{\vec{u}}
\newcommand{\vecv}{\vec{v}}
\newcommand{\vecw}{\vec{w}}
\newcommand{\vecx}{\vec{x}}
\newcommand{\zerovector}{\vec{0}}

\newcommand{\ds}{\displaystyle}

\newcommand{\fn}{\insertframenumber}

\newcommand{\rank}{\operatorname{rank}}
\newcommand{\adj}{\operatorname{adj}}

\newcommand{\blank}[1]{\underline{\hspace*{#1}}}


%%%%%%%%%%
%Custom Theorem Environments
%%%%%%%%%%
\theoremstyle{definition}
\newtheorem{exercise}{Exercise}
\newtheorem{question}[exercise]{Question}
\newtheorem*{defn}{Definition}
\newtheorem*{exa}{Example}
\newtheorem*{disc}{Group Discussion}
\newtheorem*{nb}{Note}
\newtheorem*{recall}{Recall}
\renewcommand{\emph}[1]{{\color{blue}\texttt{#1}}}

\definecolor{Gold}{RGB}{237, 172, 26}
%Statement block
\newenvironment{statementblock}[1]{%
	\setbeamercolor{block body}{bg=Gold!20}
	\setbeamercolor{block title}{bg=Gold}
	\begin{block}{\textbf{#1.}}}{\end{block}}

\begin{document}
	\title{Math 324: Linear Algebra}
	\subtitle{2.2: Properties of Matrix Operations}
	\author{Mckenzie West}
	\date{Last Updated: \today}
\begin{frame}
\maketitle
\end{frame}

\begin{frame}{\insertframenumber}
	\begin{block}{\textbf{Last Time.}}
	\begin{itemize}[label=--]
		\item Row and Column Matrices
		\item Linear Combinations
		\item Linear Systems as Matrix Products
	\end{itemize}
	\end{block}
\begin{block}{\textbf{Today.}}
	\begin{itemize}[label=--]
		\item Properties of Matrix Addition
		\item Additive and Multiplicative Identities
		\item Properties of Matrix Multiplication
		\item Transposes and Properties of
	\end{itemize}
\end{block}
\end{frame}

\begin{frame}{\fn}
	\begin{statementblock}{Theorem 2.1}
		If $A$, $B$, and $C$ are $m\times n$ matrices, and $c$ and $d$ are scalars, then the following properties are true:
			\begin{enumerate}[label=\arabic*.]
				\item $A+B=B+A$\hfill \emph{Commutativity (of addition)}
				\item $A+(B+C)=(A+B)+C$ \hfill\emph{Associativity (of addition)}
				\item $(cd)A=c(dA)$\hfill\emph{Associativity (of scalar multiplication)}
				\item $1A=A$\hfill\emph{Scalar multiplication identity}
				\item $c(A+B)=cA+cB$\hfill\emph{Distributivity}
				\item $(c+d)A=cA+dA$\hfill\emph{Distributivity}
			\end{enumerate}
	\end{statementblock}
\end{frame}

\begin{frame}{\fn}
	\begin{exercise}
		Let $A=\begin{bmatrix}
		3&-2&4\\0&1&5
		\end{bmatrix}$
		$c=2$ and $d=-1$.
		
		Carefully work through the left-hand side and right-hand side of property (6) of Theorem 2.1. 
		
		What is the significance of this part of the Theorem?
	\end{exercise}\pause
	\begin{nb}
		Theorem 2.1 feels intuitive but it is absolutely necessary that we write it down and verify it.
	\end{nb}
\end{frame}

\begin{frame}{\fn}
	When adding, we want to have the option to add ``zero'' though what we mean by zero depends on how addition is performed.
	
	\begin{defn}
		A \emph{zero matrix}, or \emph{additive identity} for the set of $m\times n$ matrices is the $m\times n$ matrix that consists entirely of zeros, denoted by $O_{mn}$.
	\end{defn}
\end{frame}

\begin{frame}{\fn}
	\begin{statementblock}{Theorem 2.2}
		If $A$ is an $m\times n$ matrix and $c$ is a scalar, then the following are true.
		\begin{enumerate}[label=\arabic*.]
			\item $A+O_{mn}=A$
			\item $A+(-A)=O_{mn}$
			\item If $cA=O_{mn}$ then $c=0$ or $A=O_{mn}$.
		\end{enumerate}
	\end{statementblock}
	\begin{exercise}
		Use Theorems 2.1 and 2.2 to solve for the matrix $X$ in terms of the matrices $A$ and $B$ if $3A-2X=B$.
		
		Carefully list what properties of Theorems 2.1 and 2.2 you are using when you use them.
		\pause
		
		(Hint: You might use properties (1) and (2) of Thm 2.2, and properties (3), (4), and (5) of Thm 2.1).
	\end{exercise}
\end{frame}

\begin{frame}{\fn}
	\begin{statementblock}{Theorem 2.3}
		If $A$, $B$, and $C$ are matrices of appropriate size and $c$ is a scalar then the following are true
		\begin{enumerate}[label=\arabic*.]
			\item $A(BC)=(AB)C$\hfill \emph{Associativity (of multiplication)}
			\item $A(B+C)=AB+AC$\hfill \emph{Distributivity}
			\item $(A+B)C=AC+BC$\hfill \emph{Distributivity}
			\item $c(AB)=(cA)B=A(cB)$
		\end{enumerate}
	\end{statementblock}
	\begin{exercise}
		Let $A=\begin{bmatrix}
			1&2\\0&1
		\end{bmatrix}$, 
		$B=\begin{bmatrix}
		1&2&3\\-1&0&1
		\end{bmatrix}$, and 
		$C=\begin{bmatrix}
		0\\1\\4
		\end{bmatrix}$.
		
		Have some people at the table compute $A(BC)$ with the others compute $(AB)C$.
	\end{exercise}
\end{frame}

\begin{frame}{\fn}
	\begin{block}{\textbf{Warnings.}}
		\begin{itemize}[label=--]
			\item 		Commutativity does not apply for matrices.  In general \[AB\neq BA.\]
			\item	We can't divide by matrices.  It is possible for 
				\[AB=AC \textup{ but } B\neq C.\]
		\end{itemize}
	\end{block}
\end{frame}

\begin{frame}{\fn}
	\begin{defn}
		The \emph{identity matrix of order $n$} is the $n\times n$ matrix with 1's on the diagonal and zeros everywhere else.  Denote it by $I_n$.
	\end{defn}
	\begin{exa}
		\[I_4=\begin{bmatrix}
			1&0&0&0\\
			0&1&0&0\\
			0&0&1&0\\
			0&0&0&1
		\end{bmatrix}\]
	\end{exa}
	\begin{statementblock}{Theorem 2.4}
		If $A$ is a matrix of size $m\times n$ then 
			\begin{enumerate}[label=\arabic*.]
				\item $AI_n=A$
				\item $I_m A=A$
			\end{enumerate}
	\end{statementblock}
\end{frame}

\begin{frame}{\fn}
	\begin{exercise}
		Compute\[I_3\begin{bmatrix}
			3&4\\7&9\\-5&2
		\end{bmatrix}\]
	\end{exercise}
	\pause
	\begin{exercise}
		Compute
		$\begin{bmatrix}
		6&2\\13&1
		\end{bmatrix}
		\begin{bmatrix}
		2&0\\0&-1
		\end{bmatrix}$, and find a pattern.
		
		Without going through the steps of multiplication, determine
		$$\begin{bmatrix}
		-2&3\\6&5
		\end{bmatrix}
		\begin{bmatrix}
		2&0\\0&-1
		\end{bmatrix}.$$
	\end{exercise}
\end{frame}

\begin{frame}{\fn}
	\begin{block}{\textbf{Brain Break}}
		What book/movie/TV show would you change the ending to if you could?
		\begin{center}
		\includegraphics[width=2in]{../images/end}
		\end{center}
	\end{block}
\end{frame}
\begin{frame}{\fn}
	\begin{defn}
		If $k$ is a positive integer and $A$ is a square matrix, define the \emph{$k$th power of $A$} by
		\[A^k=AA\cdots A\]
		multiplying $A$ by itself $k$ times.
		
		Define $A^0=I$.
		
	\end{defn}
	\begin{question}
		For positive integers $j$ and $k$, is $$A^{j+k}=(A^j)(A^k)?$$ 
		
		Is $$A^{jk}=(A^j)^k=(A^k)^j?$$
	\end{question}
\end{frame}

\begin{frame}{\fn}
	\begin{defn}
		The \emph{transpose} of an $m\times n$ matrix $A$ is the $n\times m$ matrix $A^T$ whose rows are the columns of $A$ in the same order.
	\end{defn}
	\begin{exercise}
		Find the transpose of each matrix
		\begin{center}
			$\begin{bmatrix}
		-1&13
		\end{bmatrix}$\hskip .5in
		$\begin{bmatrix}
		2&3\\
		9&7\\
		-3&8
		\end{bmatrix}$\hskip .5in
		$\begin{bmatrix}
		1&3&-1\\
		3&2&-5\\
		-1&-5&1
		\end{bmatrix}
		$
		\end{center}
	\end{exercise}
	\begin{defn}
		If $A^T=A$, we call $A$ \emph{symmetric}.
	\end{defn}
\end{frame}
\begin{frame}{\fn}
	\begin{statementblock}{Theorem 2.5}
		If $A$ and $B$ are matrices of appropriate size and $c$ is a scalar, then the following are true
		\begin{enumerate}[label=\arabic*.]
			\item $(A^T)^T=A$
			\item $(A+B)^T=A^T+B^T$
			\item $(cA)^T=cA^T$
			\item $(AB)^T=B^TA^T$
		\end{enumerate}
	\end{statementblock}
	\begin{exercise}
		Verify that $(AB)^T=B^TA^T$ for
			\[
			A=\begin{bmatrix}
			-1&1&-2\\2&0&1
			\end{bmatrix}\textup{ and }
			B=\begin{bmatrix}
			-3&0\\1&2\\1&-1
			\end{bmatrix}
			\]
	\end{exercise}
\end{frame}
\begin{frame}{\fn}
	\begin{exercise}
		Compute $AA^T$, what do you notice?
	\[A=\begin{bmatrix}
		4&2&1\\0&2&-1
	\end{bmatrix}\]
	\end{exercise}
	\begin{exercise}
		Compute $B+B^T$, what do you notice?
			\[B=\begin{bmatrix}
			3&-2\\4&1
			\end{bmatrix}\]
	\end{exercise}
\end{frame}

\begin{frame}{\fn}
	\begin{exercise}
		Use Theorem 2.5 (and commutativity of addition) to verify that $A+A^T$ is symmetric for all $n\times n$ matrices $A$. Also verify that $cA$ is symmetric if $A$ is symmetric.
	\end{exercise}
	\begin{defn}
		A $n\times n$ matrix $B$ is called \emph{skew symmetric} if $B^T=-B$.
	\end{defn}
	\begin{exercise}
		Use Theorem 2.5 (and 2.1) to verify that $A-A^T$ is skew symmetric for all $n\times n$ matrices $A$.
	\end{exercise}
\end{frame}
\begin{frame}{\fn}
	\begin{exercise}
		Use the previous two exercises to write the following matrix as the sum of a symmetric matrix and a skew symmetric matrix: \[A=\begin{bmatrix}
		1&3&-2\\4&0&13\\-1&2&3
		\end{bmatrix}\]
	\end{exercise}
\end{frame}

\begin{frame}{\fn}
	\begin{defn}
		The \emph{trace} of an $n\times n$ matrix $A$ is the sum:	\[\operatorname{Tr}(A)=a_{11}+a_{22}+\cdots+a_{nn}.\]
	\end{defn}
	\begin{exercise}
		Compute $\operatorname{Tr}(I_7)$ and  $\operatorname{Tr}\left(\begin{bmatrix}
		3&-2&4\\0&2&1\\-3&15&3\end{bmatrix}\right)$
	\end{exercise}

	\begin{exercise}
		Suppose $A$ and $B$ are $n\times n$ matrices and $c$ a scalar.
		\begin{enumerate}[label=(\alph*)]
			\item How does $\operatorname{Tr}(cA)$ compare to $\operatorname{Tr}(A)$?
			\item Is it true that $\operatorname{Tr}(A+B)=\operatorname{Tr}(A)+\operatorname{Tr}(B)$?
			\item Is it true that $\operatorname{Tr}(AB)=\operatorname{Tr}(A)\operatorname{Tr}(B)$?
		\end{enumerate}
	\end{exercise}
\end{frame}

\begin{frame}{\fn}
\begin{exercise}
	Consider square matrices of the form
	\[A=\begin{bmatrix}
	0&a_{12}&a_{13}&\dots&a_{1n}\\
	0&0&a_{23}&\dots &a_{2n}\\
	\vdots&\vdots&\vdots&&\vdots\\
	0&0&0&\dots &a_{(n-1)n}\\
	0&0&0&\dots&0
	\end{bmatrix}\]
	\begin{enumerate}[label=(\alph*)]
		\item Write a $2\times 2$ and a $3\times 3$ matrix in the form of $A$.
		\item Use Sage or a calculator to square, cube, fourth power,... each of the matrices. Describe the result.
		\item Make a guess at what will happen for a $4\times 4$ matrix of this form as you raise it to powers.  Test your guess using technology.
		\item Make a conjecture about the powers of $A$ when $A$ is an $n\times n$ matrix.
	\end{enumerate}
\end{exercise}
\end{frame}
\end{document}

