\documentclass[handout]{beamer}
%\usepackage[margin=1in]{geometry}
\usepackage{amsthm,amsmath,amsfonts,hyperref,graphicx,color,multicol}
\usepackage{enumitem,tikz}

%%%%%%%%%%
%Beamer Template Customization
%%%%%%%%%%
\setbeamertemplate{navigation symbols}{}
\setbeamertemplate{theorems}[ams style]
\setbeamertemplate{blocks}[rounded]

\definecolor{Blu}{RGB}{43,62,133} % UWEC Blue
\setbeamercolor{structure}{fg=Blu} % Titles

%Unnumbered footnotes:
\newcommand{\blfootnote}[1]{%
	\begingroup
	\renewcommand\thefootnote{}\footnote{#1}%
	\addtocounter{footnote}{-1}%
	\endgroup
}


%%%%%%%%%%
%Custom Commands
%%%%%%%%%%
\newcommand{\R}{\mathbb{R}}
\newcommand{\veca}{\vec{a}}
\newcommand{\vecb}{\vec{b}}
\newcommand{\vece}{\vec{e}}
\newcommand{\vecu}{\vec{u}}
\newcommand{\vecv}{\vec{v}}
\newcommand{\vecw}{\vec{w}}
\newcommand{\vecx}{\vec{x}}
\newcommand{\zerovector}{\vec{0}}

\newcommand{\ds}{\displaystyle}

\newcommand{\fn}{\insertframenumber}

\newcommand{\rank}{\operatorname{rank}}
\newcommand{\adj}{\operatorname{adj}}

\newcommand{\blank}[1]{\underline{\hspace*{#1}}}


%%%%%%%%%%
%Custom Theorem Environments
%%%%%%%%%%
\theoremstyle{definition}
\newtheorem{exercise}{Exercise}
\newtheorem{question}[exercise]{Question}
\newtheorem*{defn}{Definition}
\newtheorem*{exa}{Example}
\newtheorem*{disc}{Group Discussion}
\newtheorem*{nb}{Note}
\newtheorem*{recall}{Recall}
\renewcommand{\emph}[1]{{\color{blue}\texttt{#1}}}

\definecolor{Gold}{RGB}{237, 172, 26}
%Statement block
\newenvironment{statementblock}[1]{%
	\setbeamercolor{block body}{bg=Gold!20}
	\setbeamercolor{block title}{bg=Gold}
	\begin{block}{\textbf{#1.}}}{\end{block}}

\begin{document}
	\title{Math 324: Linear Algebra}
	\subtitle{2.3: The Inverse of a Matrix}
	\author{Mckenzie West}
	\date{Last Updated: \today}
\begin{frame}
\maketitle
\end{frame}

\begin{frame}{\insertframenumber}
	\begin{block}{\textbf{Last Time.}}
	\begin{itemize}[label=--]
		\item Inverses of Matrices
		\item Gauss-Jordan Elimination
		\item Inverses of $2\times2$ matrices
	\end{itemize}
	\end{block}
\begin{block}{\textbf{Today.}}
	\begin{itemize}[label=--]
		\item Properties of Inverses
		\item Using Inverses to Solve System of Equations
	\end{itemize}
\end{block}
\end{frame}

\begin{frame}{\fn}
	\begin{recall}
		For a $2\times 2$ matrix $A=\begin{bmatrix}
		a&b\\c&d
		\end{bmatrix}$, $A$ is invertible if and only if $ad-bc\neq 0$.  If $A$ is invertible, then
			\[A^{-1}=\frac{1}{ad-bc}\begin{bmatrix}
			d&-b\\-c&a
			\end{bmatrix}.\]
	\end{recall}
	\begin{nb}
		For the exercises today, compute the inverses of all $2\times 2 $ matrices by hand.
		
		You may use a calculator or Sage to compute the inverse of the larger matrices.  Recall the process of $[A|I]\rightarrow[I|A^{-1}]$.
	\end{nb}
\end{frame}
\begin{frame}{\fn}
	\begin{exercise}[Warm-up]
		Compute the inverse of the matrix if it exists:
			\begin{enumerate}[label=(\alph*)]
				\item $\begin{bmatrix}-1&-3\\2&4\end{bmatrix}$
				\item $\begin{bmatrix}4&-2\\1&1\end{bmatrix}$
			\end{enumerate}
	\end{exercise}
	\begin{exercise}
	Find all $x$ and $y$ that make the matrix singular
	$\begin{bmatrix}
	-x&2\\y&5
	\end{bmatrix}$
	Remember to parametrize your solutions.
	\end{exercise}
	\begin{exercise}
		Find $x$ so that $A^{-1}=A^2$ for $A=\left[\begin{array}{rr}
		-2 & -3 \\
		x & 1
		\end{array}\right]$
		
		(Hint: This means $A(A^2)=I$.)
	\end{exercise}
\end{frame}

\begin{frame}{\fn}
	\begin{statementblock}{Theorem 2.8}
			If $A$ is an invertible matrix, $k$ is a positive integer, and $c$ is a nonzero scalar, then $A^{-1}$, $A^k$, $cA$, and $A^T$ are invertible and the following are true:
			\begin{enumerate}[label=\arabic*.]
				\item $(A^{-1})^{-1}=A$
				\item $(A^k)^{-1}=(A^{-1})^k$
				\item $(cA)^{-1}=\frac{1}{c}A^{-1}$
				\item $(A^T)^{-1}=(A^{-1})^T$
			\end{enumerate}
	\end{statementblock}
\end{frame}
\begin{frame}{\fn}
	\begin{exercise}
		Verify property (1) of Theorem 2.8 by computing $A^{-1}$ and $(A^{-1})^{-1}$ for $A=\left[\begin{array}{rr}
		1 & -9 \\
		-3 & 5
		\end{array}\right]$.
	\end{exercise}
	\begin{exercise}
		Verify property (3) of Theorem 2.8 by computing $(2A)^{-1}$ and $\frac{1}{2}(A^{-1})$ for $A=\left[\begin{array}{rr}
		1 & -5 \\
		-3 & 10
		\end{array}\right]$.
	\end{exercise}
\end{frame}
\begin{frame}{\fn}
	\begin{exercise}
		Follow the steps to verify a simplified version of property (2) of Theorem 2.8.
		
		\begin{statementblock}{Claim}
		If $A$ is an invertible matrix then $A^3$ is invertible and $(A^3)^{-1}=(A^{-1})^3$.
		\end{statementblock}
		\begin{enumerate}[label=(\alph*)]
			\item Remind yourself of the definition of an inverse.
			\item To verify the result, begin by computing $A^3B$ and $BA^3$ for $B=(A^{-1})^3$.  Carefully note the location of the exponents here.
			\item If the two product equal $I$, then you can conclude that the inverse of $A^3$ is $(A^{-1})^3$.
		\end{enumerate}
	\end{exercise}
\end{frame}

\begin{frame}{\fn}
	\begin{block}{\textbf{Brain Break.}}
		What is your favorite zoo animal?\begin{center}
			\includegraphics[width=2in]{../images/zoo}
		\end{center}
	\end{block}
\end{frame}
\begin{frame}{\fn}
	\begin{statementblock}{Theorem 2.9}
		If $A$ and $B$ are invertible matrices of order $n$ then $AB$ is invertible and $(AB)^{-1}=B^{-1}A^{-1}$.
	\end{statementblock}
	\begin{nb}
		ORDER IS INCREDIBLY IMPORTANT.
	\end{nb}
\end{frame}

\begin{frame}{\fn}
	\begin{exercise}
	Let \[A=\left[\begin{array}{rrr}
	-2 & 1 & -2 \\
	-1 & 2 & -3 \\
	1 & -1 & 2
	\end{array}\right]\textup{ and }B=\left[\begin{array}{rrr}
	1 & 0 & -2 \\
	1 & 1 & 2 \\
	-3 & -1 & 3
	\end{array}\right].\]
	
	Use the \textit{Gauss-Jordan method} to compute $A^{-1}$ and $B^{-1}$.  
	
	Multiply these in the right order to compute $(AB)^{-1}$.
	
	*Use Sage to row reduce and verify that you actually did get $(AB)^{-1}$.
	
	**Recall the \textit{Gauss-Jordan method} from last time: if $A$ is invertible, $[A|I]$ row reduces to $[I|A^{-1}]$.
\end{exercise}
\end{frame}
\begin{frame}{\fn}
	\begin{statementblock}{Theorem 2.10}
		If $C$ is an invertible matrix, then the following properties hold:	
			\begin{enumerate}[label=\arabic*.]
				\item If $AC=BC$, then $A=B$ \hfill\emph{Right cancellation}
				\item If $CA=CB$, then $A=B$\hfill\emph{Left cancellation}
			\end{enumerate}
	\end{statementblock}
	\begin{exercise}
		Suppose $A$ is invertible and $A^2=A$, what must be true about $A$? (Hint: Write the equation as $AA=AI$.)
		
		Is the same true for invertible matrices with $A^3=A$?
	\end{exercise}
\end{frame}

\begin{frame}{\fn}
	
	\begin{nb}
		ORDER IS INCREDIBLY IMPORTANT.
	\end{nb}
	\begin{exercise}
		Consider 
		$A=\begin{bmatrix}
		-6 & 3 \\
		-4 & 3
		\end{bmatrix}$, 
		$B = \begin{bmatrix}
		0 & -3 \\
		-2 & -3
		\end{bmatrix}$, and
		$C=\begin{bmatrix}
		-1 & 0 \\
		-2 & 1
		\end{bmatrix}$.
		
		Verify that although $A\neq B$, $AC=CB$.
	\end{exercise}
\end{frame}

\begin{frame}{\fn}
\begin{recall}
	We can write a system of linear equations as the matrix product $A\vecx=\vecb$, where 
	\begin{itemize}[label=--]
		\item $A$ is the coefficient matrix;
		\item $\vecx$ is the column vector consisting of the variables;
		\item and $\vecb$ is the column vector consisting of the constant terms.
	\end{itemize}
\end{recall}
\begin{statementblock}{Theorem 2.11}
	If $A$ is an invertible matrix, then the system of linear equations $A\vecx=\vecb$ has a unique solution given by $\vecx=A^{-1}\vecb$.
\end{statementblock}
\end{frame}
\begin{frame}{\fn}
\begin{exercise}
	Use an inverse matrix to solve each system of linear equations:
	\begin{enumerate}[label=(\alph*)]
		\item $\begin{array}{rcrcrcr}
		x_1&+&2x_2&+&x_3&=&2\\
		x_1&+&2x_2&-&x_3&=&4\\
		x_1&-&2x_2&+&x_3&=&-2
		\end{array}$
		\item $\begin{array}{rcrcrcr}
		x_1&+&2x_2&+&x_3&=&1\\
		x_1&+&2x_2&-&x_3&=&3\\
		x_1&-&2x_2&+&x_3&=&-3
		\end{array}$
	\end{enumerate}
\end{exercise}
\end{frame}

\begin{frame}{\fn}
	\begin{exercise}
		Use a matrix inverse to solve the following.
		
		A group took a trip to the zoo on a bus, at \$3 per child and \$3.20 per adult for a total of \$118.40.
		
		They took the train back from the zoo at \$3.50 per child and \$3.60 per adult for a total of \$135.20.
		
		How many children, and how many adults?
	\end{exercise}
\end{frame}

\begin{frame}{\fn}
	\begin{exercise}
		Compute the inverse of each of the following matrices, if it exists. (Use a calculator or Sage.)
		\begin{multicols}{3}
			\begin{enumerate}[label=(\alph*)]
			\item $\begin{bmatrix}
		4 & 0 & 2 \\
		0 & -1 & -4 \\
		0 & 0 & 1
		\end{bmatrix}$
			\item $\begin{bmatrix}
			-5 & -2 & 0 \\
			0 & 1 & 0 \\
			0 & 0 & 4
			\end{bmatrix}$
			\item $\begin{bmatrix}
			-3 & 4 & 3 \\
			0 & 0 & 5 \\
			0 & 0 & 1
			\end{bmatrix}$
		\end{enumerate}
		\end{multicols}
		We call a matrix of this form \emph{upper triangular} because the only non-zero entries are in the upper right-hand triangle.
		
		Make a conjecture about the shape of the inverses of upper triangular matrices -- and when they exist.
	\end{exercise}
\end{frame}
%\begin{frame}{\fn}
%	\begin{exercise}
%		Let $A=\left[\begin{array}{rr}
%		-3 & -2 \\
%		3 & -1
%		\end{array}\right]$.
%		\begin{enumerate}[label=(\alph*)]
%			\item Verify that $A^2+4A+9I=O$.
%			\item Verify that $A^{-1}=\frac{-1}{9}(A+4I)$.
%			\item Show that for any square matrix satisfying $A^2+4A+9I=O$, the inverse of $A$ is given by $A^{-1}=\frac{-1}{9}(A+4I)$.
%		\end{enumerate}
%	\end{exercise}
%\end{frame}
\end{document}

 