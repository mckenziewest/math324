\documentclass[handout]{beamer}
%\usepackage[margin=1in]{geometry}
\usepackage{amsthm,amsmath,amsfonts,hyperref,graphicx,color,multicol}
\usepackage{enumitem,tikz}

%%%%%%%%%%
%Beamer Template Customization
%%%%%%%%%%
\setbeamertemplate{navigation symbols}{}
\setbeamertemplate{theorems}[ams style]
\setbeamertemplate{blocks}[rounded]

\definecolor{Blu}{RGB}{43,62,133} % UWEC Blue
\setbeamercolor{structure}{fg=Blu} % Titles

%Unnumbered footnotes:
\newcommand{\blfootnote}[1]{%
	\begingroup
	\renewcommand\thefootnote{}\footnote{#1}%
	\addtocounter{footnote}{-1}%
	\endgroup
}


%%%%%%%%%%
%Custom Commands
%%%%%%%%%%
\newcommand{\R}{\mathbb{R}}
\newcommand{\veca}{\vec{a}}
\newcommand{\vecb}{\vec{b}}
\newcommand{\vece}{\vec{e}}
\newcommand{\vecu}{\vec{u}}
\newcommand{\vecv}{\vec{v}}
\newcommand{\vecw}{\vec{w}}
\newcommand{\vecx}{\vec{x}}
\newcommand{\zerovector}{\vec{0}}

\newcommand{\ds}{\displaystyle}

\newcommand{\fn}{\insertframenumber}

\newcommand{\rank}{\operatorname{rank}}
\newcommand{\adj}{\operatorname{adj}}

\newcommand{\blank}[1]{\underline{\hspace*{#1}}}


%%%%%%%%%%
%Custom Theorem Environments
%%%%%%%%%%
\theoremstyle{definition}
\newtheorem{exercise}{Exercise}
\newtheorem{question}[exercise]{Question}
\newtheorem*{defn}{Definition}
\newtheorem*{exa}{Example}
\newtheorem*{disc}{Group Discussion}
\newtheorem*{nb}{Note}
\newtheorem*{recall}{Recall}
\renewcommand{\emph}[1]{{\color{blue}\texttt{#1}}}

\definecolor{Gold}{RGB}{237, 172, 26}
%Statement block
\newenvironment{statementblock}[1]{%
	\setbeamercolor{block body}{bg=Gold!20}
	\setbeamercolor{block title}{bg=Gold}
	\begin{block}{\textbf{#1.}}}{\end{block}}


\begin{document}
	\title{Math 324: Linear Algebra}
	\subtitle{2.1: Operations with Matrices}
	\author{Mckenzie West}
	\date{Last Updated: \today}
\begin{frame}
\maketitle
\end{frame}

\begin{frame}{\insertframenumber}
	\begin{block}{\textbf{Last Time.}}
	\begin{itemize}[label=--]
		\item Polynomial Curve Fitting
		\item Network Analysis
	\end{itemize}
	\end{block}
\begin{block}{\textbf{Today.}}
	\begin{itemize}[label=--]
		\item Matrix Equality
		\item Matrix Addition, Subtraction and Scalar Multiplication
		\item Matrix Multiplication
	\end{itemize}
\end{block}
\end{frame}

\begin{frame}{\insertframenumber}
	We begin by recalling some matrix notation:
		\begin{itemize}[label=--]
			\item An $m\times n$ matrix is a grid with $m$ rows and $n$ columns.
			\item Matrices are usually denoted using capital letters.
			\item The $i,j$-entry of a matrix $A$ is the entry in the $i$-th row and $j$-th column.  It may also be denoted by $a_{ij}$.
			\item Other ways to denote a matrix include a compact version, $A=[a_{ij}]$, and a rectangular array,
				\[A=\begin{bmatrix}
				a_{11}&a_{12}&a_{13}&\dots&a_{1n}\\
				a_{21}&a_{22}&a_{23}&\dots&a_{2n}\\
				a_{31}&a_{32}&a_{33}&\dots&a_{3n}\\
				\vdots&\vdots&\vdots&&\vdots\\
				a_{m1}&a_{m2}&a_{m3}&\dots&a_{mn}\\
				\end{bmatrix}.\]
			Which one we use may depend on the circumstance.
		\end{itemize}
\end{frame}

\begin{frame}{\fn}
	\begin{defn}
		Two matrices $A=[a_{ij}]$ and $B=[b_{ij}]$ are \emph{equal} if they have the same size ($m\times n$) and $a_{ij}=b_{ij}$ for all $1\leq i\leq m$ and $1\leq j\leq n$.
	\end{defn}	
	\begin{exercise}
		Find $x$ and $y$:
			\[\begin{bmatrix}
				-5 & x\\3y-2 & 8
			\end{bmatrix}=
			\begin{bmatrix}
			-5 & 7\\13 & 8
			\end{bmatrix}
			\]
	\end{exercise}
	\begin{exercise}
		Are the following matrices equal?
		\[\begin{bmatrix}
		54 \\ -654
		\end{bmatrix}=
		\begin{bmatrix}
		54 & -654
		\end{bmatrix}
		\]
	\end{exercise}
\end{frame}

\begin{frame}{\fn}
	\begin{defn}
		The \emph{sum} of two $m\times n$ matrices $A=[a_{ij}]$ and $B=[b_{ij}]$ is the $m\times n$ matrix $A+B=[a_{ij}+b_{ij}]$. 
	\end{defn}
	\begin{nb}
		We're using the fact that we know how to add real numbers in order to add matrices.
	\end{nb}
	\begin{defn}
		If $A=[a_{ij}]$ is an $m\times n$ matrix and $c$ is a real number, then the \emph{scalar multiple} of $A$ by $c$ is the $m\times n$ matrix given by $cA=[ca_{ij}]$.
	\end{defn}
\end{frame}

\begin{frame}{\fn}
	\begin{exercise}
		Solve for $x,y$ and $z$ in the matrix equation:	
		\[
			4\begin{bmatrix}
			x&y\\z&-1
			\end{bmatrix}
			=
			2\begin{bmatrix}
			y&z\\-x&1
			\end{bmatrix}
			+
			2\begin{bmatrix}
			4&x\\5&-x
			\end{bmatrix}
		\]
		
		Hint: Simplify the right-hand and left-hand sides using scalar multiplication and matrix addition, then remember back to what it means for two matrices to be equal.
	\end{exercise}
	\pause
	\begin{question}
		What does it mean, in terms of addition and scalar multiplication, to subtract one matrix from another?
	\end{question}
\end{frame}

\begin{frame}{\fn}
	\begin{block}{\textbf{Brain Break.}}
		Remind you group of your name.
		
		What would you name your boat if you owned one?
		
		\begin{center}
			\includegraphics[width=2in]{images/boat}
		\end{center}
	\end{block}
\end{frame}
\begin{frame}{\fn}
	\begin{defn}[\emph{Matrix Multiplication}]
		If $A=[a_{ij}]$ is an $m\times n$ matrix and $B=[b_{ij}]$ is an $n\times p$ matrix, then the \emph{product} of $A$ and $B$ is an $m\times p$ matrix 
		\[AB=[c_{ij}],\]
		where
		\[c_{ij}=a_{i1}b_{1j}+a_{i2}b_{2j}+\cdots+a_{in}b_{nj}=\sum_{k=1}^n a_{ik}b_{kj}.\]
	\end{defn}
\end{frame}

\begin{frame}{\fn}
	\begin{exercise}
		We begin by asking which products we can take.  Go back to the definition and make careful note about the size of each matrix.
	\end{exercise}
	\pause
	\begin{exercise}
		Here are six matrices.  List several products that we can take along with the size of the resulting matrix.  Then list several products that are not allowed.
		\[A=\begin{bmatrix}2&6\\-3&-7\end{bmatrix}\
		B=\begin{bmatrix}0&1&-1\\3&-1&1\end{bmatrix}\
		C=\begin{bmatrix}-1&7\\0&3\\-1&2\end{bmatrix}
		\]
		\[
		D=\begin{bmatrix}1&-1\\-3&5\end{bmatrix}\
		E=\begin{bmatrix}5\\4\end{bmatrix}\
		F=\begin{bmatrix}-1&1\end{bmatrix}
		\]
	\end{exercise}
\end{frame}

\begin{frame}{\fn}
	\begin{exercise}
		Compute $CA$, where
		$A=\begin{bmatrix}2&6\\-3&-7\end{bmatrix}\textup{ and }
		C=\begin{bmatrix}-1&7\\0&3\\-1&2\end{bmatrix}.$\vskip 3in
	\end{exercise}
	\begin{exercise}
		Compute $DE$, where
		$D=\begin{bmatrix}1&-1\\-3&5\end{bmatrix}\textup{ and }
		E=\begin{bmatrix}5\\4\end{bmatrix}.$\vskip 3in
	\end{exercise}
	\begin{exercise}
		Compute $AD$, where
		$A=\begin{bmatrix}2&6\\-3&-7\end{bmatrix}\textup{ and }
		D=\begin{bmatrix}1&-1\\-3&5\end{bmatrix}$.
	\end{exercise}
\end{frame}

\begin{frame}{\fn}
	\begin{exercise}
		Let $A$ be a matrix such that 
		$\begin{bmatrix}
		1&2\\3&5
		\end{bmatrix}A=
		\begin{bmatrix}
		1&0\\0&1
		\end{bmatrix}$
		\begin{enumerate}[label=(\alph*)]
			\item What is the size of $A$?\pause
			\item Using variables as placeholders for the entries of $A$, multiply the matrices on the left-hand side of the equation.\pause
			\item Set the entries on the LHS equal to the corresponding entries on the RHS to create a system of linear equations.\pause
			\item Solve the system of linear equations to find $A$.
		\end{enumerate}
	\end{exercise}
\end{frame}


\begin{frame}{\fn}
	\begin{exercise}
		Let $B$ be a matrix such that 
		$B\begin{bmatrix}
		-3&0\\1&2
		\end{bmatrix}=
		\begin{bmatrix}
		1&0\\-1&0\\0&1
		\end{bmatrix}$
		\begin{enumerate}[label=(\alph*)]
			\item What is the size of $B$?\pause
			\item Using variables as placeholders for the entries of $B$, multiply the matrices on the left-hand side of the equation.\pause
			\item Set the entries on the LHS equal to the corresponding entries on the RHS to create a system of linear equations.\pause
			\item Solve the system of linear equations to find $B$.
		\end{enumerate}
	\end{exercise}
\end{frame}

\end{document}

